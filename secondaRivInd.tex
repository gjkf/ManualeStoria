%!TEX ROOT=storia.tex

\section{Seconda rivoluzione industriale}
La seconda rivoluzione industriale non ha dei limiti temporali definiti. La si può indicativamente
far andare dal \textbf{1870} al \textbf{1914} circa.\\
Una delle invenzioni che hanno caratterizzato questo periodo è stata quella del \textbf{motore
elettrico} e quella del \textbf{motore a scoppio}. Di conseguenza sono nate \textbf{dinamo},
\textbf{lampadine}, \textbf{aerei}, \textbf{telefoni} e \textbf{radio}.\\
A queste innovazioni si collega la nascita di molte industrie e aziende che producevano e sostenevano
queste innovazioni. Le più importanti furono aziende \textbf{chimiche}, \textbf{siderurgiche} ed
\textbf{elettriche}. Gli \emph{Stati Uniti} e la \emph{Germania} erano le più innovative nazioni,
superando persino l'\emph{Inghilterra} che però deteneva ancora il primato finanziario. Le altre 
nazioni stanno piano piano intraprendendo la strada dell'innovazione, l'Italia avrà il suo boom a 
fine '800.\\ [\baselineskip]
Il \textbf{Giappone} sta anch'esso industrializzandosi a poco a poco. Lì, è lo stato che decide di 
avere la stessa potenza dei paesi europei. Quindi lo stato invia ``spie'' a verificare cosa si fa
in Europa e il Giappone copia, e copia bene.\\
Anche l'\textbf{agricoltura} si comincia a modernizzare con l'uso di concimi chimici e macchine
agricole.\\ [\baselineskip]
Si cominciano a completare \textbf{reti ferroviarie} con locomotive a vapore che diventano elettiche,
acciaio per i binari. Viene inventata la \textbf{turbina} e l'\textbf{elica} e tutta la navigazione
diventa a motore, più sicura e rapida con costi minori. Ciò rendeva più conveniente i cibi americani
di quelli europei e si sviluppava la concorrenza. Così si cominciano anche a studiare metodi di
conservazione delle derrate alimentari.\\ [\baselineskip]
Lo sviluppo provoca una \textbf{forte deflazione} in quanto per la stessa domanda, l'offerta aumenta
considerevolemente. Viene questa definita la \textbf{\textit{Grande Depressione}}. Si sono attuate
3 diverse politiche per contrarstare questo fenomeno:
\begin{description}
  \item[Protezionismo] Gli imprenditori premono sui governi per aggiungere dazi e proteggere 
    l'industria interna. Nel 1873 la Germania introduce le prime tariffe, poi gli altri paesi si
    adegueranno. Da qui in poi lo stato interverrà sempre di più nella vita economica
  \item[Trust, cartelli e concentrazioni industriali] Si vengono a formare aziende frutto di fusione
    di altre più piccole
    \begin{description}
      \item[Cartelli] Accordi tra aziende che producono lo stesso bene per non farsi o ridurre la
        concorrenza (prezzi fissi, scelte di zone di vendita, \ldots). Genera prezzi più alti
      \item[Trust] Unione di aziende
        \begin{description}
          \item[Orizzontali] Che producono un bene e accorpano altre aziende del settore
          \item[Verticali] Che vanno dalla materia prima al bene finito. Sono le prime multinazionali
        \end{description}
    \end{description}
  \item[Commissioni statali] Lo stato alimenta direttamente alcune zone d'industria
\end{description}
Cambia anche il \textbf{rapporto tra aziende e banche}. Le più grandi aziende erano S.P.A. ma i fondi
non erano sufficienti, quindi chiedono dei prestiti alle banche con cui si indebiteranno. Le banche 
acquistano azioni dalle aziende finanziandole e diventandone co-proprietarie come forma di garanzia. 
La distinzione banca-azienda si fa sempre più debole. I consumatori sono danneggiati dall'aumento
dei prezzi, quindi si creano delle \textbf{norme anti-trust}.\\ [\baselineskip]
In campo sociale, c'è stata un'enorme espansione demografica, gli abitanti in Europa sono più che
raddoppiati in un secolo. Questo ha provocato un'eccedenza di mano d'opera nelle campagne che a sua
volta ha portato a una forte \textbf{emigrazione} dall'Europa verso l'America.\\ [\baselineskip]
In questo periodo si va anche a formare il \textbf{Taylorismo} ovvero l'\emph{organizzazione 
scientifica del lavoro}. Bisogna rendere il lavoro il più efficiente possibile, per fare ciò lo
si deve dividere, specializzare il lavoro in lavori più semplici e particolari. Questo porterebbe
a vantaggi per lavoratori (con salari più alti) e agli imprenditori. I sindacati erano contrari in
quanto il \textbf{lavoro era alienante}. Nel \textbf{1911} Ford crea la prima \textbf{catena di
montaggio}. La produzione era in serie, tutti i prodotti uguali con il lavoro suddiviso. Diventerà
un modello. Le industrie vanno sempre più verso la produzione di massa.
