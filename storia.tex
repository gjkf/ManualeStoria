\documentclass[usenames,dvipsnames]{article}
\usepackage[italian]{babel}
\usepackage[latin1]{inputenc}
\usepackage{bm}
\usepackage{amsmath, amsfonts, amssymb, mathtools, calrsfs}
\usepackage{xfrac}
\usepackage[pdftex,dvipsnames]{xcolor}
\usepackage[colorinlistoftodos,prependcaption,textsize=tiny]{todonotes}
\usepackage{xtab, booktabs, array}
\usepackage[labelformat=empty]{caption}
\usepackage{tikz}
\usetikzlibrary{arrows,decorations,calc,intersections,shapes.geometric}
\usepackage{parskip} % To avoid indentation
\usepackage{textcomp} % To have \textcelsius and other symbols
\usepackage{multirow} % For nicer tables with split rows/columns
\usepackage{multicol}
\usepackage{cancel} % Cacelled equations and simplifications
\usepackage{hyperref}
\hypersetup{
    colorlinks,
    citecolor=black,
    filecolor=black,
    urlcolor=black,
    linkcolor=blue
}
\usepackage[all]{hypcap}

\usepackage[showframe=false, top=2cm, bottom=2.5cm, left=2.5cm, right=2.5cm]{geometry}

% Some useful TODO commands
\usepackage{xargs}
\newcommandx{\unsure}[2][1=]{\todo[linecolor=red,backgroundcolor=red!25,bordercolor=red,#1]{#2}}
\newcommandx{\change}[2][1=]{\todo[linecolor=blue,backgroundcolor=blue!25,bordercolor=blue,#1]{#2}}
\newcommandx{\info}[2][1=]{\todo[linecolor=OliveGreen,backgroundcolor=OliveGreen!25, bordercolor=OliveGreen,#1]{#2}}
\newcommandx{\improvement}[2][1=]{\todo[linecolor=Plum,backgroundcolor=Plum!25, bordercolor=Plum,#1]{#2}}
\newcommandx{\thiswillnotshow}[2][1=]{\todo[disable,#1]{#2}}
\setlength{\marginparwidth}{2cm}

% For an older but clearer root. Still \oldsqrt is valid
\usepackage{letltxmacro}
\makeatletter
\let\oldr@@t\r@@t
\def\r@@t#1#2{%
  \setbox0=\hbox{$\oldr@@t#1{#2\,}$}\dimen0=\ht0
  \advance\dimen0-0.2\ht0
  \setbox2=\hbox{\vrule height\ht0 depth -\dimen0}%
  {\box0\lower0.4pt\box2}}
\LetLtxMacro{\oldsqrt}{\sqrt}
\renewcommand*{\sqrt}[2][\ ]{\oldsqrt[#1]{#2} }
\makeatother

% Matrix spacing
\makeatletter
\renewcommand*\env@matrix[1][\arraystretch]{%
  \edef\arraystretch{#1}%
  \hskip -\arraycolsep
  \let\@ifnextchar\new@ifnextchar
  \array{*\c@MaxMatrixCols c}}
\makeatother

\newcommand\twodigits[1]{%
  \ifnum#1<10 0\number#1 \else #1\fi
}
\usepackage[yyyymmdd]{datetime}
\renewcommand{\dateseparator}{-}
\usepackage{fancyhdr}
\pagestyle{fancy}
\fancyhead{} % clear all header fields
\renewcommand{\headrulewidth}{0pt} % no line in header area
\fancyfoot{} % clear all footer fields
\fancyfoot[C,CO]{\thepage}% page number in "outer" position of footer line
\fancyfoot[R,RO]{Copyright \copyright 2017--\the\year$\,$Cossu Davide
} % other info in "inner" position of footer line
\fancyfoot[L,LO]{Version 0.0.2 \today
}

\DeclarePairedDelimiter\norm{\lVert}{\rVert} % ||v||  
\DeclarePairedDelimiter\abs{\lvert}{\rvert} % |v|
\newcommand\markangle[7][red]{% [color] origin A B radius radiusmark mark
  % fill red circle
  \begin{scope}
    \path[clip] (#2) -- (#3) -- (#4);
    \fill[color=#1,fill opacity=0.5,draw=#1,name path=circle]
    (#2) circle (#5);
  \end{scope}
  % middle calculation
  \path[name path=line one] (#2) -- (#3);
  \path[name path=line two] (#2) -- (#4);
  \path[%
    name intersections={of=line one and circle, by={inter one}},
    name intersections={of=line two and circle, by={inter two}}
  ] (inter one) -- (inter two) coordinate[pos=.5] (middle);
  % put mark
  \node at ($(#2)!#6!(middle)$) {#7};
}
\def\mathcolor#1#{\@mathcolor{#1}}
\def\@mathcolor#1#2#3{%
      \protect\leavevmode
      \begingroup
      \color#1{#2}#3%
      \endgroup
}
% For better visual in tables
\renewcommand*{\arraystretch}{2}
% To center with m{}
\newcolumntype{M}[1]{>{\centering\arraybackslash}m{#1}}
\newcommand{\divisor}{\rule{8.7cm}{0.4pt}}

\makeatletter
\newcommand*{\rom}[1]{\expandafter\@slowromancap\romannumeral #1@}
\makeatother

\begin{document}
{
\hypersetup{linkcolor=black}
\tableofcontents
}
%!TEX ROOT=storia.tex

\section{Seconda rivoluzione industriale}
La seconda rivoluzione industriale non ha dei limiti temporali definiti. La si può indicativamente
far andare dal \textbf{1870} al \textbf{1914} circa.\\
Una delle invenzioni che hanno caratterizzato questo periodo è stata quella del \textbf{motore
elettrico} e quella del \textbf{motore a scoppio}. Di conseguenza sono nate \textbf{dinamo},
\textbf{lampadine}, \textbf{aerei}, \textbf{telefoni} e \textbf{radio}.\\
A queste innovazioni si collega la nascita di molte industrie e aziende che producevano e sostenevano
queste innovazioni. Le più importanti furono aziende \textbf{chimiche}, \textbf{siderurgiche} ed
\textbf{elettriche}. Gli \emph{Stati Uniti} e la \emph{Germania} erano le più innovative nazioni,
superando persino l'\emph{Inghilterra} che però deteneva ancora il primato finanziario. Le altre 
nazioni stanno piano piano intraprendendo la strada dell'innovazione, l'Italia avrà il suo boom a 
fine '800.\\ [\baselineskip]
Il \textbf{Giappone} sta anch'esso industrializzandosi a poco a poco. Lì, è lo stato che decide di 
avere la stessa potenza dei paesi europei. Quindi lo stato invia ``spie'' a verificare cosa si fa
in Europa e il Giappone copia, e copia bene.\\
Anche l'\textbf{agricoltura} si comincia a modernizzare con l'uso di concimi chimici e macchine
agricole.\\ [\baselineskip]
Si cominciano a completare \textbf{reti ferroviarie} con locomotive a vapore che diventano elettiche,
acciaio per i binari. Viene inventata la \textbf{turbina} e l'\textbf{elica} e tutta la navigazione
diventa a motore, più sicura e rapida con costi minori. Ciò rendeva più conveniente i cibi americani
di quelli europei e si sviluppava la concorrenza. Così si cominciano anche a studiare metodi di
conservazione delle derrate alimentari.\\ [\baselineskip]
Lo sviluppo provoca una \textbf{forte deflazione} in quanto per la stessa domanda, l'offerta aumenta
considerevolemente. Viene questa definita la \textbf{\textit{Grande Depressione}}. Si sono attuate
3 diverse politiche per contrarstare questo fenomeno:
\begin{description}
  \item[Protezionismo] Gli imprenditori premono sui governi per aggiungere dazi e proteggere 
    l'industria interna. Nel 1873 la Germania introduce le prime tariffe, poi gli altri paesi si
    adegueranno. Da qui in poi lo stato interverrà sempre di più nella vita economica
  \item[Trust, cartelli e concentrazioni industriali] Si vengono a formare aziende frutto di fusione
    di altre più piccole
    \begin{description}
      \item[Cartelli] Accordi tra aziende che producono lo stesso bene per non farsi o ridurre la
        concorrenza (prezzi fissi, scelte di zone di vendita, \ldots). Genera prezzi più alti
      \item[Trust] Unione di aziende
        \begin{description}
          \item[Orizzontali] Che producono un bene e accorpano altre aziende del settore
          \item[Verticali] Che vanno dalla materia prima al bene finito. Sono le prime multinazionali
        \end{description}
    \end{description}
  \item[Commissioni statali] Lo stato alimenta direttamente alcune zone d'industria
\end{description}
Cambia anche il \textbf{rapporto tra aziende e banche}. Le più grandi aziende erano S.P.A. ma i fondi
non erano sufficienti, quindi chiedono dei prestiti alle banche con cui si indebiteranno. Le banche 
acquistano azioni dalle aziende finanziandole e diventandone co-proprietarie come forma di garanzia. 
La distinzione banca-azienda si fa sempre più debole. I consumatori sono danneggiati dall'aumento
dei prezzi, quindi si creano delle \textbf{norme anti-trust}.\\ [\baselineskip]
In campo sociale, c'è stata un'enorme espansione demografica, gli abitanti in Europa sono più che
raddoppiati in un secolo. Questo ha provocato un'eccedenza di mano d'opera nelle campagne che a sua
volta ha portato a una forte \textbf{emigrazione} dall'Europa verso l'America.\\ [\baselineskip]
In questo periodo si va anche a formare il \textbf{Taylorismo} ovvero l'\emph{organizzazione 
scientifica del lavoro}. Bisogna rendere il lavoro il più efficiente possibile, per fare ciò lo
si deve dividere, specializzare il lavoro in lavori più semplici e particolari. Questo porterebbe
a vantaggi per lavoratori (con salari più alti) e agli imprenditori. I sindacati erano contrari in
quanto il \textbf{lavoro era alienante}. Nel \textbf{1911} Ford crea la prima \textbf{catena di
montaggio}. La produzione era in serie, tutti i prodotti uguali con il lavoro suddiviso. Diventerà
un modello. Le industrie vanno sempre più verso la produzione di massa.

%!TEX ROOT=storia.tex

\section{Imperialismo}
In questo periodo di espansione economica si nota anche un'espansione coloniale. Pi� precisamente
avviene il fenomeno dell'\textbf{imperialismo}. Dalla fine dell'800 si attua una politica di
potenza coloniale che aveva come principali cause economiche (avere un mercato dove vendere i
propri prodotti, nuove materie prime, pi� mano d'opera, nuovi contratti statali, \ldots). Secondo
Lenin ``L'Imperialismo � la fase suprema del capitalismo''. L'imperialismo � quindi una causa 
dell'economia. Nascono da questo i movimenti \textbf{nazionalisti} non solo per questioni economiche
ma anche politiche (pi� territori si controllano, pi� si � prestigiosi) e militari. Alcuni movimenti
nazionalisti sfociano nel razzismo e nell'anti-semitismo.\\ [\baselineskip]
L'impero pi� grande era quello inglese (possedeva $\sfrac{1}{4}$ delle terre emerse e $\sfrac{1}{4}$
della popolazione). Quello francese era secondo ma meno ricco. Poi venivano tutti gli altri.\\
L'\textbf{Africa} era la nuova terra di conquista. Nel 1885 la spartizione era stata fatta a tavolino
su proposta di Bismark. Le spartizioni non tenevano minimamente conto delle popolazioni. 
L'Inghilterra voleva collegare Egitto e Sud Africa, la Francia voleva andare ad est (Marocco e 
Algeria), la Germania il Belgio e l'Italia quello che rimaneva. In \textbf{Asia} l'Inghilterra
ha l'India e la Birmania, la Francia l'Indonesia. La Cina non � stata conquistata perch� non ci
furono accordi a riguardo. Il Giapppone ha anche lui un impero (Corea). La Russia si espande verso
est fino al Giappone e a sud fino all'Afghanistan. Anche gli Stati Uniti, nati come stato coloniale
si espandono verso l'America centro-meridionale. Spacciavano le conquiste come ``liberazioni''. Gli
USA aiutano Cuba con l'indipendenza dalla Spagna per� scrivono loro la costituzione e tengono le
basi militari. Fanno lo stesso a Puerto Rico e nelle Filippine. Fanno nascere un movimento di rivolta
a Panama e nasce lo stato panamense. Gli USA hanno il controllo del canale per un secolo.

%!TEX ROOT=storia.tex

\section{Societ� di massa}
La societ� di massa � la societ� industrializzata di fine '800. L'agricoltura ha un'importanza sempre
minore, il settore terziario invece aumenta e si ingrandisce. Le citt� si ingrandiscono e diventa
una societ� \textbf{sempre pi� complessa}. Gli operai aumentano e si dividono in ruoli, la borghesia
aumenta il suo potere. La societ� si va stratificando sempre di pi�. I colletti bianchi (media 
borghesia) aumentano sempre di pi� di numero, aumentano i dipendenti pubblici (lo stato interviene
nella vita sociale). La piccola/media borghesia aveva un tenore di vita simile a quello degli operai
ma facevano di tutto pur di distinguersi (in questo clima di disagio nascono i partiti di estrema
destra).\\ [\baselineskip]
L'istruzione si diffonde sempre di pi�, piano piano. Pi� giornali vengono venduti, nascono i
giornali sportivi e si diffonde lo sport.\\ [\baselineskip]
Gli eserciti si rinforzano (leva obbligatoria) e gli ufficiali diffondono idee di patriottismo, 
\ldots. Favorivano lo studio delle lingue e la nascita di nuove idee.\\ [\baselineskip]
Il suffragio si allarga sempre di pi�. Il suffragio � universale maschile prima della WW1 e anche
in qualche paese femminile.

\subsection{Partiti socialisti}
I primi partiti sono quelli socialisti. La Seconda Internazionale si tiene a Parigi
nel 1889. Il pi� grande partito � quello \textbf{social-democratico tedesco}. L'obiettivo era di
coordinare i partiti per ottenere migliori condizioni lavorative per gli operai. Erano sostenitori
dell'\textbf{internazionalismo}. L'ideale di nazione � un ideale borghese, il proletariato non �
nazionale.\\ [\baselineskip]
Erano divisi in due correnti
\begin{description}
  \item[Rivoluzionari] Volevano i cambiamenti con violenza, senza riforme
  \item[Riformisti] Volevano i cambiamenti con graduali riforme, in modo pacifico
\end{description}
Tra i \emph{Riformisti}, \textbf{Bernstein} era uno dei pi� importanti. Nel 1899 pubblica 
\emph{``I presupposti del socialismo e i compiti della social-democrazia''}. I presupposti e gli 
ideali sono gli stessi di Marx ma lui ha commesso un errore: la situazione non sta peggiorando e la 
borghesia non si sta proletarizzando. Il crollo del capitalismo non � quindi vicino, � necessario 
migliorare la situazione dei lavoratori tramite riforme.\\ [\baselineskip]
Agli inizi del '900, si formano piccoli gruppi di rivoluzionari (estrema sinistra). Il primo era
guidato da \textbf{Lenin}. Il proletariato da solo non pu� fare la rivoluzione, ha bisogno del
partito come guida perch� non ha la coscienza di classe. Il partito � fatto da intellettuali che
pensano di capire l'economia. � composto da un'el�te di rivoluzionari per professione.\\ 
[\baselineskip]
Nel 1903 si tiene il congresso del PSD, a Londra. Lenin ottiene la maggioranza. Tra queste correnti
c'erano anche dei \emph{sindacalisti rivoluzionari} tra cui \textbf{Sorel}. Pubblica nel 1905
\emph{``Considerazioni sulla violenza''}. Erano critici contro i partiti socialisti che tendevano ad
allontanarsi dal proletariato ed erano guidati da chi viveva come un borghese. Il sindacato invece
era fatto da lavoratori che erano a stretto contatto con i proletari. L'azione spontanea � esaltata.
L'inizio della rivoluzione sarebbe stato uno sciopero generale che metter� in crisi l'economia
capitalista. � una forma di \emph{anarco-sindacalismo}.

\subsection{Partiti nazionalisti}
In questa societ� di massa si vengono a formare anche dei partiti nazionalisti. L'idea di fondo era
di valorizzare la forza e la potenza della nazione. Sono \textbf{interclassisti} in quanto tutte le
classi sociali devono collaborare per la forza della nazione. Il modello � l'esercito e la sua 
gerarchia. Sono a favore del protezionismo e dell'imperialismo. Le idee democratiche sono pericolose,
al potere deve starci chi ha veramente l'abilit�. La libert� deve essere ridotta.\\
C'erano alcuni partiti di spicco
\begin{description}
  \item[Pangermanesimo] nazionalismo tedesco che voleva riunire tutti i tedeschi in un unico stato
  \item[Revanescismo] nazionalismo francese che voleva la rivincita contro i tedeschi
  \item[Panslavismo] nazionalismo slavo per riunire tutti gli stati sotto la Russia
  \item[Inglese] per il colonialismo e l'impero
  \item[Italiano] \textbf{Enrico Corradini} � il primo ideologo. Usava un linguaggio marxista con
    significato nazionalista. Ci sono due tipi di nazioni: \emph{borghesi} (ricche, coloniali, 
    \ldots) e \emph{proletarie} (giovani, povere, sovrappopolate). L'Italia rientra in queste ultime.
\end{description}
Il \textbf{razzismo} � anche un fenomeno che � collegato al nazionalismo. La societ� umana � divisa 
in \emph{razze} che si differenziano non solo per le qualit� fisiche, ma anche per quelle morali e 
culturali che dipendono da fattori biologici.\\
\textbf{De Gobineau} � uno degli esponenti. Pubblica \emph{``Saggio sull'inuguaglianza delle razze
umane''}. Ci sono 3 razze: gialla, nera e bianca. La bianca (ariana = Europa centro-nord) � la
superiore sia sul piano fisico che intellettuale. Ha creato la cultura e solo quella ha i veri
valori. Il razzismo teme l'\textbf{ibridazione} ovvero la mescolanza fra razze. Il sangue non deve 
contaminarsi. Secondo De Gobineau sopratutto le classi superiori (di cui lui fa parte) rappresentano 
la razza ariana.\\
Legato al razzismo c'� anche l'\textbf{antisemitismo}. Comunit� ebraiche ci sono sempre state in 
Europa. Nel Medioevo erano \emph{infedeli}, dal '500 in avanti vivono in ghetti, solo nel '700 
cominciano ad integrarsi meglio. L'antisemitismo non � scomparso ma modificato. Dopo l'emancipazione,
gli ebrei si sono assimilati alla societ� e alcuni hanno anche avuto successo. Gli ebrei erano una
razza che si contrapponeva a quella ariana, anche dopo la conversione si rimaneva ebrei. La loro
pericolosit� deriva dalla loro somiglianza a noi (Chamberlain, \emph{``Fondamenti del \rom{19} 
secolo''}).
Due sono i casi-esempio di anti-semitismo che vanno ricordati
\begin{description}
  \item[Caso Dreyfuss] Dreyfuss era un capitano francese che faceva parte dello Stato Maggiore. Era
    ebreo. Nel 1894 i servizi segreti francesi scoprono che nello Stato Maggiore c'era una spia.
    Essendo l'unico ebreo, Dreyfuss fu sospettato e messo sotto processo. Vengon create prove false
    e poi condannato. Emergono ora due correnti di pensiero
    \begin{description}
      \item[Dreyfussardi] A favore di Dreyfuss (democratici, socialisti)
      \item[Anti-dreyfussardi] Contro Dreyfuss (nazionalisti, Chiesa)
    \end{description}
    Dopo qualche anno il processo viene rivisto e nel 1906 Dreyfuss � stato reintegrato.
  \item[I (falsi) protoclli dei Savi di Sion] � un libro in cui si descrive un complotto ideato
    dai rabbini per fare in modo che gli ebrei governino il mondo. Fu considerata la dimostrazione
    della pericolosit� degli ebrei. Si scopr� poi che in realt� era un falso scritto dai servizi
    segreti zaristi in quanto alcune parti erano ricopiate da romanzi di bassa lega dell'800.
    Nonostante questo c'� chi ancora crede siano veri.
\end{description}
Sotto questi influssi nasce il \textbf{Sionismo} overo il nazionalismo ebraico. 
\emph{Theodor Herzl} era un ungherese, fondatore del sionismo. Era il tipico ebreo assimilato, 
socialista e non religioso. Va a Parigi a seguire il caso Dreyfuss e nota che gli ebrei vogliono 
assimilarsi ai cristiani ma non possono perch� l'antisemitismo � troppo forte. Serve uno stato 
ebraico. Nel 1896 scrive \emph{``Lo Stato Ebraico''}. Deve nascere per accordi internazionali ed 
essere neutrale. Lo stato non � necesssariamente la Palestina. Viene creata un'organizzazione 
sionista che si riunisce la prima volta a Basilea nel 1897. L'unico territorio che avesse senso er
la Palestina che era sotto l'impero Ottomano. Non ottennero nulla. Nel '900 cambiano strategia 
facendo emigrare gli ebrei verso la Palestina dove avrebbero comprato terra e fatto i contadini. 
Dopo la WW1 cominciano i problemi in quanto gli arabi non volevano si costituisse uno stato.

\subsection{Partiti cattolici}
Nel parlamento a sinistra c'erano i socialisti, a destra i nazionalisti e in centro i cattolici.
In Germania nasce la CDU. Erano partiti sotenuti dalla Chiesa.\\
Pio \rom{9} era molto conservatore politicamente, invitava a non impegnarsi politicamente. Muore 
nel '76. Il successore Leone \rom{13} cambia atteggiamento. Nel 1891 pubblica 
\textbf{\emph{``Rerum Novarum''}} che non � altro che la dottrina sociale della Chiesa. Interviene
per la prima volta sulla "questione operaia". I lavoratori hanno dei doveri nei confronti del 
datore di lavoro (impegno, fedelt�, rispetto) ma anche dei diritti (giusto stipendio, corretti
trattamenti). Non � una riforma socialista, anzi, critica i socialisti (sono atei, senza propriet�
privata, crea lotta di classe). Voleva evitare una perdita di contatto con i lavoratori. Non �
nemmeno liberista (critica l'individualismo, esclude lo stato dalla vita economica). Rifiuta i
sindacati ma promuove le corporazioni. In Italia esistevano sindacati ma non corporazioni, non
sarebbero pi� stati credibili. I sindacati cattolici non sempre seguivano il Papa, se avessero 
rinunciato agli scioperi, sarebbero sembrati deboli.\\
Dopo la Rerum Novarum molti cattolici si sentono spinti verso la vita sociale. Agli inizi del '900
comincia a nascere la \textbf{Democrazia Cristiana}: Romolo Marri e Luigi Sturzo sono sacerdoti che
volevano fare un partito. Pio \rom{10} era pi� conservatore del predecessore e blocca l'iniziativa.
Far nascere in Italia un partito significherebbe riconoscere lo stato Italiano. Sturzo abbandona,
Marri invece continua, abbandona la Chiesa. Proprio Sturzo per� nel '19 con il sostegno della 
Chiesa fonder� il partito Cattolico.

%!TEX ROOT=storia.tex

\section{Et� Giolittiana}
L'et� giolittiana va dal 1900 al 1914.

\subsection{Economia}
Il periodo � caratterizzato da generale crescita economica. L'industrializzazione cresce ma solo in
alcune zone (Lombardia, Piemonte, Liguria) e solo alcuni settori.

\subsection{Societ�}
C'� una fortissima emigrazione (circa 500k all'anno).

\subsection{Politica}
Giolitti era un piemontese liberale. La formazione dei sindacati era inevitabile in quanto � una 
tendenza causata dall'industrializzazione. Lo stato \textbf{non deve impedire l'organizzazione}
perch� altrimenti si organizzano clandestinamente contro lo Stato. Non deve reprimere manifestazioni
pacifiche.\\
Giolitti cerc� alleanze con socialisti (offre a Turati un ministero se il PSI si fosse alelato con
il governo, rifiuta per non dividere il partito). Quando Giolitti era presidente del Consiglio, era
anche ministro dell'Interno.\\
Nel 1906 � fondata la \textbf{CGL} legata al partito socialista. Guidata da socialisti riformisti.\\
Nel 1910 � fondata la \textbf{Confindustria}.\\
Giolitti ha portato avanti importanti riforme tra cui le \textbf{prime leggi per regolamentare il 
lavoro} (obbligo del riposo festivo, vietato per donne e bambini il lavoro notturno). Nel 1911 viene 
creata l'INA (Istituto Nazionale Assicurazioni) a cui � dato il monopolio delle assicurazioni sulla 
vita. In questo modo i lavoratori erano pi� sicuri e i fondi andavano a formare un sistema 
previdenziale. Nel 1913 viene data una pensione agli infortunati sul lavoro.\\
Vengono \textbf{nazionalizzate le ferrovie} cos� si sarebbe risparmiato e si sarebbero collegati 
anche i punti pi� sfavorevoli. Fa anche riforme per il sud che dovevano favorire lo sviluppo
(costruito un acquedotto in Puglia, \ldots), in realt� non ebbero grandi risultati in quanto i
provvedimenti erano clientelari (favoritismi, \ldots).\\ [\baselineskip]
Nel 1912 � stata varata una \textbf{riforma elettorale} che permetteva il suffragio universale 
maschile per chi avesse avuto 21 anni e fato la leva militare o 30 altrimenti. Circa 9 milioni di
elettori. Uninominale a doppio turno (1 deputato per collegio, 50\% dei voti al primo turno, 
ballottaggio dei primi due). Nel 1913 le prime elezioni di massa. I socialisti erano organizzati per 
massa, non i liberali. Cos� si form� il \textbf{Patto Gentiloni} che sanciva che i candidati liberali
cattolici sarebbero stati sostenuti dalla Chiesa se poi in parlamento non avessero sostenuto 
provvedimenti che la Chiesa riteneva scomodi (divorzio, scuole cattoliche, \ldots). La Chiesa temeva
i socialisti, viene cos� eliminato il ``Non Expedit'' e i cattolici entrano nella vita dello stato
italiano.

\subsection{Esteri}
Tra il 1911 e il 1912 Giolitti intraprende una guerra coloniale. Furono presi accordi segreti con la 
Francia: l'Italia permette concede il Marocco alla Francia, la Francia non ostacola l'Italia. La
Chiesa sosteneva la guerra come fosse di civilt�. La guerra fu durissima, quasi barbara 
(avvelenamenti, capi di concentramento).\\
Nel 1912 si stipula la \textbf{Pace di Losanna}. La Libia ora � colonia Italiana. La Libia era 
allettante per l'economia secondo Giolitti, non tutti erano d'accordo (Sanvemini disse che la Libia
era una ``Scatola di Sabbia'').

%!TEX ROOT=storia.tex

\section{La Prima Guerra Mondiale}
Come ogni fenomeno complesso, la guerra non ha avuto una sola causa. Forse nessuno dei fattori, presi
singolarmente, sarebbe bastato.\\
Gli storici marxisti sottolineavano le \textbf{cause economiche} (concorrenza industriale, 
protezionismo e guerre doganali, concorrenza coloniale).\\

\subsection{Bismarck e la Germania}
Bismarck � rimasto cancelliere fino al 1890. Fino ad allora non aveva fatto una politica coloniale
in quanto sarebbe entrato in conflitto con l'Inghilterra e doveva mantenere buoni rapporti con la
Russia (il suo principale obiettivo era isolare politicamente la Francia). Bismarck si dimette nel
1890.\\
Guglielmo \rom{2} voleva una politica pi� aggressiva, coloniale. Quindi minaccia gli Inglesi creando
una flotta che possa competere con la loro. \textbf{Francia e Inghilterra si accordano sulle
colonie}.\\
I tedeschi ottengono un appalto per la costruzione di una ferrovia da Istanbul a Baghdad. Favoriscono
cos� il commercio con l'impero Ottomano delle merci tedesche.\\
Il Marocco era diviso a met� tra Spagnoli e Ottomani. Sia la Francia che la Germania lo volevano.
\textbf{Due crisi Marocchine}: 1905--1906, vinta dall'alleanza Inghilterra-Francia e 1911. 
\textbf{Francia e Germania sono sull'orlo della guerra}.\\
La Germania aveva solo l'Austria come alleata ma era continentale, senza sbocchi sul mare. L'Italia
aveva accordi con la Francia. Ormai la guerra pareva come l'unica maniera per realizzare i piani
tedeschi.\\ [\baselineskip]
Negli anni '90 Francia e Russia fanno un'alleanza militare, cos� come anche Inghilterra e Russia.
Ci sono ora due schieramenti
\begin{description}
  \item[Triplice Alleanza] Germania, Austria, Italia
  \item[Triplice Intesa] Francia, Inghilterra, Russia (accordi bilaterali)
\end{description}

\subsection{L'inizio della Guerra}
Nei Balcani c'era un contrasto fra Austria e Russia. Molti stati ottengono l'indipendenza, tra cui
la Serbia (che conteneva anche Croazia e Slovenia). Si forma cos� la \textbf{Iugoslavia} ovvero lo
Stato degli Slavi del Sud. Alcune delle popolazioni erano sotto l'Austria per�. La Russia era alleata
della Serbia. La guerra era ormai scontata anche per i movimenti nazionalisti che si andavano
diffondendo che offrivano una visione della guerra come modo per dimostrare la forza.\\ 
[\baselineskip]
Il \textbf{21 giugno 1914} Gavrilo Princip assasina Franecsco Ferdinando per protesta dell'annessione
della Bosnia all'Austria. Lo fa con il sostegno dei servizi segreti Serbi. Il \textbf{21 luglio}
scoppia la guerra. Fra il 28 e il 4 agosto si attivano le alleanze: \textbf{Austria e Germania} 
contro \textbf{Russia, Francia, Inghilterra e Serbia}.

\subsubsection{La questione Italiana}
\textbf{Antonio Salandra} guida il governo in modo liberale, di Destra. \textbf{Sonnino} � il
ministro degli esteri. Dichiara la
\textbf{neutralit�} dicendo che l'alleanza era difensiva. La popolazione si divide in due: 
\textbf{Neutralisti} e \textbf{Interventisti}.\\
\textbf{Giolitti} voleva la neutralit� in quanto non sarebbe stata sostenibile un'altra guerra dopo
quella in Libia. Contrattando la neutralit� invece si sarebbe potuto ottenere molto. La Chiesa
condivideva. \textbf{La Chiesa} esprimeva i pensieri dei contadini: una guerra contro gli Austriaci,
cattolici, non era vista bene (Benedetto \rom{15}) era il nuovo papa. Anche i \textbf{socialisti} 
erano neutralisti in quanto si sarebbe intaccato l'internazionalismo.\\
\textbf{I Nazionalisti} invece erano interventisti, per dimostrare la propria forza, i 
\textbf{democratici} si ricollegavano a Mazzini e all'idea di un completamento del Risorgimento
italiano con l'annessione delle terre irredente. Anche i \textbf{sindacati rivoluzionari} erano 
favorevoli in quanto ritenevano che la guerra avrebbe scosso il capitalismo e fatto crollare, creando
i presupposti per una rivoluzione. Infine anche i \textbf{liberali conservatori}.\\ [\baselineskip]
\textbf{Dopo mesi, entriamo in guerra contro Austria e Germania}. Per convenienza. Il 26 aprile 1915
fu stipulato segretamente il \textbf{Patto di Londra} tra l'Italia e l'Intesa. Entro un mese l'Italia
sarebbe dovuta entrare in guerra contro l'Austria, in cambio avrebbe ricevuto
\begin{enumerate}
  \item Le terre irredente
  \item L'Alto-Adige
  \item L'Istria
  \item La Dalmazia e il porto di Valona
  \item Il controllo della politica estera dell'Albania
  \item Il Dodecaneso
  \item Un bacino di carbone in Turchia
  \item Alcune colonie tedesche in Africa
\end{enumerate}
L'Italia ora doveva entrare in guerra ma i neutralisti erano in maggioranza in parlamento e tra il
popolo. \textbf{Salandra si dimette}. Ci furono molte manifestazioni causate da questa crisi di
governo (studenti, borghesi, socialisti, \ldots). Il governo allora \textbf{lascia liberi gli
interventisti} e \textbf{asseconda i socialisti} per dimostrare che l'Italia voleva la guerra.\\
Vittorio Emanuele \rom{3} chiama Giolitti e lo informa sul patto di Londra. Giolitti, temendo una
crisi istituzionale della monarchia ed essendo comunque piemontese, abbandona Roma (ovvero rinuncia
a tenere l'Italia fuori dalla guerra). Richiama Salandra al Quirinale e gli conferisce poteri 
speciali (20 maggio) e finanziamenti per sostenere la guerra. \textbf{Il parlamento vota l'entrata in
guerra con il sostegno anche dei liberali giolittiani} (non dei socialisti). Il 
\textbf{24 maggio 1915} l'Italia entra in guerra contro l'Austria.\\
L'entrata in guerra � importante anche per la politica interna in quanto Salandra, Sonnino e il re 
sono riusciti a togliere il potere al parlamento e a dare al re il governo.

\subsection{Lo spirito del combattente}
Perch� combatte un soldato?
\begin{description}
  \item[Per solidariet�] nei confronti dei compagni
  \item[Per rassegnazione] dopo il primo inverno e dopo l'abitudine
\end{description}
Nonostante le nobili intenzioni, i fenomeni di diserzione e ribellione non erano infrequenti. 
L'istinto di sopravvivenza e il sottrarsi alla morte avevano la meglio.\\
Al polo opposto stava un'idologia ``bellicista'', secondo la quale la guerra � la massima 
esaltazione ed espressione pi� alta dell'esperienza umana. Ernst J�nger riteneva che la guerra 
fosse un momento costitutivo di una nuova razza superiore alle precedenti. In Italia l'equivalente
di questi erano gli ``arditi'', capaci di rovesciare le regole tradizionali di combattimento (secondo
Giorgio Rochat).

\subsection{La partecipazione delle masse nella guerra}
Oltre alle innovazioni tecnologiche, il vero motore della guerra era la forza d'urto delle masse
di uomini mandati al fronte. Quelli che combattevano non erano altro che contadini, operai, impiegati
pubblici o privati che avevano alimentato l'intervento Italiano. La guerra dunque fu un modo per
\textbf{rafforzare lo spirito delle masse} che ora diventavano le vere protagoniste. Omogeneizz�
inoltre tutti gli strati sociali accomunandoli con il concetto di ``nazione''.\\
In Italia in particolar modo la guerra fu vista come un completamento del Risorgimento. Nei primi
50 anni dell'unit� buona parte della massa non si sentiva unita, non si sentiva ``Italiana''.
Al fronte i ceti pi� bassi avevano scoperto l'ideale di nazione pi� grande del loro paese d'origine.
Il destino comune, la lingua comune, amalgamavano tutti gli strati sociali.

\subsection{Lo stallo del 1915--1916}
Con l'entrata in guerra dell'Italia si apre un nuovo fronte, meridionale. La guerra era lenta, cos�
detta di \textbf{logoramento} in trincea. Nessuno dei due fronti otteneva vere e proprie vittorie
e riusciva ad avanzare. Questo stallo danneggia prevalentemente gli \textbf{Imperi centrali} in 
quanto sono isolati dal resto del mondo e il blocco commerciale li danneggiava.\\
Per smuovere la situazione, gli imperi tedeschi concentrano le forze a \textbf{Verdun} dove la
battaglia si protrasse per oltre 5 mesi con pi� di mezzo milione di morti. I tedeschi falliscono
in questo tentativo.\\
Provano anche a sfondare via mare, con una battaglia contro gli Inglesi nello \textbf{Jutland},
vicino allo Skagerrak.\\
Tentarono quindi la guerra \textbf{sottomarina} per affondare le navi inglesi. Ottenne buoni 
risultati ma vide anche l'entrata in guerra degli Stati Uniti.\\ [\baselineskip]
Sul fronte meridionale i tedeschi fecero una \textbf{spedizione punitiva} contro l'Italia che
vide l'occupazione di Asiago come risultato. L'impreparazione dell'esercito Italiano port� Salandra
a dimettersi. Dopo una cruenta battaglia le truppe italiane furono capaci di \textbf{prendere 
Gorizia} (9 agosto 1916).\\ [\baselineskip]
Per fronteggiare il malcontento, molti governi formarono dei \textbf{governi di unit� nazionale}, 
ovvero dei governi di grandi alleanze. Dopo le dimissioni di Salandra, in Italia si form� il governo
\textbf{Boselli}, in Francia \textbf{Briand} e in Gran Bretagna \textbf{David Lloyd George}. In
Germania tutto il potere fu concentrato nell'imperatore e nelle pi� alte gerarchie militari.\\
Questo accentramento porta i governi a pianificare e dirigere direttamente la guerra. Infatti ci fu
l'influsso sulle aziende di \textbf{innovare} e migliorare i propri prodotti. Queste modifiche
portarono anche alla \textbf{militarizzazione del lavoro in fabbrica} e alla limitazione delle
libert� sindacali. La guerra inoltre era costosa e questo port� all'introduzione di nuove tasse
coon conseguente aumento del debito pubblico e inflazione.

\subsection{La guerra ``Mondiale''}
Nel 1917 la \textbf{Russia} si ritira dal conflitto a causa della rivoluzione bolscevica. Cos� entra
una nova nazione: gli \textbf{Stati Uniti}. Lo zar a causa della rivoluzione che ne esce, �
costretto ad abdicare (rivolte di operai e soldati nella capitale) e si forma un governo repubblicano
provvisorio. \textbf{Karenskij} era a capo del governo e decise, dopo una sconfitta militare, di
uscire definitivamente dalla guerra. Gli Stati Uniti entrano in guerra principalmente a causa della
guerra sottomarina che stava flagellando gli inglesi.\\ [\baselineskip]
Se all'inizio la popolazione aveva preso di buon grado l'entrata in guerra, ora si diffondeva
malcontento, stanchezza e insofferenza. I soldati erano abbandonati a s� stessi, malnutriti e quasi
sommersi dalle trincee. L'utilizzo di nuove armi (bombe a mano, gas, lanciafiamme) e le pessime
condizioni, favorivono \textbf{diserzioni di massa} e ammutinamenti. Il ``disfattismo'' ormai dilgava
in tutte le fasce della gerarchia.\\
Per fermare questa ondata, i governi agiscono con battente propaganda e devono anche arginare il 
problema del ``fronte interno'', ovvero di tutti quegli strati sociali che per varie ragioni si
opponevano alla guerra. La scarsit� di cibo e di beni era anche aumentata dai prezzi esorbitanti che
i proprietari delle aziende che avevano avuto il monopolio, fissavano.\\
In Francia si cambi� gli uomini al governo: \textbf{P�tain} � il nuovo generale e \textbf{Clemenceau}
� il nuovo primo ministro, determinato alla guerra.\\
In Germania si fece qualcosa di simile: si \textbf{militarizzarono le industrie} e il potere si
concentr� nel capo di stato maggiore \textbf{von Hindenburg}.\\ [\baselineskip]
Nel 1917, gli imperi centrali provano uno \textbf{sforzo offensivo eccezionale} nella speranza
di risolvere il conflitto in breve. Sferrarono un attacco nell'Isonzo. Il generale Italiano era
\textbf{Luigi Cadorna} che con il suo esercito non resistette all'urto. Fu la \textbf{Disfatta di
Caporetto} che port� una ritirata fino al Piave. Si forma cos� un \textbf{nuovo governo} e l'esercito
fu affidato a \textbf{Armando Diaz}.

\subsection{La disfatta di Caporetto}
\textbf{Il 24 ottobre 1917 un attacco austro-tedesco sfonda la linea italiana}. L'esercito italiano
fugge verso ovest con circa 300 mila prigionieri e altrettanti sbandati. I tedeschi sono fermati sul
Piave. 11 mesi dopo, con la battaglia di Vittorio Veneto si riconquista il Veneto e il Friuli.\\
\textbf{Cadorna sapeva ci sarebbe stato un attacco} eppure non ha fatto niente. Sul campo di 
battaglia in montagna, per ogni soldato c'era bisogno di 4 uomini. Dopo la sconfitta, Cadorna
\textbf{accusa} i soldati di essere stati vili ed aver abbandonato il campo di battaglia. \textbf{Un
generale non dovrebbe mai condannare i propri soldati}.\\
Le truppe italiane si sentivano ``distaccate'' dalla guerra, ecco perch� molti disertavano. O almeno
cos� si diceva. In realt� \textbf{quasi nessuno disert� e fugg� dal campo di battaglia} tranne i
colonnelli e i generali. Tutti i reparti continuarono a combattere. Alcuni generali rimasero
al fronte e combatterono assieme ai soldati, molti per� abbandonarono i propri reparti.\\
La prima guerra mondiale � stata una \textbf{guerra di popoli} in cui si voleva solo l'annientamento
del nemico. I reparti italiani avevano 600 cannoni, quelli tedeschi pi� di 1200 (li avevano nascosti
e sapeva dove fossero quelli italiani).\\
La leggenda dello ``sciopero militare'' di Caporetto nacque da un libro di \textbf{Alberti}. Infatti
circa 7000 prigionieri descrivevano le azioni eroiche del battaglione che continuava a combattere
nonostante tutto.

\subsection{La fine della guerra}
I tedeschi spinti dalla vittoria di Caporetto, sferrarono un attacco sul fronte occidentale vicino a
\textbf{San Quintino} in cui l'Intesa fu sfondata fino alla \textbf{Marna}. La battaglia per�
riprese con l'uso di nuove tecnologie (cannoni tedeschi e arerei e carri armati inglesi).\\
A luglio con l'arrivo degli americani il fronte fu sfondato verso \textbf{Amiens} e cominci� 
l'avanzata anche a sud, in Italia fino a Vittorio Veneto. \textbf{Il 4 novembre 1918 fu firmato
l'armistizio tra Austria e Italia}. L'impero asburgico si stava disgregando e la Germania deve
arrendersi anche lei, l'11 novembre.\\ [\baselineskip]
Nel gennaio del 1919 a \textbf{Versailles} si ritrovano i paesi vincitori (Francia, Gran Pretagna,
Stati Uniti e Italia). C'erano due schieramenti diversi
\begin{description}
  \item[Europa] capitanata da \textbf{Clemenceau} voleva mantenere le tradizionali annessioni
    territoriali, incentrate sull'egemonia della Francia e della Gran Bretagna in Europa.
  \item[Stati Uniti] con \textbf{Wilson} voleva affermare il principio dell'autodeterminazione,
    ovvero i vincitori dovevano solo ridisegnare la mappa geo-politica dell'Europa.
\end{description}
L'Italia no riusc� ad ottenere l'annessione di Fiume e della Dalmazia conducendo ad un'insistita
campagna nazionalistica per la vittoria mutilata.\\ [\baselineskip]
Prevalse la linea dura di Clemenceau che \textbf{impose} il trattato di Versailles ai tedeschi.
Esso stabiliva
\begin{itemize}
  \item La restituzione alla Francia dell'\textbf{Alsazia e Lorena}, lo smembramento dei possedimenti
    coloniali e il pagamento dei danni di guerra
  \item La creazione di \textbf{Austria}, \textbf{Ungheria} e \textbf{Cecoslovacchia}, 
    \textbf{Iugoslavia}
  \item La \textbf{Polonia} fu ricostruita
  \item L'Italia ottenne l'\textbf{Alto Adige}, \textbf{Trieste} e l'\textbf{Istria} 
  \item Gli altri territori furono messi sotto il controllo Francese o Inglese
\end{itemize}
Infine, su volere di Winson, fu istituita la \textbf{Societ� delle Nazioni} che aveva lo scopo di 
tutelare la pace facendo da arbitro nelle controversie internazionali. Questo progetto per�
non and� molto lontano in quanto Russia, Germania e gli stessi Stati Uniti restarono fuori. Divenne
quindi un mezzo per Francia e Gran Bretagna per esercitare il potere.

%!TEX ROOT=storia.tex

\section{Rivoluzione Russa}
La Russia era il paese più vasto, un impero multietnico con più di 100 milioni di abitanti di cui
la metà russi, gli altri di varie etnie (Ucraini, Armeni, \ldots) con lingue e culture diverse. Non
sempre accettavano di buon grado il governo russo.\\
L'80\% della popolazione era contadina, erano analfabeti e solo nel 1861 era stata vieteta la
servitù della gleba. Solo alla fine dell'800 comincia un minimo di rivoluzione industriale (a San
Pietroburgo (la capitale), a Mosca (per i tessuti), negli Urali (per il ferro) e nel Mar Nero (per il
petrolio)). L'agricoltura era arretrata e i capitali per le industrie erano principalmente 
provenienti dall'estero. C'era poca borghesia e tanta nobiltà che non aveva la mentalità 
imprenditoriale.\\
Lo Zar era \textbf{Nicola \rom{2}} che regnava con un potere autocratico, senza costituzione, 
parlamento, diritti o libertà. La Chiesa ortodossa legittimava il potere dello zar.\\ [\baselineskip]
Nel 1905 era in \textbf{guerra con il Giappone}. La Russia perde e aumenta il malcontento. 
Manifestazioni di protesta e l'esercito le reprime con la forza. Continuano e lo zar concede la
\textbf{Duma}, un parlamento con potere legislativo, e libertà di stampa e associazione. Negli anni
seguenti però pian piano riduce i poteri alla Duma e riduce anche i diritti e il diritto di voto.\\
Nel 1914 arriva in guerra con circa 6 milioni di uomini. I più numerosi ma i peggio armati. I beni
di prima necessità scarseggiano sia al fronte che in città.\\
Nel \textbf{1916} lo zar convoca la Duma per ricevere sostegno per introdurre nuove tasse, la Duma si
oppone e viene sciolta. I leader politici si tengono in contatto.

\subsection{Inizio delle manifestazioni}
\textbf{23 febbraio 1917} a Pietrograto (= Pietroburgo) si tiene la prima manifestazione 
rivoluzionaria. Le successive manifestazioni vengono represse dall'esercito ma successivamente i
soldati sostengono le manifestazioni. Lo zar richiama dal fronte alcune truppe fedeli. Non arrivarono
mai in quanto bloccate dai ferrovieri, a favore della rivoluzione. Le manifestazioni si diffondono
fino a Mosca e al fronte; ad inizio marzo la situazione è fuori controllo.\\
I generali consigliano l'abdicazione dello zar, infatti nel \textbf{2 marzo 1917} lo zar abdica a
favore del fratello Michele che rinuncia al trono. In Russia termina così la dinastia Romanov. La
Duma elegge un governo provvisorio. Gli obiettivi di questo governo erano
\begin{description}
  \item[Continuare la guerra] Mantenere un legame con la Francia e l'Inghilterra, arrivare alla
    conquista
  \item[Democratizzare lo stato] Fare una costituzione che garantisca diritti e libertà. Si poteva
    fare solo dopo la fine della guerra
  \item[Promulgare una riforma agraria] Dare terra ai contadini
\end{description}
Il governo era sostenuto dai \textbf{cadetti} (liberali, democratici, borghesi russi) e dal
\textbf{partito social-rivoluzionario} (non è marxista, si richiamava alla tradizione russa che aveva
come base i contadini russi e i villaggi (Mir)) e i \textbf{menscevichi} (marxisti riformisti,
deve svilupparsi la politica come democrazia rappresentativa). Lenin nel frattempo era in esilio,
i capi bolscevichi erano divisi sul sostenere o meno il governo.\\
Accanto al governo si formano i \textbf{Soviet} ovvero dei consigli di fabbrica o di settore, eletti.
C'era il soviet della città che riunisce i rappresentanti dei locali soviet. Va contro l'idea
liberal-democratica rappresentativa. Il soviet voleva essere un esempio di \textbf{democrazia
diretta} anche se in realtà non era così.\\
Si ottiene così una situazione di diarchia: da un lato c'è il governo provvisorio della Duma, da un
altro i soviet.

\subsection{Lenin al potere}
Lenin torna a Pietrogrado grazie ai servizi segreti tedeschi. Tornato propone le \textbf{Tesi 
d'aprile} che in generale raccolgono il programma leninista. Lenin \textbf{vuole la pace subito}, il
governo (borghese) invece vuole continuare la guerra con obiettivi imperialisti.\\
La \textbf{Quarta tesi} è importante: i bolscevichi erano in minoranza nei soviet. Così i bolscevichi
vogliono prima prendere la maggioranza nei soviet e poi trasferire tutto il potere ai soviet,
esautorando il governo. Quindi Lenin vuole \textbf{fare la rivoluzione, subito} (andando un po'
contro Marx). Il modello era la Comune di Parigi. Lenin vuole nazionalizzare la terra senza 
indennizzo per i proprietari, così aumenta il consenso tra i contadini.\\
Un'altra tesi \textbf{voleva cambiare il nome del partito in ``Partito Comunista''}. L'ultima creava
la \textbf{Terza Internazionale} dei partiti rivoluzionari.\\ [\baselineskip]
Vengono proposte delle ffensive contro i tedeschi ma senza alcun successo (enormi diserzioni,
screditavano il governo, aumenta il malcontento). Nel luglio del 1917 ci sono \textbf{manifestazioni
a Pietrogrado}, il governo (Karenskij era a capo) risponde con la forza mettendo fuori legge il 
partito bolscevico.\\
Ad agosto il generale Kornikov voleva prendere il potere portando i propri soldati a Pietrogrado ma
i bolscevichi interrompono i collegamenti ferroviari e non ha successo. Da questo momento i 
bolscevichi soon organizzati.\\ [\baselineskip]
Tra \textbf{agosto e settembre 1917} i bolscevichi prendono la maggioranza nei soviet delle maggiori
città, il mese successivo l'obiettivo diventa il potere. Lenin propone il colpo di stato, Trotzkij
organizza le guardie rosse. \textbf{24--25 ottobre} (6--7 novembre) le guardie rose prendono il 
controllo delle vie di comunicazione e dei centri di potere. Assaltano il Palazzo d'Inverno in cui si
era riunito il governo provvisorio. I ministri fuggono.\\
Viene istituito il \textbf{Consiglio dei commissari del popolo} (eletti tramite democrazia diretta in
teoria). Furono stabiliti due decreti
\begin{description}
  \item[Sulla guerra] Appello ai paesi in guerra per interromperla senza indennità o spartizioni. 
    Pace incondizionata subito. Questo mette a favore i proletari
  \item[Sulla terra] Latifondisti espropriati senza indennizzo, creare aziende statali per 
    organizzare, in realtà i contadini si prendevano la terra e la amministravano privatamente. 
    Potava il favore dei contadini
\end{description}
\textbf{A gennaio del 1918 viene convocata l'assemblea costituente}. La maggioranza è ai socialisti 
rivoluzionari, i bolscevichi hanno circa un quarto dei voti. Il primo giorno i menscevichi e i
socialisti rivoluzionari criticano i bolscevichi, così l'assemblea viene sciolta la sera stessa
in quanto una forma di governo borghese. Si allontanano così dalla democrazia. I bolscevichi quindi
governano da soli con un minimo supporto della parte più esterma dei socialisti rivoluzionari.
Tortzkij tratta la pace con la Germania: \textbf{Trattatto di Brest-Litovsk}. La Russia perde un
quarto dei territori riconoscendo l'indipendenza dell'Ucraina, Estonia, Lettonia e Lituania.\\
[\baselineskip]
Lenin doveva fare un governo comunista da zero, con tutto il mondo ostile. Era fiducioso che la
rivoluzione fosse vicina anche in occidente. I primi provvedimenti che fa sono:
\begin{itemize}
  \item Nelle fabbriche l'organizzazione del lavoro la faceva il soviet
  \item Abolita la leva obbligatoria
  \item Uguaglianza nei confronti della legge
\end{itemize}

\subsection{La nascita dello stato sovietico e la guerra civile}
Il governo comunista era forte fuori dalle città principali dove invece le \textbf{armate bianche},
guidate da generali zaristi, vincevano. A luglio la famiglia dello zar e lo zar stesso vengono
fucilati. Le armate sono aiutate dalla comunità internazionale per viveri e denaro. Le truppe inviate
nel 1918 però erano per buona parte contadini che sentivano delle riforme nella russa comunista.\\
Il governo di Lenin è in pericolo, vuole fare un esercito e quindi reintroduce la leva obbligatoria.
Vengono anche richiamati degli ufficiali zaristi a guidare l'esercito. Trotzkij organizza
\textbf{l'Armata Rossa}. Le industrie belliche vengono fatte lavorare a pieno regime (anche se erano
gestite dai soviet, erano sottomesse alle direttive del governo). Nelle campagne i contadini 
nascondono i raccolti e li vendono al mercato nero perché il governo fissava i prezzi. Il governo
attua una \textbf{politica di requisizione}, portando via il raccolto. Lenin definisce questa 
politica \textbf{``Comunismo della guerra''}, sul modello tedesco, questa è la dittatura del 
proletariato, non l'autentico comunismo. Questo comporta anche l'abolizione di libertà di 
associazione politica (mono-partitismo). Viene istituita la CEKA, la polizia politica.\\
[\baselineskip]
Nel 1920 la Polonia attacca la Russia tentando di conquistare territori approfittandone della 
debolezza. Nulla di fatto. Durante la guerra civile era stata riconquistata l'Ucraina. \textbf{Si
ribella la base navale di Kronstadt} criticando il governo leninista in quanto troppo autoritario e 
non marxista. Viene mandata l'armata rossa, metà morti e gli altri imprigionati.\\
Nei primi mesi del 1921 \textbf{la guerra civile finisce} e in marzo si riunisce il decimo congresso
del partito comunista dove vengono prese due decisioni sull'organizzazione interna e sull'economia.
\begin{description}
  \item[Organizzazione interna] Viene condannato il frazionismo, non ci devono essere correnti di 
    partito
  \item[Economia] (NEP) La produzione industriale viene ridotta al 13\%, Lenin vuole reintrodurre
    alcune cose capitaliste (consentire agli imprenditore qualche dipendente (circa 10), consentire
    il commercio al dettaglio, consentire ai propietari terrieri di avere dipendenti). I livelli
    più alti dell'economia erano statali, quelli più bassi privati.
\end{description} 
Trotzkij era contrario in quanto rivedeva una rinascita della borghesia. Lenin sperava di migliorare
l'economia in ginocchio. Queste riforme durano fino al 1928.\\ [\baselineskip]
Nel 1924 viene fatta una \textbf{costituzione} (la più importante di tutte quelle che furono fatte).
Formalmente era una repubblica federale (nasce l'\textbf{URSS}), in realtà non era così. Il 
parlamento è il congresso dei soviet, era una dittatura del parrtito.\\
L'ultima tesi di Aprle voleva una Terza Internazionale, a Mosca viene istituito il \textbf{Cominter},
l'internazionale comunista. Anche partiti europei entrano a farne parte. Si poteva solo se si 
cambiava nome in ``Partito Comunista'' e avere come modello il leninismo. I riformisti inoltre
vanno espulsi dal partito.


\newpage
\listoftodos[Note]
\end{document}
