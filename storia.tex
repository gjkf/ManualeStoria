\documentclass[usenames,dvipsnames]{article}
\usepackage[italian]{babel}
\usepackage[utf8]{inputenc}
\usepackage{bm}
\usepackage{amsmath, amsfonts, amssymb, mathtools, calrsfs}
\usepackage{xfrac}
\usepackage[pdftex,dvipsnames]{xcolor}
\usepackage[colorinlistoftodos,prependcaption,textsize=tiny]{todonotes}
\usepackage{xtab, booktabs, array}
\usepackage[labelformat=empty]{caption}
\usepackage{tikz}
\usetikzlibrary{arrows,decorations,calc,intersections,shapes.geometric}
\usepackage{parskip} % To avoid indentation
\usepackage{textcomp} % To have \textcelsius and other symbols
\usepackage{multirow} % For nicer tables with split rows/columns
\usepackage{multicol}
\usepackage{cancel} % Cacelled equations and simplifications
\usepackage{hyperref}
\hypersetup{
  colorlinks,
  citecolor=black,
  filecolor=black,
  urlcolor=black,
  linkcolor=blue
}
\usepackage[all]{hypcap}

\usepackage[showframe=false, top=2cm, bottom=2.5cm, left=2.5cm, right=2.5cm]{geometry}

% Some useful TODO commands
\usepackage{xargs}
\newcommandx{\unsure}[2][1=]{\todo[linecolor=red,backgroundcolor=red!25,bordercolor=red,#1]{#2}}
\newcommandx{\change}[2][1=]{\todo[linecolor=blue,backgroundcolor=blue!25,bordercolor=blue,#1]{#2}}
\newcommandx{\info}[2][1=]{\todo[linecolor=OliveGreen,backgroundcolor=OliveGreen!25, bordercolor=OliveGreen,#1]{#2}}
\newcommandx{\improvement}[2][1=]{\todo[linecolor=Plum,backgroundcolor=Plum!25, bordercolor=Plum,#1]{#2}}
\newcommandx{\thiswillnotshow}[2][1=]{\todo[disable,#1]{#2}}
\setlength{\marginparwidth}{2cm}

% For an older but clearer root. Still \oldsqrt is valid
\usepackage{letltxmacro}
\makeatletter
\let\oldr@@t\r@@t
\def\r@@t#1#2{%
  \setbox0=\hbox{$\oldr@@t#1{#2\,}$}\dimen0=\ht0
  \advance\dimen0-0.2\ht0
  \setbox2=\hbox{\vrule height\ht0 depth -\dimen0}%
{\box0\lower0.4pt\box2}}
\LetLtxMacro{\oldsqrt}{\sqrt}
\renewcommand*{\sqrt}[2][\ ]{\oldsqrt[#1]{#2} }
\makeatother

% Matrix spacing
\makeatletter
\renewcommand*\env@matrix[1][\arraystretch]{%
  \edef\arraystretch{#1}%
  \hskip -\arraycolsep
  \let\@ifnextchar\new@ifnextchar
\array{*\c@MaxMatrixCols c}}
\makeatother

\newcommand\twodigits[1]{%
  \ifnum#1<10 0\number#1 \else #1\fi
}
\usepackage[yyyymmdd]{datetime}
\renewcommand{\dateseparator}{-}
\usepackage{fancyhdr}
\pagestyle{fancy}
\fancyhead{} % clear all header fields
\renewcommand{\headrulewidth}{0pt} % no line in header area
\fancyfoot{} % clear all footer fields
\fancyfoot[C,CO]{\thepage}% page number in "outer" position of footer line
\fancyfoot[R,RO]{Copyright \copyright 2017--\the\year$\,$Cossu Davide
} % other info in "inner" position of footer line
\fancyfoot[L,LO]{Version 1.3.4 \today
}

\DeclarePairedDelimiter\norm{\lVert}{\rVert} % ||v||  
\DeclarePairedDelimiter\abs{\lvert}{\rvert} % |v|
\newcommand\markangle[7][red]{% [color] origin A B radius radiusmark mark
  % fill red circle
  \begin{scope}
    \path[clip] (#2) -- (#3) -- (#4);
    \fill[color=#1,fill opacity=0.5,draw=#1,name path=circle]
    (#2) circle (#5);
  \end{scope}
  % middle calculation
  \path[name path=line one] (#2) -- (#3);
  \path[name path=line two] (#2) -- (#4);
  \path[%
  name intersections={of=line one and circle, by={inter one}},
  name intersections={of=line two and circle, by={inter two}}
  ] (inter one) -- (inter two) coordinate[pos=.5] (middle);
  % put mark
  \node at ($(#2)!#6!(middle)$) {#7};
}
\def\mathcolor#1#{\@mathcolor{#1}}
\def\@mathcolor#1#2#3{%
  \protect\leavevmode
  \begingroup
  \color#1{#2}#3%
  \endgroup
}
% For better visual in tables
\renewcommand*{\arraystretch}{2}
% To center with m{}
\newcolumntype{M}[1]{>{\centering\arraybackslash}m{#1}}
\newcommand{\divisor}{\rule{8.7cm}{0.4pt}}

\makeatletter
\newcommand*{\rom}[1]{\expandafter\@slowromancap\romannumeral #1@}
\makeatother

\begin{document}
{
  \hypersetup{linkcolor=black}
  \tableofcontents
}
%!TEX ROOT=storia.tex

\section{Seconda rivoluzione industriale}
La seconda rivoluzione industriale non ha dei limiti temporali definiti. La si può indicativamente
far andare dal \textbf{1870} al \textbf{1914} circa.\\
Una delle invenzioni che hanno caratterizzato questo periodo è stata quella del \textbf{motore
elettrico} e quella del \textbf{motore a scoppio}. Di conseguenza sono nate \textbf{dinamo},
\textbf{lampadine}, \textbf{aerei}, \textbf{telefoni} e \textbf{radio}.\\
A queste innovazioni si collega la nascita di molte industrie e aziende che producevano e sostenevano
queste innovazioni. Le più importanti furono aziende \textbf{chimiche}, \textbf{siderurgiche} ed
\textbf{elettriche}. Gli \emph{Stati Uniti} e la \emph{Germania} erano le più innovative nazioni,
superando persino l'\emph{Inghilterra} che però deteneva ancora il primato finanziario. Le altre 
nazioni stanno piano piano intraprendendo la strada dell'innovazione, l'Italia avrà il suo boom a 
fine '800.\\ [\baselineskip]
Il \textbf{Giappone} sta anch'esso industrializzandosi a poco a poco. Lì, è lo stato che decide di 
avere la stessa potenza dei paesi europei. Quindi lo stato invia ``spie'' a verificare cosa si fa
in Europa e il Giappone copia, e copia bene.\\
Anche l'\textbf{agricoltura} si comincia a modernizzare con l'uso di concimi chimici e macchine
agricole.\\ [\baselineskip]
Si cominciano a completare \textbf{reti ferroviarie} con locomotive a vapore che diventano elettiche,
acciaio per i binari. Viene inventata la \textbf{turbina} e l'\textbf{elica} e tutta la navigazione
diventa a motore, più sicura e rapida con costi minori. Ciò rendeva più conveniente i cibi americani
di quelli europei e si sviluppava la concorrenza. Così si cominciano anche a studiare metodi di
conservazione delle derrate alimentari.\\ [\baselineskip]
Lo sviluppo provoca una \textbf{forte deflazione} in quanto per la stessa domanda, l'offerta aumenta
considerevolemente. Viene questa definita la \textbf{\textit{Grande Depressione}}. Si sono attuate
3 diverse politiche per contrarstare questo fenomeno:
\begin{description}
  \item[Protezionismo] Gli imprenditori premono sui governi per aggiungere dazi e proteggere 
    l'industria interna. Nel 1873 la Germania introduce le prime tariffe, poi gli altri paesi si
    adegueranno. Da qui in poi lo stato interverrà sempre di più nella vita economica
  \item[Trust, cartelli e concentrazioni industriali] Si vengono a formare aziende frutto di fusione
    di altre più piccole
    \begin{description}
      \item[Cartelli] Accordi tra aziende che producono lo stesso bene per non farsi o ridurre la
        concorrenza (prezzi fissi, scelte di zone di vendita, \ldots). Genera prezzi più alti
      \item[Trust] Unione di aziende
        \begin{description}
          \item[Orizzontali] Che producono un bene e accorpano altre aziende del settore
          \item[Verticali] Che vanno dalla materia prima al bene finito. Sono le prime multinazionali
        \end{description}
    \end{description}
  \item[Commissioni statali] Lo stato alimenta direttamente alcune zone d'industria
\end{description}
Cambia anche il \textbf{rapporto tra aziende e banche}. Le più grandi aziende erano S.P.A. ma i fondi
non erano sufficienti, quindi chiedono dei prestiti alle banche con cui si indebiteranno. Le banche 
acquistano azioni dalle aziende finanziandole e diventandone co-proprietarie come forma di garanzia. 
La distinzione banca-azienda si fa sempre più debole. I consumatori sono danneggiati dall'aumento
dei prezzi, quindi si creano delle \textbf{norme anti-trust}.\\ [\baselineskip]
In campo sociale, c'è stata un'enorme espansione demografica, gli abitanti in Europa sono più che
raddoppiati in un secolo. Questo ha provocato un'eccedenza di mano d'opera nelle campagne che a sua
volta ha portato a una forte \textbf{emigrazione} dall'Europa verso l'America.\\ [\baselineskip]
In questo periodo si va anche a formare il \textbf{Taylorismo} ovvero l'\emph{organizzazione 
scientifica del lavoro}. Bisogna rendere il lavoro il più efficiente possibile, per fare ciò lo
si deve dividere, specializzare il lavoro in lavori più semplici e particolari. Questo porterebbe
a vantaggi per lavoratori (con salari più alti) e agli imprenditori. I sindacati erano contrari in
quanto il \textbf{lavoro era alienante}. Nel \textbf{1911} Ford crea la prima \textbf{catena di
montaggio}. La produzione era in serie, tutti i prodotti uguali con il lavoro suddiviso. Diventerà
un modello. Le industrie vanno sempre più verso la produzione di massa.

%!TEX ROOT=storia.tex

\section{Imperialismo}
In questo periodo di espansione economica si nota anche un'espansione coloniale. Pi� precisamente
avviene il fenomeno dell'\textbf{imperialismo}. Dalla fine dell'800 si attua una politica di
potenza coloniale che aveva come principali cause economiche (avere un mercato dove vendere i
propri prodotti, nuove materie prime, pi� mano d'opera, nuovi contratti statali, \ldots). Secondo
Lenin ``L'Imperialismo � la fase suprema del capitalismo''. L'imperialismo � quindi una causa 
dell'economia. Nascono da questo i movimenti \textbf{nazionalisti} non solo per questioni economiche
ma anche politiche (pi� territori si controllano, pi� si � prestigiosi) e militari. Alcuni movimenti
nazionalisti sfociano nel razzismo e nell'anti-semitismo.\\ [\baselineskip]
L'impero pi� grande era quello inglese (possedeva $\sfrac{1}{4}$ delle terre emerse e $\sfrac{1}{4}$
della popolazione). Quello francese era secondo ma meno ricco. Poi venivano tutti gli altri.\\
L'\textbf{Africa} era la nuova terra di conquista. Nel 1885 la spartizione era stata fatta a tavolino
su proposta di Bismark. Le spartizioni non tenevano minimamente conto delle popolazioni. 
L'Inghilterra voleva collegare Egitto e Sud Africa, la Francia voleva andare ad est (Marocco e 
Algeria), la Germania il Belgio e l'Italia quello che rimaneva. In \textbf{Asia} l'Inghilterra
ha l'India e la Birmania, la Francia l'Indonesia. La Cina non � stata conquistata perch� non ci
furono accordi a riguardo. Il Giapppone ha anche lui un impero (Corea). La Russia si espande verso
est fino al Giappone e a sud fino all'Afghanistan. Anche gli Stati Uniti, nati come stato coloniale
si espandono verso l'America centro-meridionale. Spacciavano le conquiste come ``liberazioni''. Gli
USA aiutano Cuba con l'indipendenza dalla Spagna per� scrivono loro la costituzione e tengono le
basi militari. Fanno lo stesso a Puerto Rico e nelle Filippine. Fanno nascere un movimento di rivolta
a Panama e nasce lo stato panamense. Gli USA hanno il controllo del canale per un secolo.

%!TEX ROOT=storia.tex

\section{Societ� di massa}
La societ� di massa � la societ� industrializzata di fine '800. L'agricoltura ha un'importanza sempre
minore, il settore terziario invece aumenta e si ingrandisce. Le citt� si ingrandiscono e diventa
una societ� \textbf{sempre pi� complessa}. Gli operai aumentano e si dividono in ruoli, la borghesia
aumenta il suo potere. La societ� si va stratificando sempre di pi�. I colletti bianchi (media 
borghesia) aumentano sempre di pi� di numero, aumentano i dipendenti pubblici (lo stato interviene
nella vita sociale). La piccola/media borghesia aveva un tenore di vita simile a quello degli operai
ma facevano di tutto pur di distinguersi (in questo clima di disagio nascono i partiti di estrema
destra).\\ [\baselineskip]
L'istruzione si diffonde sempre di pi�, piano piano. Pi� giornali vengono venduti, nascono i
giornali sportivi e si diffonde lo sport.\\ [\baselineskip]
Gli eserciti si rinforzano (leva obbligatoria) e gli ufficiali diffondono idee di patriottismo, 
\ldots. Favorivano lo studio delle lingue e la nascita di nuove idee.\\ [\baselineskip]
Il suffragio si allarga sempre di pi�. Il suffragio � universale maschile prima della WW1 e anche
in qualche paese femminile.

\subsection{Partiti socialisti}
I primi partiti sono quelli socialisti. La Seconda Internazionale si tiene a Parigi
nel 1889. Il pi� grande partito � quello \textbf{social-democratico tedesco}. L'obiettivo era di
coordinare i partiti per ottenere migliori condizioni lavorative per gli operai. Erano sostenitori
dell'\textbf{internazionalismo}. L'ideale di nazione � un ideale borghese, il proletariato non �
nazionale.\\ [\baselineskip]
Erano divisi in due correnti
\begin{description}
  \item[Rivoluzionari] Volevano i cambiamenti con violenza, senza riforme
  \item[Riformisti] Volevano i cambiamenti con graduali riforme, in modo pacifico
\end{description}
Tra i \emph{Riformisti}, \textbf{Bernstein} era uno dei pi� importanti. Nel 1899 pubblica 
\emph{``I presupposti del socialismo e i compiti della social-democrazia''}. I presupposti e gli 
ideali sono gli stessi di Marx ma lui ha commesso un errore: la situazione non sta peggiorando e la 
borghesia non si sta proletarizzando. Il crollo del capitalismo non � quindi vicino, � necessario 
migliorare la situazione dei lavoratori tramite riforme.\\ [\baselineskip]
Agli inizi del '900, si formano piccoli gruppi di rivoluzionari (estrema sinistra). Il primo era
guidato da \textbf{Lenin}. Il proletariato da solo non pu� fare la rivoluzione, ha bisogno del
partito come guida perch� non ha la coscienza di classe. Il partito � fatto da intellettuali che
pensano di capire l'economia. � composto da un'el�te di rivoluzionari per professione.\\ 
[\baselineskip]
Nel 1903 si tiene il congresso del PSD, a Londra. Lenin ottiene la maggioranza. Tra queste correnti
c'erano anche dei \emph{sindacalisti rivoluzionari} tra cui \textbf{Sorel}. Pubblica nel 1905
\emph{``Considerazioni sulla violenza''}. Erano critici contro i partiti socialisti che tendevano ad
allontanarsi dal proletariato ed erano guidati da chi viveva come un borghese. Il sindacato invece
era fatto da lavoratori che erano a stretto contatto con i proletari. L'azione spontanea � esaltata.
L'inizio della rivoluzione sarebbe stato uno sciopero generale che metter� in crisi l'economia
capitalista. � una forma di \emph{anarco-sindacalismo}.

\subsection{Partiti nazionalisti}
In questa societ� di massa si vengono a formare anche dei partiti nazionalisti. L'idea di fondo era
di valorizzare la forza e la potenza della nazione. Sono \textbf{interclassisti} in quanto tutte le
classi sociali devono collaborare per la forza della nazione. Il modello � l'esercito e la sua 
gerarchia. Sono a favore del protezionismo e dell'imperialismo. Le idee democratiche sono pericolose,
al potere deve starci chi ha veramente l'abilit�. La libert� deve essere ridotta.\\
C'erano alcuni partiti di spicco
\begin{description}
  \item[Pangermanesimo] nazionalismo tedesco che voleva riunire tutti i tedeschi in un unico stato
  \item[Revanescismo] nazionalismo francese che voleva la rivincita contro i tedeschi
  \item[Panslavismo] nazionalismo slavo per riunire tutti gli stati sotto la Russia
  \item[Inglese] per il colonialismo e l'impero
  \item[Italiano] \textbf{Enrico Corradini} � il primo ideologo. Usava un linguaggio marxista con
    significato nazionalista. Ci sono due tipi di nazioni: \emph{borghesi} (ricche, coloniali, 
    \ldots) e \emph{proletarie} (giovani, povere, sovrappopolate). L'Italia rientra in queste ultime.
\end{description}
Il \textbf{razzismo} � anche un fenomeno che � collegato al nazionalismo. La societ� umana � divisa 
in \emph{razze} che si differenziano non solo per le qualit� fisiche, ma anche per quelle morali e 
culturali che dipendono da fattori biologici.\\
\textbf{De Gobineau} � uno degli esponenti. Pubblica \emph{``Saggio sull'inuguaglianza delle razze
umane''}. Ci sono 3 razze: gialla, nera e bianca. La bianca (ariana = Europa centro-nord) � la
superiore sia sul piano fisico che intellettuale. Ha creato la cultura e solo quella ha i veri
valori. Il razzismo teme l'\textbf{ibridazione} ovvero la mescolanza fra razze. Il sangue non deve 
contaminarsi. Secondo De Gobineau sopratutto le classi superiori (di cui lui fa parte) rappresentano 
la razza ariana.\\
Legato al razzismo c'� anche l'\textbf{antisemitismo}. Comunit� ebraiche ci sono sempre state in 
Europa. Nel Medioevo erano \emph{infedeli}, dal '500 in avanti vivono in ghetti, solo nel '700 
cominciano ad integrarsi meglio. L'antisemitismo non � scomparso ma modificato. Dopo l'emancipazione,
gli ebrei si sono assimilati alla societ� e alcuni hanno anche avuto successo. Gli ebrei erano una
razza che si contrapponeva a quella ariana, anche dopo la conversione si rimaneva ebrei. La loro
pericolosit� deriva dalla loro somiglianza a noi (Chamberlain, \emph{``Fondamenti del \rom{19} 
secolo''}).
Due sono i casi-esempio di anti-semitismo che vanno ricordati
\begin{description}
  \item[Caso Dreyfuss] Dreyfuss era un capitano francese che faceva parte dello Stato Maggiore. Era
    ebreo. Nel 1894 i servizi segreti francesi scoprono che nello Stato Maggiore c'era una spia.
    Essendo l'unico ebreo, Dreyfuss fu sospettato e messo sotto processo. Vengon create prove false
    e poi condannato. Emergono ora due correnti di pensiero
    \begin{description}
      \item[Dreyfussardi] A favore di Dreyfuss (democratici, socialisti)
      \item[Anti-dreyfussardi] Contro Dreyfuss (nazionalisti, Chiesa)
    \end{description}
    Dopo qualche anno il processo viene rivisto e nel 1906 Dreyfuss � stato reintegrato.
  \item[I (falsi) protoclli dei Savi di Sion] � un libro in cui si descrive un complotto ideato
    dai rabbini per fare in modo che gli ebrei governino il mondo. Fu considerata la dimostrazione
    della pericolosit� degli ebrei. Si scopr� poi che in realt� era un falso scritto dai servizi
    segreti zaristi in quanto alcune parti erano ricopiate da romanzi di bassa lega dell'800.
    Nonostante questo c'� chi ancora crede siano veri.
\end{description}
Sotto questi influssi nasce il \textbf{Sionismo} overo il nazionalismo ebraico. 
\emph{Theodor Herzl} era un ungherese, fondatore del sionismo. Era il tipico ebreo assimilato, 
socialista e non religioso. Va a Parigi a seguire il caso Dreyfuss e nota che gli ebrei vogliono 
assimilarsi ai cristiani ma non possono perch� l'antisemitismo � troppo forte. Serve uno stato 
ebraico. Nel 1896 scrive \emph{``Lo Stato Ebraico''}. Deve nascere per accordi internazionali ed 
essere neutrale. Lo stato non � necesssariamente la Palestina. Viene creata un'organizzazione 
sionista che si riunisce la prima volta a Basilea nel 1897. L'unico territorio che avesse senso er
la Palestina che era sotto l'impero Ottomano. Non ottennero nulla. Nel '900 cambiano strategia 
facendo emigrare gli ebrei verso la Palestina dove avrebbero comprato terra e fatto i contadini. 
Dopo la WW1 cominciano i problemi in quanto gli arabi non volevano si costituisse uno stato.

\subsection{Partiti cattolici}
Nel parlamento a sinistra c'erano i socialisti, a destra i nazionalisti e in centro i cattolici.
In Germania nasce la CDU. Erano partiti sotenuti dalla Chiesa.\\
Pio \rom{9} era molto conservatore politicamente, invitava a non impegnarsi politicamente. Muore 
nel '76. Il successore Leone \rom{13} cambia atteggiamento. Nel 1891 pubblica 
\textbf{\emph{``Rerum Novarum''}} che non � altro che la dottrina sociale della Chiesa. Interviene
per la prima volta sulla "questione operaia". I lavoratori hanno dei doveri nei confronti del 
datore di lavoro (impegno, fedelt�, rispetto) ma anche dei diritti (giusto stipendio, corretti
trattamenti). Non � una riforma socialista, anzi, critica i socialisti (sono atei, senza propriet�
privata, crea lotta di classe). Voleva evitare una perdita di contatto con i lavoratori. Non �
nemmeno liberista (critica l'individualismo, esclude lo stato dalla vita economica). Rifiuta i
sindacati ma promuove le corporazioni. In Italia esistevano sindacati ma non corporazioni, non
sarebbero pi� stati credibili. I sindacati cattolici non sempre seguivano il Papa, se avessero 
rinunciato agli scioperi, sarebbero sembrati deboli.\\
Dopo la Rerum Novarum molti cattolici si sentono spinti verso la vita sociale. Agli inizi del '900
comincia a nascere la \textbf{Democrazia Cristiana}: Romolo Marri e Luigi Sturzo sono sacerdoti che
volevano fare un partito. Pio \rom{10} era pi� conservatore del predecessore e blocca l'iniziativa.
Far nascere in Italia un partito significherebbe riconoscere lo stato Italiano. Sturzo abbandona,
Marri invece continua, abbandona la Chiesa. Proprio Sturzo per� nel '19 con il sostegno della 
Chiesa fonder� il partito Cattolico.

%!TEX ROOT=storia.tex

\section{Et� Giolittiana}
L'et� giolittiana va dal 1900 al 1914.

\subsection{Economia}
Il periodo � caratterizzato da generale crescita economica. L'industrializzazione cresce ma solo in
alcune zone (Lombardia, Piemonte, Liguria) e solo alcuni settori.

\subsection{Societ�}
C'� una fortissima emigrazione (circa 500k all'anno).

\subsection{Politica}
Giolitti era un piemontese liberale. La formazione dei sindacati era inevitabile in quanto � una 
tendenza causata dall'industrializzazione. Lo stato \textbf{non deve impedire l'organizzazione}
perch� altrimenti si organizzano clandestinamente contro lo Stato. Non deve reprimere manifestazioni
pacifiche.\\
Giolitti cerc� alleanze con socialisti (offre a Turati un ministero se il PSI si fosse alelato con
il governo, rifiuta per non dividere il partito). Quando Giolitti era presidente del Consiglio, era
anche ministro dell'Interno.\\
Nel 1906 � fondata la \textbf{CGL} legata al partito socialista. Guidata da socialisti riformisti.\\
Nel 1910 � fondata la \textbf{Confindustria}.\\
Giolitti ha portato avanti importanti riforme tra cui le \textbf{prime leggi per regolamentare il 
lavoro} (obbligo del riposo festivo, vietato per donne e bambini il lavoro notturno). Nel 1911 viene 
creata l'INA (Istituto Nazionale Assicurazioni) a cui � dato il monopolio delle assicurazioni sulla 
vita. In questo modo i lavoratori erano pi� sicuri e i fondi andavano a formare un sistema 
previdenziale. Nel 1913 viene data una pensione agli infortunati sul lavoro.\\
Vengono \textbf{nazionalizzate le ferrovie} cos� si sarebbe risparmiato e si sarebbero collegati 
anche i punti pi� sfavorevoli. Fa anche riforme per il sud che dovevano favorire lo sviluppo
(costruito un acquedotto in Puglia, \ldots), in realt� non ebbero grandi risultati in quanto i
provvedimenti erano clientelari (favoritismi, \ldots).\\ [\baselineskip]
Nel 1912 � stata varata una \textbf{riforma elettorale} che permetteva il suffragio universale 
maschile per chi avesse avuto 21 anni e fato la leva militare o 30 altrimenti. Circa 9 milioni di
elettori. Uninominale a doppio turno (1 deputato per collegio, 50\% dei voti al primo turno, 
ballottaggio dei primi due). Nel 1913 le prime elezioni di massa. I socialisti erano organizzati per 
massa, non i liberali. Cos� si form� il \textbf{Patto Gentiloni} che sanciva che i candidati liberali
cattolici sarebbero stati sostenuti dalla Chiesa se poi in parlamento non avessero sostenuto 
provvedimenti che la Chiesa riteneva scomodi (divorzio, scuole cattoliche, \ldots). La Chiesa temeva
i socialisti, viene cos� eliminato il ``Non Expedit'' e i cattolici entrano nella vita dello stato
italiano.

\subsection{Esteri}
Tra il 1911 e il 1912 Giolitti intraprende una guerra coloniale. Furono presi accordi segreti con la 
Francia: l'Italia permette concede il Marocco alla Francia, la Francia non ostacola l'Italia. La
Chiesa sosteneva la guerra come fosse di civilt�. La guerra fu durissima, quasi barbara 
(avvelenamenti, capi di concentramento).\\
Nel 1912 si stipula la \textbf{Pace di Losanna}. La Libia ora � colonia Italiana. La Libia era 
allettante per l'economia secondo Giolitti, non tutti erano d'accordo (Sanvemini disse che la Libia
era una ``Scatola di Sabbia'').

%!TEX ROOT=storia.tex

\section{La Prima Guerra Mondiale}
Come ogni fenomeno complesso, la guerra non ha avuto una sola causa. Forse nessuno dei fattori, presi
singolarmente, sarebbe bastato.\\
Gli storici marxisti sottolineavano le \textbf{cause economiche} (concorrenza industriale, 
protezionismo e guerre doganali, concorrenza coloniale).\\

\subsection{Bismarck e la Germania}
Bismarck � rimasto cancelliere fino al 1890. Fino ad allora non aveva fatto una politica coloniale
in quanto sarebbe entrato in conflitto con l'Inghilterra e doveva mantenere buoni rapporti con la
Russia (il suo principale obiettivo era isolare politicamente la Francia). Bismarck si dimette nel
1890.\\
Guglielmo \rom{2} voleva una politica pi� aggressiva, coloniale. Quindi minaccia gli Inglesi creando
una flotta che possa competere con la loro. \textbf{Francia e Inghilterra si accordano sulle
colonie}.\\
I tedeschi ottengono un appalto per la costruzione di una ferrovia da Istanbul a Baghdad. Favoriscono
cos� il commercio con l'impero Ottomano delle merci tedesche.\\
Il Marocco era diviso a met� tra Spagnoli e Ottomani. Sia la Francia che la Germania lo volevano.
\textbf{Due crisi Marocchine}: 1905--1906, vinta dall'alleanza Inghilterra-Francia e 1911. 
\textbf{Francia e Germania sono sull'orlo della guerra}.\\
La Germania aveva solo l'Austria come alleata ma era continentale, senza sbocchi sul mare. L'Italia
aveva accordi con la Francia. Ormai la guerra pareva come l'unica maniera per realizzare i piani
tedeschi.\\ [\baselineskip]
Negli anni '90 Francia e Russia fanno un'alleanza militare, cos� come anche Inghilterra e Russia.
Ci sono ora due schieramenti
\begin{description}
  \item[Triplice Alleanza] Germania, Austria, Italia
  \item[Triplice Intesa] Francia, Inghilterra, Russia (accordi bilaterali)
\end{description}

\subsection{L'inizio della Guerra}
Nei Balcani c'era un contrasto fra Austria e Russia. Molti stati ottengono l'indipendenza, tra cui
la Serbia (che conteneva anche Croazia e Slovenia). Si forma cos� la \textbf{Iugoslavia} ovvero lo
Stato degli Slavi del Sud. Alcune delle popolazioni erano sotto l'Austria per�. La Russia era alleata
della Serbia. La guerra era ormai scontata anche per i movimenti nazionalisti che si andavano
diffondendo che offrivano una visione della guerra come modo per dimostrare la forza.\\ 
[\baselineskip]
Il \textbf{21 giugno 1914} Gavrilo Princip assasina Franecsco Ferdinando per protesta dell'annessione
della Bosnia all'Austria. Lo fa con il sostegno dei servizi segreti Serbi. Il \textbf{21 luglio}
scoppia la guerra. Fra il 28 e il 4 agosto si attivano le alleanze: \textbf{Austria e Germania} 
contro \textbf{Russia, Francia, Inghilterra e Serbia}.

\subsubsection{La questione Italiana}
\textbf{Antonio Salandra} guida il governo in modo liberale, di Destra. \textbf{Sonnino} � il
ministro degli esteri. Dichiara la
\textbf{neutralit�} dicendo che l'alleanza era difensiva. La popolazione si divide in due: 
\textbf{Neutralisti} e \textbf{Interventisti}.\\
\textbf{Giolitti} voleva la neutralit� in quanto non sarebbe stata sostenibile un'altra guerra dopo
quella in Libia. Contrattando la neutralit� invece si sarebbe potuto ottenere molto. La Chiesa
condivideva. \textbf{La Chiesa} esprimeva i pensieri dei contadini: una guerra contro gli Austriaci,
cattolici, non era vista bene (Benedetto \rom{15}) era il nuovo papa. Anche i \textbf{socialisti} 
erano neutralisti in quanto si sarebbe intaccato l'internazionalismo.\\
\textbf{I Nazionalisti} invece erano interventisti, per dimostrare la propria forza, i 
\textbf{democratici} si ricollegavano a Mazzini e all'idea di un completamento del Risorgimento
italiano con l'annessione delle terre irredente. Anche i \textbf{sindacati rivoluzionari} erano 
favorevoli in quanto ritenevano che la guerra avrebbe scosso il capitalismo e fatto crollare, creando
i presupposti per una rivoluzione. Infine anche i \textbf{liberali conservatori}.\\ [\baselineskip]
\textbf{Dopo mesi, entriamo in guerra contro Austria e Germania}. Per convenienza. Il 26 aprile 1915
fu stipulato segretamente il \textbf{Patto di Londra} tra l'Italia e l'Intesa. Entro un mese l'Italia
sarebbe dovuta entrare in guerra contro l'Austria, in cambio avrebbe ricevuto
\begin{enumerate}
  \item Le terre irredente
  \item L'Alto-Adige
  \item L'Istria
  \item La Dalmazia e il porto di Valona
  \item Il controllo della politica estera dell'Albania
  \item Il Dodecaneso
  \item Un bacino di carbone in Turchia
  \item Alcune colonie tedesche in Africa
\end{enumerate}
L'Italia ora doveva entrare in guerra ma i neutralisti erano in maggioranza in parlamento e tra il
popolo. \textbf{Salandra si dimette}. Ci furono molte manifestazioni causate da questa crisi di
governo (studenti, borghesi, socialisti, \ldots). Il governo allora \textbf{lascia liberi gli
interventisti} e \textbf{asseconda i socialisti} per dimostrare che l'Italia voleva la guerra.\\
Vittorio Emanuele \rom{3} chiama Giolitti e lo informa sul patto di Londra. Giolitti, temendo una
crisi istituzionale della monarchia ed essendo comunque piemontese, abbandona Roma (ovvero rinuncia
a tenere l'Italia fuori dalla guerra). Richiama Salandra al Quirinale e gli conferisce poteri 
speciali (20 maggio) e finanziamenti per sostenere la guerra. \textbf{Il parlamento vota l'entrata in
guerra con il sostegno anche dei liberali giolittiani} (non dei socialisti). Il 
\textbf{24 maggio 1915} l'Italia entra in guerra contro l'Austria.\\
L'entrata in guerra � importante anche per la politica interna in quanto Salandra, Sonnino e il re 
sono riusciti a togliere il potere al parlamento e a dare al re il governo.

\subsection{Lo spirito del combattente}
Perch� combatte un soldato?
\begin{description}
  \item[Per solidariet�] nei confronti dei compagni
  \item[Per rassegnazione] dopo il primo inverno e dopo l'abitudine
\end{description}
Nonostante le nobili intenzioni, i fenomeni di diserzione e ribellione non erano infrequenti. 
L'istinto di sopravvivenza e il sottrarsi alla morte avevano la meglio.\\
Al polo opposto stava un'idologia ``bellicista'', secondo la quale la guerra � la massima 
esaltazione ed espressione pi� alta dell'esperienza umana. Ernst J�nger riteneva che la guerra 
fosse un momento costitutivo di una nuova razza superiore alle precedenti. In Italia l'equivalente
di questi erano gli ``arditi'', capaci di rovesciare le regole tradizionali di combattimento (secondo
Giorgio Rochat).

\subsection{La partecipazione delle masse nella guerra}
Oltre alle innovazioni tecnologiche, il vero motore della guerra era la forza d'urto delle masse
di uomini mandati al fronte. Quelli che combattevano non erano altro che contadini, operai, impiegati
pubblici o privati che avevano alimentato l'intervento Italiano. La guerra dunque fu un modo per
\textbf{rafforzare lo spirito delle masse} che ora diventavano le vere protagoniste. Omogeneizz�
inoltre tutti gli strati sociali accomunandoli con il concetto di ``nazione''.\\
In Italia in particolar modo la guerra fu vista come un completamento del Risorgimento. Nei primi
50 anni dell'unit� buona parte della massa non si sentiva unita, non si sentiva ``Italiana''.
Al fronte i ceti pi� bassi avevano scoperto l'ideale di nazione pi� grande del loro paese d'origine.
Il destino comune, la lingua comune, amalgamavano tutti gli strati sociali.

\subsection{Lo stallo del 1915--1916}
Con l'entrata in guerra dell'Italia si apre un nuovo fronte, meridionale. La guerra era lenta, cos�
detta di \textbf{logoramento} in trincea. Nessuno dei due fronti otteneva vere e proprie vittorie
e riusciva ad avanzare. Questo stallo danneggia prevalentemente gli \textbf{Imperi centrali} in 
quanto sono isolati dal resto del mondo e il blocco commerciale li danneggiava.\\
Per smuovere la situazione, gli imperi tedeschi concentrano le forze a \textbf{Verdun} dove la
battaglia si protrasse per oltre 5 mesi con pi� di mezzo milione di morti. I tedeschi falliscono
in questo tentativo.\\
Provano anche a sfondare via mare, con una battaglia contro gli Inglesi nello \textbf{Jutland},
vicino allo Skagerrak.\\
Tentarono quindi la guerra \textbf{sottomarina} per affondare le navi inglesi. Ottenne buoni 
risultati ma vide anche l'entrata in guerra degli Stati Uniti.\\ [\baselineskip]
Sul fronte meridionale i tedeschi fecero una \textbf{spedizione punitiva} contro l'Italia che
vide l'occupazione di Asiago come risultato. L'impreparazione dell'esercito Italiano port� Salandra
a dimettersi. Dopo una cruenta battaglia le truppe italiane furono capaci di \textbf{prendere 
Gorizia} (9 agosto 1916).\\ [\baselineskip]
Per fronteggiare il malcontento, molti governi formarono dei \textbf{governi di unit� nazionale}, 
ovvero dei governi di grandi alleanze. Dopo le dimissioni di Salandra, in Italia si form� il governo
\textbf{Boselli}, in Francia \textbf{Briand} e in Gran Bretagna \textbf{David Lloyd George}. In
Germania tutto il potere fu concentrato nell'imperatore e nelle pi� alte gerarchie militari.\\
Questo accentramento porta i governi a pianificare e dirigere direttamente la guerra. Infatti ci fu
l'influsso sulle aziende di \textbf{innovare} e migliorare i propri prodotti. Queste modifiche
portarono anche alla \textbf{militarizzazione del lavoro in fabbrica} e alla limitazione delle
libert� sindacali. La guerra inoltre era costosa e questo port� all'introduzione di nuove tasse
coon conseguente aumento del debito pubblico e inflazione.

\subsection{La guerra ``Mondiale''}
Nel 1917 la \textbf{Russia} si ritira dal conflitto a causa della rivoluzione bolscevica. Cos� entra
una nova nazione: gli \textbf{Stati Uniti}. Lo zar a causa della rivoluzione che ne esce, �
costretto ad abdicare (rivolte di operai e soldati nella capitale) e si forma un governo repubblicano
provvisorio. \textbf{Karenskij} era a capo del governo e decise, dopo una sconfitta militare, di
uscire definitivamente dalla guerra. Gli Stati Uniti entrano in guerra principalmente a causa della
guerra sottomarina che stava flagellando gli inglesi.\\ [\baselineskip]
Se all'inizio la popolazione aveva preso di buon grado l'entrata in guerra, ora si diffondeva
malcontento, stanchezza e insofferenza. I soldati erano abbandonati a s� stessi, malnutriti e quasi
sommersi dalle trincee. L'utilizzo di nuove armi (bombe a mano, gas, lanciafiamme) e le pessime
condizioni, favorivono \textbf{diserzioni di massa} e ammutinamenti. Il ``disfattismo'' ormai dilgava
in tutte le fasce della gerarchia.\\
Per fermare questa ondata, i governi agiscono con battente propaganda e devono anche arginare il 
problema del ``fronte interno'', ovvero di tutti quegli strati sociali che per varie ragioni si
opponevano alla guerra. La scarsit� di cibo e di beni era anche aumentata dai prezzi esorbitanti che
i proprietari delle aziende che avevano avuto il monopolio, fissavano.\\
In Francia si cambi� gli uomini al governo: \textbf{P�tain} � il nuovo generale e \textbf{Clemenceau}
� il nuovo primo ministro, determinato alla guerra.\\
In Germania si fece qualcosa di simile: si \textbf{militarizzarono le industrie} e il potere si
concentr� nel capo di stato maggiore \textbf{von Hindenburg}.\\ [\baselineskip]
Nel 1917, gli imperi centrali provano uno \textbf{sforzo offensivo eccezionale} nella speranza
di risolvere il conflitto in breve. Sferrarono un attacco nell'Isonzo. Il generale Italiano era
\textbf{Luigi Cadorna} che con il suo esercito non resistette all'urto. Fu la \textbf{Disfatta di
Caporetto} che port� una ritirata fino al Piave. Si forma cos� un \textbf{nuovo governo} e l'esercito
fu affidato a \textbf{Armando Diaz}.

\subsection{La disfatta di Caporetto}
\textbf{Il 24 ottobre 1917 un attacco austro-tedesco sfonda la linea italiana}. L'esercito italiano
fugge verso ovest con circa 300 mila prigionieri e altrettanti sbandati. I tedeschi sono fermati sul
Piave. 11 mesi dopo, con la battaglia di Vittorio Veneto si riconquista il Veneto e il Friuli.\\
\textbf{Cadorna sapeva ci sarebbe stato un attacco} eppure non ha fatto niente. Sul campo di 
battaglia in montagna, per ogni soldato c'era bisogno di 4 uomini. Dopo la sconfitta, Cadorna
\textbf{accusa} i soldati di essere stati vili ed aver abbandonato il campo di battaglia. \textbf{Un
generale non dovrebbe mai condannare i propri soldati}.\\
Le truppe italiane si sentivano ``distaccate'' dalla guerra, ecco perch� molti disertavano. O almeno
cos� si diceva. In realt� \textbf{quasi nessuno disert� e fugg� dal campo di battaglia} tranne i
colonnelli e i generali. Tutti i reparti continuarono a combattere. Alcuni generali rimasero
al fronte e combatterono assieme ai soldati, molti per� abbandonarono i propri reparti.\\
La prima guerra mondiale � stata una \textbf{guerra di popoli} in cui si voleva solo l'annientamento
del nemico. I reparti italiani avevano 600 cannoni, quelli tedeschi pi� di 1200 (li avevano nascosti
e sapeva dove fossero quelli italiani).\\
La leggenda dello ``sciopero militare'' di Caporetto nacque da un libro di \textbf{Alberti}. Infatti
circa 7000 prigionieri descrivevano le azioni eroiche del battaglione che continuava a combattere
nonostante tutto.

\subsection{La fine della guerra}
I tedeschi spinti dalla vittoria di Caporetto, sferrarono un attacco sul fronte occidentale vicino a
\textbf{San Quintino} in cui l'Intesa fu sfondata fino alla \textbf{Marna}. La battaglia per�
riprese con l'uso di nuove tecnologie (cannoni tedeschi e arerei e carri armati inglesi).\\
A luglio con l'arrivo degli americani il fronte fu sfondato verso \textbf{Amiens} e cominci� 
l'avanzata anche a sud, in Italia fino a Vittorio Veneto. \textbf{Il 4 novembre 1918 fu firmato
l'armistizio tra Austria e Italia}. L'impero asburgico si stava disgregando e la Germania deve
arrendersi anche lei, l'11 novembre.\\ [\baselineskip]
Nel gennaio del 1919 a \textbf{Versailles} si ritrovano i paesi vincitori (Francia, Gran Pretagna,
Stati Uniti e Italia). C'erano due schieramenti diversi
\begin{description}
  \item[Europa] capitanata da \textbf{Clemenceau} voleva mantenere le tradizionali annessioni
    territoriali, incentrate sull'egemonia della Francia e della Gran Bretagna in Europa.
  \item[Stati Uniti] con \textbf{Wilson} voleva affermare il principio dell'autodeterminazione,
    ovvero i vincitori dovevano solo ridisegnare la mappa geo-politica dell'Europa.
\end{description}
L'Italia no riusc� ad ottenere l'annessione di Fiume e della Dalmazia conducendo ad un'insistita
campagna nazionalistica per la vittoria mutilata.\\ [\baselineskip]
Prevalse la linea dura di Clemenceau che \textbf{impose} il trattato di Versailles ai tedeschi.
Esso stabiliva
\begin{itemize}
  \item La restituzione alla Francia dell'\textbf{Alsazia e Lorena}, lo smembramento dei possedimenti
    coloniali e il pagamento dei danni di guerra
  \item La creazione di \textbf{Austria}, \textbf{Ungheria} e \textbf{Cecoslovacchia}, 
    \textbf{Iugoslavia}
  \item La \textbf{Polonia} fu ricostruita
  \item L'Italia ottenne l'\textbf{Alto Adige}, \textbf{Trieste} e l'\textbf{Istria} 
  \item Gli altri territori furono messi sotto il controllo Francese o Inglese
\end{itemize}
Infine, su volere di Winson, fu istituita la \textbf{Societ� delle Nazioni} che aveva lo scopo di 
tutelare la pace facendo da arbitro nelle controversie internazionali. Questo progetto per�
non and� molto lontano in quanto Russia, Germania e gli stessi Stati Uniti restarono fuori. Divenne
quindi un mezzo per Francia e Gran Bretagna per esercitare il potere.

%!TEX ROOT=storia.tex

\section{Rivoluzione Russa}
La Russia era il paese più vasto, un impero multietnico con più di 100 milioni di abitanti di cui
la metà russi, gli altri di varie etnie (Ucraini, Armeni, \ldots) con lingue e culture diverse. Non
sempre accettavano di buon grado il governo russo.\\
L'80\% della popolazione era contadina, erano analfabeti e solo nel 1861 era stata vieteta la
servitù della gleba. Solo alla fine dell'800 comincia un minimo di rivoluzione industriale (a San
Pietroburgo (la capitale), a Mosca (per i tessuti), negli Urali (per il ferro) e nel Mar Nero (per il
petrolio)). L'agricoltura era arretrata e i capitali per le industrie erano principalmente 
provenienti dall'estero. C'era poca borghesia e tanta nobiltà che non aveva la mentalità 
imprenditoriale.\\
Lo Zar era \textbf{Nicola \rom{2}} che regnava con un potere autocratico, senza costituzione, 
parlamento, diritti o libertà. La Chiesa ortodossa legittimava il potere dello zar.\\ [\baselineskip]
Nel 1905 era in \textbf{guerra con il Giappone}. La Russia perde e aumenta il malcontento. 
Manifestazioni di protesta e l'esercito le reprime con la forza. Continuano e lo zar concede la
\textbf{Duma}, un parlamento con potere legislativo, e libertà di stampa e associazione. Negli anni
seguenti però pian piano riduce i poteri alla Duma e riduce anche i diritti e il diritto di voto.\\
Nel 1914 arriva in guerra con circa 6 milioni di uomini. I più numerosi ma i peggio armati. I beni
di prima necessità scarseggiano sia al fronte che in città.\\
Nel \textbf{1916} lo zar convoca la Duma per ricevere sostegno per introdurre nuove tasse, la Duma si
oppone e viene sciolta. I leader politici si tengono in contatto.

\subsection{Inizio delle manifestazioni}
\textbf{23 febbraio 1917} a Pietrograto (= Pietroburgo) si tiene la prima manifestazione 
rivoluzionaria. Le successive manifestazioni vengono represse dall'esercito ma successivamente i
soldati sostengono le manifestazioni. Lo zar richiama dal fronte alcune truppe fedeli. Non arrivarono
mai in quanto bloccate dai ferrovieri, a favore della rivoluzione. Le manifestazioni si diffondono
fino a Mosca e al fronte; ad inizio marzo la situazione è fuori controllo.\\
I generali consigliano l'abdicazione dello zar, infatti nel \textbf{2 marzo 1917} lo zar abdica a
favore del fratello Michele che rinuncia al trono. In Russia termina così la dinastia Romanov. La
Duma elegge un governo provvisorio. Gli obiettivi di questo governo erano
\begin{description}
  \item[Continuare la guerra] Mantenere un legame con la Francia e l'Inghilterra, arrivare alla
    conquista
  \item[Democratizzare lo stato] Fare una costituzione che garantisca diritti e libertà. Si poteva
    fare solo dopo la fine della guerra
  \item[Promulgare una riforma agraria] Dare terra ai contadini
\end{description}
Il governo era sostenuto dai \textbf{cadetti} (liberali, democratici, borghesi russi) e dal
\textbf{partito social-rivoluzionario} (non è marxista, si richiamava alla tradizione russa che aveva
come base i contadini russi e i villaggi (Mir)) e i \textbf{menscevichi} (marxisti riformisti,
deve svilupparsi la politica come democrazia rappresentativa). Lenin nel frattempo era in esilio,
i capi bolscevichi erano divisi sul sostenere o meno il governo.\\
Accanto al governo si formano i \textbf{Soviet} ovvero dei consigli di fabbrica o di settore, eletti.
C'era il soviet della città che riunisce i rappresentanti dei locali soviet. Va contro l'idea
liberal-democratica rappresentativa. Il soviet voleva essere un esempio di \textbf{democrazia
diretta} anche se in realtà non era così.\\
Si ottiene così una situazione di diarchia: da un lato c'è il governo provvisorio della Duma, da un
altro i soviet.

\subsection{Lenin al potere}
Lenin torna a Pietrogrado grazie ai servizi segreti tedeschi. Tornato propone le \textbf{Tesi 
d'aprile} che in generale raccolgono il programma leninista. Lenin \textbf{vuole la pace subito}, il
governo (borghese) invece vuole continuare la guerra con obiettivi imperialisti.\\
La \textbf{Quarta tesi} è importante: i bolscevichi erano in minoranza nei soviet. Così i bolscevichi
vogliono prima prendere la maggioranza nei soviet e poi trasferire tutto il potere ai soviet,
esautorando il governo. Quindi Lenin vuole \textbf{fare la rivoluzione, subito} (andando un po'
contro Marx). Il modello era la Comune di Parigi. Lenin vuole nazionalizzare la terra senza 
indennizzo per i proprietari, così aumenta il consenso tra i contadini.\\
Un'altra tesi \textbf{voleva cambiare il nome del partito in ``Partito Comunista''}. L'ultima creava
la \textbf{Terza Internazionale} dei partiti rivoluzionari.\\ [\baselineskip]
Vengono proposte delle ffensive contro i tedeschi ma senza alcun successo (enormi diserzioni,
screditavano il governo, aumenta il malcontento). Nel luglio del 1917 ci sono \textbf{manifestazioni
a Pietrogrado}, il governo (Karenskij era a capo) risponde con la forza mettendo fuori legge il 
partito bolscevico.\\
Ad agosto il generale Kornikov voleva prendere il potere portando i propri soldati a Pietrogrado ma
i bolscevichi interrompono i collegamenti ferroviari e non ha successo. Da questo momento i 
bolscevichi soon organizzati.\\ [\baselineskip]
Tra \textbf{agosto e settembre 1917} i bolscevichi prendono la maggioranza nei soviet delle maggiori
città, il mese successivo l'obiettivo diventa il potere. Lenin propone il colpo di stato, Trotzkij
organizza le guardie rosse. \textbf{24--25 ottobre} (6--7 novembre) le guardie rose prendono il 
controllo delle vie di comunicazione e dei centri di potere. Assaltano il Palazzo d'Inverno in cui si
era riunito il governo provvisorio. I ministri fuggono.\\
Viene istituito il \textbf{Consiglio dei commissari del popolo} (eletti tramite democrazia diretta in
teoria). Furono stabiliti due decreti
\begin{description}
  \item[Sulla guerra] Appello ai paesi in guerra per interromperla senza indennità o spartizioni. 
    Pace incondizionata subito. Questo mette a favore i proletari
  \item[Sulla terra] Latifondisti espropriati senza indennizzo, creare aziende statali per 
    organizzare, in realtà i contadini si prendevano la terra e la amministravano privatamente. 
    Potava il favore dei contadini
\end{description}
\textbf{A gennaio del 1918 viene convocata l'assemblea costituente}. La maggioranza è ai socialisti 
rivoluzionari, i bolscevichi hanno circa un quarto dei voti. Il primo giorno i menscevichi e i
socialisti rivoluzionari criticano i bolscevichi, così l'assemblea viene sciolta la sera stessa
in quanto una forma di governo borghese. Si allontanano così dalla democrazia. I bolscevichi quindi
governano da soli con un minimo supporto della parte più esterma dei socialisti rivoluzionari.
Tortzkij tratta la pace con la Germania: \textbf{Trattatto di Brest-Litovsk}. La Russia perde un
quarto dei territori riconoscendo l'indipendenza dell'Ucraina, Estonia, Lettonia e Lituania.\\
[\baselineskip]
Lenin doveva fare un governo comunista da zero, con tutto il mondo ostile. Era fiducioso che la
rivoluzione fosse vicina anche in occidente. I primi provvedimenti che fa sono:
\begin{itemize}
  \item Nelle fabbriche l'organizzazione del lavoro la faceva il soviet
  \item Abolita la leva obbligatoria
  \item Uguaglianza nei confronti della legge
\end{itemize}

\subsection{La nascita dello stato sovietico e la guerra civile}
Il governo comunista era forte fuori dalle città principali dove invece le \textbf{armate bianche},
guidate da generali zaristi, vincevano. A luglio la famiglia dello zar e lo zar stesso vengono
fucilati. Le armate sono aiutate dalla comunità internazionale per viveri e denaro. Le truppe inviate
nel 1918 però erano per buona parte contadini che sentivano delle riforme nella russa comunista.\\
Il governo di Lenin è in pericolo, vuole fare un esercito e quindi reintroduce la leva obbligatoria.
Vengono anche richiamati degli ufficiali zaristi a guidare l'esercito. Trotzkij organizza
\textbf{l'Armata Rossa}. Le industrie belliche vengono fatte lavorare a pieno regime (anche se erano
gestite dai soviet, erano sottomesse alle direttive del governo). Nelle campagne i contadini 
nascondono i raccolti e li vendono al mercato nero perché il governo fissava i prezzi. Il governo
attua una \textbf{politica di requisizione}, portando via il raccolto. Lenin definisce questa 
politica \textbf{``Comunismo della guerra''}, sul modello tedesco, questa è la dittatura del 
proletariato, non l'autentico comunismo. Questo comporta anche l'abolizione di libertà di 
associazione politica (mono-partitismo). Viene istituita la CEKA, la polizia politica.\\
[\baselineskip]
Nel 1920 la Polonia attacca la Russia tentando di conquistare territori approfittandone della 
debolezza. Nulla di fatto. Durante la guerra civile era stata riconquistata l'Ucraina. \textbf{Si
ribella la base navale di Kronstadt} criticando il governo leninista in quanto troppo autoritario e 
non marxista. Viene mandata l'armata rossa, metà morti e gli altri imprigionati.\\
Nei primi mesi del 1921 \textbf{la guerra civile finisce} e in marzo si riunisce il decimo congresso
del partito comunista dove vengono prese due decisioni sull'organizzazione interna e sull'economia.
\begin{description}
  \item[Organizzazione interna] Viene condannato il frazionismo, non ci devono essere correnti di 
    partito
  \item[Economia] (NEP) La produzione industriale viene ridotta al 13\%, Lenin vuole reintrodurre
    alcune cose capitaliste (consentire agli imprenditore qualche dipendente (circa 10), consentire
    il commercio al dettaglio, consentire ai propietari terrieri di avere dipendenti). I livelli
    più alti dell'economia erano statali, quelli più bassi privati.
\end{description} 
Trotzkij era contrario in quanto rivedeva una rinascita della borghesia. Lenin sperava di migliorare
l'economia in ginocchio. Queste riforme durano fino al 1928.\\ [\baselineskip]
Nel 1924 viene fatta una \textbf{costituzione} (la più importante di tutte quelle che furono fatte).
Formalmente era una repubblica federale (nasce l'\textbf{URSS}), in realtà non era così. Il 
parlamento è il congresso dei soviet, era una dittatura del parrtito.\\
L'ultima tesi di Aprle voleva una Terza Internazionale, a Mosca viene istituito il \textbf{Cominter},
l'internazionale comunista. Anche partiti europei entrano a farne parte. Si poteva solo se si 
cambiava nome in ``Partito Comunista'' e avere come modello il leninismo. I riformisti inoltre
vanno espulsi dal partito.

%!TEX ROOT=storia.tex

\section{Dopo guerra}
In totale ci furono circa 65 milioni di soldati coinvolti, 10 milioni sono morti al fronte, 20 a 
causa dell'influenza spagnola. Ci furono inoltre milioni e milioni di feriti ed invalidati. Per 
sostenere i veterani del fronte si creano \textbf{pensioni} di guerra, d'invalidità, \dots

\subsection{Economia e società}
In campo economico la guerra fu distruttiva. Oltre alle enormi spese di ricostruzione, gli stati 
(tranne gli USA) uscirono \textbf{indebitati}, gli USA avevano da soli metà delle riserve auree del 
pianeta. I prezzi di conseguenza erano regolati dallo stato, anche se la regolazione era considerata
solamente provvisoria. Nel 1919 vengono eliminate le leggi e si torna al \textbf{libero mercato} e 
questo provoca un'enorme inflazione e il conseguente aumento dei prezzi. Vengono danneggiati quelli
con lo stipendio fisso.\\
È necessario inoltre ristabilire la produzione industriale e tornare a trasformarla in quella 
originale dopo la guerra. Le grandi aziende avevano guadagnato molto e ora devono riconvertirsi,
però ciò richiede tempo e \textbf{molti operai furono licenziati}. Questo genera malcontento e i 
sindacati e i partiti socialisti vedono un boom di iscritti. Tra il 1919 e il 1920 ci furono molti
scioperi e perciò venne definito il \textbf{biennio rosso}, sulla spinta socialista.\\
Ai contadini era stata promessa la terra, una riforma agraria che però non fu mai attuata. Infatti
durante il biennio roso molti contadini occupavano da sè la terra tentando una propria rivoluzione.\\
La grande borghesia (industriali) era molto più ricca di prima, la media-piccola borghesia invece no.
Molti erano uffciali in guerra, si erano abituati al potere, vedevano la guerra come \textit{``igiene
del mondo''}. Tornavano dalla guerra in una vita tra difficoltà economiche e una vita anonima. Da
una parte vedevano gli industriali, ricchi, dall'altra i contadini, poveri. Loro erano in mezzo e
da questo ambiente scaturiranno i \textbf{movimenti di estrema destra}. Il piccolo borghese dal
suo canto non voleva uniformarsi al proletario. Questo è il fenomeno del \textbf{reducismo}, il
sentimento di delusione dei reduci dalla guerra che avendo combattuto per la patria si aspettavano
qualcosa di diverso dalla guerra.\\
La guerra aveva quasi ``normalizzato'' la violenza e questo lo si vede anche in politica dove gli
scontri non sono più solo verbali.

\subsection{Politica}
La prima guerra mondiale a avuto il pregio di aver contribuito all'emancipazione femminile. Infatti
quando il marito era al fronte erano le donne a gestire la casa. Molte donne inoltre furono assunte
nelle industrie durante la guerra e prendendo anche dei posti di comando. Molti paesi, dopo il
conflitto daranno diritto di voto alle donne.

\subsection{Repubblica di Weimar}
A novembre del 1918 la Germania non è più capace di sostenere la guerra, non ha più risorse da 
spendere e il malcontento cresce (si vengono a formare dei consigli di fabbrica simili ai soviet)
sostenuto dalla ``Lega di Spartaco'' (partito comunista che aveva a capo Rosa Luxemburg, una donna,
ebrea, colta). Sempre nel novembre del 1918 alcuni marinai si ammutinano e l'imperatore è costretto
a fuggire in Olanda. Ci sono quindi due forze al potere: il \textbf{PSD} con a capo Ebert e 
l'esercito. Il governo era in mano ad Ebert e come in Russia si definiva ``Consiglio dei commissari 
del popolo''. Voleva ricordare la Russia ma non voleva la rivoluzione, voleva un regime democratico 
ma con quel nome si portava vicino gli operai. L'esercito (guidato da Hindemburg e Ludendorf) accetta
il governo anche se erano contrari al PSD in quanto il governo era il male minore. A gennaio il
partito comunista tenta una rivoluzione ma fallisce, vengono allora creati i \textbf{Freicorps} in
comune accordo tra PSD ed esercito. Erano formati da ex arditi, erano nazionalisti opposti alla
rivoluzione. Rosa Lindemburg viene uccisa.

\subsubsection{Costituzione}
La costituzione che viene creata è \textbf{democratica ed egalitaria}. Formava una repubblica 
federale con 17 Länder che avevano molta autonomia. Era una repubblica parlamentare, il presidente 
era eletto a suffragio universale direttamente, rimaneva in carica 7 anni. Il presidente sceglieva
il cancelliere ed il governo che doveva avere la fiducia del parlamento. \textbf{Ebert è il primo
presidente}. Fino al 1933 il governo sarà conteso tra PSD e CDU, quando il nazismo prenderà il 
potere.\\
Come può uno stato con una costituzione del genere eleggere il nazismo? La repubblica non era forte,
la borghesia aveva nostalgia dell'impero. Inoltre l'Art.\ 48 dice che in caso di pericolo il 
presidente può sospendere le libertà e i diritti del popolo. Con il nazismo sarà sempre un caso di
pericolo. Molti cittadini vedevano la repubblica come un ripiego, meglio del socialismo ma non il
meglio che si potesse avere. Non tutti i partiti sostenevano questo governo (quelli di estrema destra
e sinistra volevano abbatterlo). Quelli di \textbf{estrema destra} erano antidemocratici, razzisti. 
La repubblica secondo loro nasce dalla sconfitta e a guidarla è chi ha portato a questa sconfitta
(è una pugnalata alle spalle da parte del PSD, della CDU e degli ebrei). \textbf{L'estrema sinistra}
era comunista, la repubblica è un governo borghese.

\subsubsection{Origine del partito Nazista}
Hitler nasce in Austria nel 1889 in una famiglia piccolo borhese. Dopo gli studi va a Vienna dove
voleva iscriversi all'Accademia delle Belle Arti. Tira avanti facendo lavoretti. Era molto 
interessato alla politica ma non vi partecipava attivamente. Impara le dinamiche della società di
massa. Nel 1913 si trasferisce a Monaco di Baviera. Avrebbe dovuto andare a combattere per l'esercito
Austro-Ungarico ma voleva combattere per l'esercito tedesco in Francia, diventa caporale e viene
smobilitato dall'esercito alla fine. Continuerà a collaborare come informatore: si doveva infiltrare
nelle manifestazioni politiche ed informare i sueriori. Così Hitler entra in contatto con il
\textbf{partito dei Lavoratori Tedeschi} di Drexler (estrema destra). Abbandona l'esercito e milita
per questo partito. Tra il 1919 e il 1920 diventa leader e nel 1920 il partito diventa 
\textbf{partito Nazional-socialista dei Lavoratori Tedeschi}. Era ancora un piccolo partito.\\
[\baselineskip]
Il programma prevedeva
\begin{description}
  \item[Riunione dei Tedeschi in un'unica Germania] Pangermanesimo, andava contro i cattolici
  \item[Abolizione ufficiale dei trattati internazionali]
  \item[Terra e suolo per le popolazioni in eccedenza] ad esempio gli slavi
  \item[Cittadino dello stato solo chi è di sangue tedesco] senza problemi di religione, 
    \textbf{no ebrei}, non per la religione o la cultura, ma per il sangue
  \item[Tutti i non-cittadini sono stranieri] e quindi ospiti
  \item[I diritti politici li hanno solo i cittadini]
  \item[Anti-parlamentarismo] 
  \item[Lo stato deve assicurare lavoro e assistenza ai cittadini] Se non si arriva a sostenere, si
    espellano gli ospiti
  \item[Espulsione degli immigrati dopo il 2 agosto 1914]
  \item[Dovere è produrre per il bene di tutti] Abolizione dei redditi di chi non fattura, confisca
    integrale dei profitti di guerra
  \item[Statalizzazione dei trust] Gli operai partecipano alla ridistribuzione degli utili
  \item[Conservata la classe media]
  \item[Comunizzati i grandi magazzini] Si affittano ai privati, si avvicina al socialismo e cerca il
    consenso presso gli operai, ciò che conta è la Germania, si avvicina al nazionalismo
  \item[La scuola insegna il nazionalismo]
  \item[Lotta legale contro le menzogne politiche] Controllo della stampa
  \item[I giornalisti sono solo tedeschi]
  \item[Libertà di religione finché non si danneggia la razza germanica] No ebrei
  \item[Forte potere centrale nel Reich] Quasi dittatoriale
\end{description}

\subsubsection{Le riparazioni}
Nel 1921 arrivarono le spese di riparazione di guerra. Ammontavano a circa 132 miliardi di marchi in
42 anni (era il 25\% del PIL tedesco). I governi del 1921 e del 1922 pagarono senza aumentare le 
tasse vendendo le riserve auree e aumentando carta moneta. Questo però portò ad una forte inflazione.
Nel 1923 non riuscendo a pagare la Francia e il Belgio, i loro eserciti occupano la Ruhr, una zona
ricca di trust, miniere e fabbriche per far lavorare le industrie per la Francia. La Germania non
potè opporsi. Il governo della CDU chiese ai cittadini di non lavorare per i francesi, il governo
avrebbe pagato loro lo stipendio. Nel 1929 ci fu un'altra enorme inflazione e il marco crolla.\\
I nazionalisti tentano l'insurrezione, i socialisti la rivoluzione. Anche i nazisti ci provano
a novembre a Putch e a Monaco. Ludendorf e Hitler volevano prendere il potere in Baviera e poi a 
Berlino, sul modello di Mussolini. Ludendorf fu libero, Hitler invece venne imprigionato e processato
nel 1924. Hitler sfruttò il processo per farsi conoscere. Passò un anno in carcere, dove scrive il
``Mein Kampf'' che contiene il programma del 1920 e il rifiuto del parlamentarismo con un'aggiunta 
riguardo allo spazio vitale.\\ [\baselineskip]
Nella seconda metà del 1923 il governo fu di Gro{\ss}e Koalition (CDU+PSD). \textbf{Streussman} guida
il governo, era un liberale. La Germania doveva prendere accordi internazionali con la mediazione
degli USA, vengono così allargate le rate dei danni di guerra. Gli Stati Uniti temevano la
sovraproduzione quindi \textbf{le banche americane si impegnarono ad investire in Germania}. Avevano 
interessi che l'economia tedesca si riprendesse per avere un mercato. Venne creato un nuovo marco
garantito dal suolo tedesco, non da riserve auree. \textbf{Dal 1924 al 1929 l'economia si riprende}.
Pagano le sanzioni e in generale l'età di Streussman è un'epoca di stabilità e sviluppo economico.
Infatti voleva inserire la Germania in un piano di parità con gli altri stati così che firma gli
\textbf{Accordi di Locarno} riaffermando i confini (la Germania perde ufficialmente l'Alsazia e la 
Lorena).\\ [\baselineskip]
Dopo il processo il partito Nazista e Hitler sono molto conosciuti. Fino al 1929 il suo partito
prenderà sempre attorno al 2\%, principalmente al sud. Aveva formato le Squadre di Assalto di 
partito, combattenti contro gli avversari politici. Le due più illustri vittime furono Erzberger 
(CDU, aveva firmato l'armistizio) e Ruthermau (liberale, organizzò l'economia nella guerra, diventato
ministro degli esteri voleva pagari le riparazioni di guerra).

\subsection{Italia}
Anche se l'Italia ufficialmente aveva vinto la guerra, rimaneva devastata. Con un debito pubblico
enorme, inflazione e malcontento.\\
Fino al 1922 i liberali rimasero al potere ma si erano indeboliti sempre di più, non erano stati
capaci di gestire la crisi.\\
Il PSI aumenta notevolmente il numero di iscritti, però era diviso in due correnti
\begin{description}
  \item[Massimalisti] Rivoluzionari. \textbf{Serrati} era il capo e anche dirigente di partito. Si
    aspettavano come imminente il crollo del capitalismo ma volevano che la rivoluzione arrivasse da
    sè, non fare come Lenin. Non aderirono alla Terza Internazionale. Il PSI secondo loro non poteva
    fare accordi con altri partiti in quanto borghesi
  \item[Riformisti] \textbf{Turati} era a capo. La rivoluzione non era così vicina, erano disposti a
    fare alleanze se necessario
  \item[Ordine nuovo] Una rivista torinese. \textbf{Gramsci} era l'esponente e direttore, 
    \textbf{Togliatti} collaborava. L'URSS era il modello da seguire, i consigli di fabbrica, 
    democrazia dal basso. Erano un piccolo gruppo di giovani che nel 1921 fonderanno il PCI
\end{description}
Nel \textbf{1919} nasce il \textbf{Partito popolare Italiano}, il primo partito cattolico. 
\textbf{Sturzo} è a capo. Benedetto \rom{15} era meno conservatore di Pio \rom{10} e la Chiesa era
preoccupata che i socialisti potessero prendere il sopravvento. Sturzo inoltre voleva avere una
propria rappresentnza politica. \textbf{Diventa subito partito di massa}. Sturzo presenta il PPI
come \textit{a-confessionale}, ovvero che non era necesasrio essere cattolici per aderire. Il 
programma conteneva elementi chiave della religione cristiana:
\begin{itemize}
  \item No divorzio
  \item Libertà d'azione per la Chiesa
  \item Libertà per scuole private
  \item Suffragio universale anche femminile
  \item Legge elettorale proporzionale
  \item Introdurre le regioni (decentramento del potere e maggiori autonomie locali)
  \item Riforma agraria (terre incolte espropriate con indennizzo e date alle singole famiglie)
\end{itemize}
Si presentava come non conservatore. È un partito interclassista che si interessava pricipalmente 
alle classi più deboli. La nascita del PPI indebolisce ancora di più i liberali.

\subsubsection{Trattati ed elezioni del 1919}
Nel 1919 Orlando era primo ministro e Sonnino era ministro degli esteri. Trattano in Francia e
\textbf{chiedono che venga rispettato integralmente il Patto di Londa} ma anche che, per il principio
di autodeterminazione dei popoli, \textbf{Fiume venga annessa}. Si contraddicono in questo modo.
Non ottengono né Fiume né la Dalmazia che finisce alla Iugoslavia. Tornano a Roma come gesto di 
sdegno con manifestazioni nazionaliste che insorgono. I trattati vanno avanti senza l'Italia.
L'Italia firma quello che gli altri avevano scelto (no Fiume, no Dalmazia e no colonie). Fiume era
una città libera e \textbf{Orlando si dimette} nel giugno del 1919. \textbf{Nitti} prende il suo
posto (liberale democratico).\\
A novembre si tengono nuove elezioni con una legge proporzionale:
\begin{description}
  \item[Liberali alla maggioranza relativa] circa 200 deputati
  \item[PSI] circa 150 deputati
  \item[PPI] circa 100 deputati
  \item[FIC] solo qualche voto a Milano
\end{description}
Liberali e PPI governeranno ma Sturzo non voleva fare la stampella e quindi il governo sarà molto 
instabile.

\subsubsection{Impresa di Fiume}
D'Annunzio ed alcuni nazionalisti partono da Ronchi e raggiungono Fiume che era sotto la società 
delle nazioni. Fiume cade nelle mani di D'Annunzio. Questa è una \textbf{violazione degli accordi
internazionali}. Molti dei seguaci erano militari che ora sono quasi disertori. Questo avvenimento
scredita ancora di più il governo italiano. \textbf{Badoglio} era il generale in Friuli e avrebbe
dovuto ordinare ai soldati di cacciare D'Annunzio ma non era sicuro l'avrebbero ascoltato. L'Italia
non è capace di governare.\\
L'impresa di Fiume suscitava molta simpatia tra gli aderenti ai FIC ma Mussolini temeva che 
D'Annunzio prendesse troppo successo. A questo punto \textbf{Mussolini capisce che la direzione da
seguire è quella nazionalista}. A luglio del 1920 a Trieste c'è una manifestazione a sostegno di
D'Annunzio (a Trieste c'erano anche i primi non-milanesi iscritti ai FCI). Un corteo parte da Piazza
Unità fino all'Hotel \textbf{Balkan} (simbolo della popolazione slava). \textbf{Viene incendiato} e
le forze dell'ordine collaborano. Questo è il primo esempio di \textbf{squadrismo}.\\
Il governo di Nitti era sia colpito dall'impresa di Fiume che dai sindacati. Nitti si dimette nel
giugno del 1920 e \textbf{Giolitti sale al potere}. Mette fine all'impresa di Fiume facendo accordi
con l'Inghilterra (\textbf{accordi di Rapallo}). \textbf{Fiume è una città libera}. Giolitti intimò
con poche cannonate di andarsene a D'Annunzio.

\subsubsection{L'occupazione delle fabbriche}
Tra agosto e settembre del 1920 la FIOM (facente parte della CGIL) e Confindustria discuterono sugli
orari di lavoro. La FIOM indisse quindi uno \textbf{sciopero bianco}, ovvero lavorare in modo da
rallentare la produzione. La Confindustria di rimando indisse la \textbf{serrata} ovvero la chiusura
delle fabbriche. A Milano, la Romeo era occpuata dagli operai. Successivamente altre fabbriche 
seguirono fino ad arrivare a 500 mila operai che cercano di mandare avanti il lavoro, dimostrando
forza e autonomia. \textbf{I consigli di fabbrica gestiscono gli operai}.\\
I socialisti \textbf{riformisti} non credevano ci fossero le condizioni per una rivoluzione, 
bisognava continuare l'occpuazione solo per trattare. I \textbf{massimalisti} non credevano ci
fossero le condizione per la rivoluzione e non era questo il modo di cominciarla. Quelli 
dell'\textbf{Ordine nuovo} volevano invece la rivoluzione in quanto il PSI deve prendere il potere
politico dalle fabbriche.\\
Confindustria voleva che il governo prendesse provvedimenti, anche con la forza.\\
Giolitti non voleva usare l'esercito, sarebbe stato l'inizio della rivoluzione, voleva far passare il
tempo favorendo un accordo sindacati-Confindustria. Dopo un mese FIOM e Confindustria firmano un 
accordo molto favorevole per gli operai: 8 ore di lavoro per tutti, aumenti salariali e diritti di 
ferie retribuite.\\
Questo evento rese ancora più visibili le correnti nel PSI tanto che a gennaio del 1921 c'è la 
scissione. A Livorno c'era il congresso che discuteva se aderire o meno alla Terza Internazionale
che implicava avere l'URSS come modello, cambiare nome di partito ed espellere i riformisti. Né
i massimalisti né i riformisti volevano aderire, solo Gramsci. \textbf{A gennaio del 1921} nasce
il \textbf{PCI} che aderisce alla Terza Internazionale.\\
I liberali vedevano l'occpuazione delle fabbriche come debolezza e la borghesia era impaurita dalla
rivoluzione. Alcuni membri del movimento liberale si avvicinano al fascismo. Infatti Mussolini sarà
bravo ad approfittarne. Promosse lo squadrismo anti socialista: nel 1920 ci sono le elezioni comunali
e il PSI vince in molte città. Gli squadristi manifestano contro di essi. Tra il 1921 e il 1922 ci 
sono oltre 1000 morti a casua dei fenomeni di squadrismo. Quindi il fascismo si diffonde e si
allarga fino ad esseere finnanziato dai propietari terrieri e gli industriali. Questo è il
\textbf{fascismo organico} ovvero dalel città si diffonde. Alla fine del 1921 avrà più iscritti del
Partito Socialista.



%!TEX ROOT=storia.tex
\section{Fascismo}
\subsection{Nascita del movimento fascista}
Benito Mussolini nasce da una famiglia rivoluzionaria. A 21 anni è socialista rivoluzionario e scappa
in Svizzera per evitare la leva militare. Torna in Italia e milita con il PSI attivamente. Nel 1912 
c'è il congresso in cui Mussolini propone di espellere tutti i leader riformisti come 
\textbf{Ivano Bonomi} in quanto non avevano criticato abbastanza duramente l'operazione in Libia e 
avevano dato fiducia la re. Bonomi e gli altri vengono espulsi e Mussolini diventa direttore 
dell'\textit{Avanti}.\\
All'inizio della guerra si schiera contro ma fra settembre ed ottobre scrive che se rimangono
neutralisti verranno isolati e arriva a scrivere a favore dell'intervento. La direzione del partito
licenzia Mussolini dalla direzione del giornale e le espelle. Mussolini allora fonda il 
\textbf{Popolo d'Italia}, il suo giornale finanziato dai Perrone e dai servizi segreti francesi 
(avevano interesse ci fosse un giornale interventista). È un \textbf{giornale socialista 
interventista}.\\
Mussolini va in guerra, combatte e viene ferito, così poi dimesso dall'esercito. Con il passare del
tempo, sopratutto dopo Caporetto il socialismo di Mussolini diventa debole e le sue posizioni si 
fanno sempre più \textbf{nazionaliste}. Finita la guerra, mancano i finanziatori così dà vita al
suo movimento. Il 23 marzo 1919 nascono i \textbf{Fasci Italiani di Combattimento} a Milano, in 
piazza San Sepolcro. Il programma prevedeva
\begin{description}
  \item[Movimento nazionalista rivoluzionario] Anti-dogmatico, anti-pregiudiziale, guerra al di sopra
    di tutto e tutti
  \item[Le idee come mezzo per l'azione politica]
  \item[Suffragio universale]
  \item[Legge elettorale proporzionale]
  \item[Maggiore età a 18 anni]
  \item[Abolizione del Senato]
  \item[Assemblea nazionale] per fare la costituzione e scegliere la forma di governo
  \item[8 ore di lavoro] Socialista
  \item[Minimo di paga]
  \item[I rappresentati dei lavoratori partecipavano ad organizzare le fabbriche]
  \item[I rappresentati dei lavoratori gestiscono industri e servizi]
  \item[Nazionalizzazione delle fabbriche di armi]
  \item[Nazionalismo] in politica estera
  \item[Espropriazione della ricchezza] Imposte progressive
  \item[Sequestro dei beni delle congregazioni religiose]
\end{description}
Si diffonde princpalmente tra gli ex combattenti ed ex arditi.\\ [\baselineskip]
Il 15 aprile c'è uno sciopero a cui i FIC si oppongono e assaltano la direzione dell'Avanti, 
devastandola.

\subsection{Mussolini acquista potere: la nascita del partito}
Le autorità facevano finta di non vedere i fenomeni di squadrismo, erano complici in quanto non 
avevano simpatie per i socialisti. Giolitti ordinava venisse rispettata la legge ma spesso i
prefetti lasciavano correre. \textbf{Giolitti era in difficoltà}.\\
Nel 1921 il \textbf{PPI toglie l'appoggio} al governo in quanto quest'ultimo aveva firmato una legge
a favore della nominalità dei titoli azionari (poter sapere chi compra e vende implica la possibilità
di tassare). Il \textbf{5 novembre 1921} ci sono nuove elezioni a suffragio universale con legge
proporzionale. Dato che non c'è un partito liberale di massa, \textbf{Giolitti propone un'alleza a
Mussolini} (dei Blocchi Nazionali, spera che i Fascisti al potere torneranno nella legalità). Con
questo accordo, i liberali hanno la maggioranza relativa, il PSI perde voti e il PPI ne guadagna.
\textbf{I FIC hanno 35 deputati}. Mussolini dichiara subito che \textbf{non avrebbe sostenuto 
Giolitti}. Mussolini tra maggio 1921 ottobre 1922 continuerà ad usare le squadre d'assalto. Giolitti
non ha più la maggioranza, \textbf{nuovo governo a Bonomi} con l'obiettivo di riportare l'ordine.
Favorsice il \textbf{Patto di Pacificazione} ad agosto del 1921. È un accordo sindacati 
socialisti-fascisti per mettere al bando le violenze (questo provoca malcontento dei Ras (capi 
locali) fascisti, tra cui Farinacci). Mussolini accetta perché altrimenti avrebbe dimostrato di 
essere loro la causa dei disordini inoltre a Sarzana (Liguria) i Carabinieri si erano opposti
ai fascisti (alcuni Ras credono che Mussolini si stia ``imborghesendo'').\\ [\baselineskip]
Nel novembre del 1921 c'è il \textbf{Congresso fascista} che trasforma il movimento in 
\textbf{Partito Nazionale Fascista} e rinuncia al patto di Pacificazione (dando la colpa al PSI).
Questo cambiamento porta ad una centralizzazione del potere nelle mani di Mussolini e meno potere
ai Ras.

\subsection{La marcia su Roma e le divisioni socialiste}
A febbraio del 1922, \textbf{Bonomi si dimette e sale Facta} che ad agosto si dimette ma gli viene
riconferito l'incarico.\\
Nel 1922 lo squadrismo si fa più duro andando ad attaccare anche le istituzioni di città imporanti.
A Milano costringono alle dimissioni il sindaco, danno potere al prefetto. Ad inzio agosto, la CGIL
indice uno \textbf{sciopero legislativo} per far rispettare la legge ma fallisce.\\ [\baselineskip]
Mussolini si prepara alla presa di potere:
\begin{description}
  \item[Cambia posizioni sulla monarchia] Originariamente era contrario ma per non avere contro
    l'esercito deve aprirsi alla monarchia
  \item[Toglie ogni riferimento socialista] Sostiene il capitalismo e fa incontro con Confindustria
  \item[Elimina l'anticlericalismo] Favorito da Pio \rom{11} (conservatore)
\end{description}
Ad inizio ottobre il PSI si scinde un'altra volta
\begin{description}
  \item[Riformisti] Arginare il PNF con accordi con i liberali o il PPI
  \item[Massimalisti] Contrari ad accordi
\end{description}
I riformisti escono dal partito e fondano il \textbf{Partito Socialista Unitario} con Matteotti 
segretario.\\[\baselineskip]
Ormai si parlava già di una marcia su Roma e Mussolini in un discorso a Napoli disse ``Se non ci 
daranno il potere, ce lo prenderemo calando su Roma'' (24 ottobre). Mussolini organizza a Milano la
marcia. \textbf{Il 27 ottobre 1922} inizia la Marcia su Roma. Alcune squadre prendono il potere
in alcune città attaccando luoghi di comunicazione, caserme, \ldots senza quasi resistenza. Dove c'è
non arrivano a prendere il potere. Erano circa 40 mila squadristi.\\
Facta vuole dichiarare lo stato d'assedio ed informa il re. Inizialmente era contrario ma poi
cambia idea: \textbf{Vittorio Emanuele \rom{3} quasi legalizzava il colpo di stato}. Voleva dare 
qualche ministero a Mussolini e far entrare nel governo i fascisti (sempre con l'idea che una volta
al potere, i fascisti si sarebbero costituzionalizzati). \textbf{Il 30 ottobre Mussolini arriva a
Roma ed entra al Quirinale con la camicia nera}.\\
Il primo governo è di coalizione. 5 ministeri sono dati ai Fascisti, \textbf{Mussolini} ha il 
\textbf{ministero dell'interno ed è presidente del consiglio}. C'erano anche liberali, popolari e
nazionalisti al governo. Il ministero della guerra era a Diaz (rassicura il re e l'esercito). A
novembre tiene il primo discorso alle camere, il \textbf{Discorso del Bivacco}. Non chiede la fiducia
in quanto il governo non è nato in parlamento ma nelle piazze. Voleva essere diverso dai liberali.
\textit{``Avrei potuto fare di quest'aula sorda e grigia un bivacco di manipoli''}, ovvero avrebbe
potuto fare un governo di soli fascisti, portando gli squadristi al Quirinale. \textit{``Avrei potuto
sprangare il parlamento. Avrei potuto ma non l'ho fatto, non ancora.''}. Soltanto PSU, PCI e PSI non
votano la fiducia al governo.\\ [\baselineskip]
A fine del 1922 viene istituito il \textbf{Gran consiglio del Fascismo} che non era previsto nello
Statuto. Il suo compito era quello di dare la direzione politica al governo, fatto solo di fascisti,
non era elettivo. Agli inizi del 1923 viene istituita la \textbf{Milizia volontaria per la Sicurezza
Nazionale}. Era la legalizzazione delle squadre. Dato che l'esercito temeva di essere soppiantato,
Mussolini fa sì che i miliziani giurino anche davanti al re.

\subsection{Riforme, elezioni e delitto Matteotti}
Nel 1923 viene fatta la riforma scolastica. \textbf{Gentile} era il ministro, oltre che filosofo
idealista. Vengono riorganizzati gli studi: i licei erano il vertice, d'élite che dovevano
formare la classe dirigente. Solo chi andava al liceo poteva andare all'università. Viene introdotta
la religione cattolica anche nelle elementari (su richiesta del PPI). Viene introdotto l'esame di
stato (richiesta del PPI, per equparare il titolo di studi di scuole pubbliche e private).\\
[\baselineskip]
Mussolini voleva assorbire le altre forze politiche all'interno del fascismo. Viene anche abolita la
nominalità dei titoli azionari (a favore del PPI). Così molti popolari aderiscono al fascismo, il
partito nazionalista si fonde per primo con il fascismo (Federzoni e Rocco).\\ [\baselineskip]
Luigi \textbf{Acerbo} nel 1923 scrive la legge elettorale. Prevedeva un forte premio di maggioranza
(con la maggioranza relativa si ha il 66\% dei deputati). Questo costringeva gli esponenti degli
altri partiti ad allearsi con il fascismo. Si formano così le \textbf{Liste Nazionali}.\\
Ad aprile del 1924 ci sono le prime elezioni, molto violente a causa della Milizia. I fascisti 
vincono con il 65\%.\\
In giugno si riunisce il parlamento e \textbf{Matteotti} (PSU) \textbf{denuncia apertamente le 
violenze}. Il parlamento era eletto illegalmente e chiedeva che il re sciogliesse il parlamento. Il
giorno successivo il Popolo d'Italia (giornale di Mussolini) minaccia Matteotti. Viene rapito qualche
giorno dopo e ad agosto viene ritrovato il cadavere. Questo fece molto scalpore e molti cambiarono
idea. \textbf{Mussolini è in difficoltà}. Cede il ministero dell'interno a Federzoni. Tiene a 
bada la milizia e prende tempo confidando sull'appoggio del re. Lascia anche che le indagini facciano
il loro corso e trovino i responsabili. Mussolini è responsabile ma non direttamente, verranno 
condannati ma per poco tempo.\\
C'erano forti opposizioni politiche: molti anti-fascisti decidono di non partecipare ai lavori del
parlamento. Si ritrano nella sala dell'\textbf{Aventino}. Essi sono PSU, PSI e liberali guidati da
Amendola. Volevano chiamare in causa il re per destituire Mussolini e fare nuove elezioni.\\
Giolitti e i suoi non partecipano. Dopo l'omicidio Giolitti diventa decisamente anti-fascista. Non
partecipa perché crede che la politica si faccia in parlamento, non fuori. Neanche il PCI partecipa
credendo che l'Aventino fosse inutile e che era da chiamare in causa il popolo.\\
\textbf{Il re asseconda Mussolini} in quanto prende in considerazione solo ciò che accade in 
parlamento. 

\subsection{Dittatura fascista}
Il \textbf{3 gennaio 1925} Mussolini fa il discorso in parlamento che dà inizio alla 
\textbf{dittatura fascista}. \textit{``Se il fascismo è stata un'associazione a delinquere, io ne
sono il capo''}, così si prende la responsabilità storica e politica delle azioni squadriste. È la
fine dello stato liberale. Tra il 1925 e il 2928 c'è la costruzione dello stato totalitario fascista.
Vengono firmate le \textbf{Leggi fascistissime}:
\begin{description}
  \item[Le proposte di legge dovevano essere approvate dal capo del governo prima che arrivino al
    parlamento] Non c'è divisione dei poteri, non è presidente del consiglio ma capo del governo
  \item[Tutti i partiti fuori legge tranne il PNF] Non c'è libertà politica
  \item[Tutte le organizzazioni sindacali fuori legge] Non c'è libertà di associazione. Confindustria
    fa l'accordo di Palazzo Vidoni dove gli industriali fanno accordi con i sindacati fascisti
  \item[Solo i giornali fascisti sono legali] Non c'è libertà di stampa. Mussolini inviava le
    \textit{veline} (ordini, direttive su cosa scrivere)
  \item[Viene istituito l'OVRA] Organizzazione per la Vigilanza e Repressione dell'Antifascismo
  \item[Istituito il Tribunale Speciale per la Difesa dello Stato] Tribunale politico che infliggeva
    anche la pena di morte
  \item[Definiti i poteri del Gran Consiglio] Il Gran Consiglio sceglie dei nomi e il re sceglie tra
    quelli. Sceglie inoltre i candidati dei deputati (400 nomi, gli elettori dicevano o sì o no, voto
    non segreto)
\end{description}
L'11 febbraio 1929 si firmano i \textbf{Patti Lateranensi} (Mussolini e Cardinale Gasparri 
(segretario di stato vaticano)). Sono divisi in 3:
\begin{itemize}
  \item Trattato internazionale: si riconoscevano recpirocamente
  \item Convenzione finanziaria: Mussolini pagava alla Chiesa quanto avrebbe pagato dalle Guarentige
    per i territori sottratti alla Chiesa
  \item Concordato (=accordo tra Chiesa ed uno stato):
    \begin{description}
      \item[Religione cattolica come religione di stato] 
      \item[Privilegi per il clero] No leva militare, \ldots
      \item[I sacerdoti apostati non potevano ricoprire cariche pubbliche] Gli ex sacerdoti, era
        retroattiva
      \item[Il matrimonio in Chiesa ha valore anche civile] Il sacerdote diventa un funzionario dello
        stato
      \item[La religione cattolica insegnata per legge fino ai licei] Diventa il fondamento e 
        coronamento dell'insegnamento pubblico
      \item[``Congrua'' ai sacerdoti] Una somma di denaro
      \item[Libertà d'azione religiosa, culturale ed educativa] purché non si occupasse di politica
    \end{description}
\end{itemize}
Mussolini sapeva di aver concesso molto alla Chiesa ma gli conveniva avere il sostegno dei cattolici.
Anche il papa lo definisce ``Uomo della provvidenza''.\\
Vengono fatte nuove elezioni con la legge elettorale, sono un successo strepitoso.

\subsection{Politica economica degli anni Trenta}
Fino al 1925 ci fu una politica economica liberista, le spese dello stato erano state ridotte, 
liberalizzati alcuni settori (assicurazioni, telefonia). L'economia era in sviluppo. Dal 1925 
Mussolini cambia strada. Passa ad una fase di \textbf{intervento dello stato} nella vita economica
con obiettivi politici.\\
Dal 1922 c'era \textbf{inflazione}, la lira perdeva valore rispetto alla sterlina. L'inflazione 
colpiva la medio-piccola borghesia dove il consenso era più forte. C'era in gioco la nazione in 
quanto una moneta debole indicava una nazione debole. Prima del fascismo il rapporto sterlina-lira
era di 1:90. Adesso era di 1:150 circa. L'\textbf{obiettivo era deflazione} e abbassamento dei prezzi
così si alzano i tassi di interesse e quindi si riduce la circolazione monetaria. Quest'obiettivo era
denominato \textbf{quota 90}. Gli industriali non erano d'accordo, 90 era troppo svantaggioso per chi
vendeva all'estero. In alcuni anni quest'obiettivo si raggiunge con salari diminuiti (a svantaggio 
dei lavoratori) e prezzi più bassi.\\
\textbf{Dal 1927 protezionismo} per difendere l'agricoltura. La \textbf{battaglia del grano} voleva
aumentare la produzione per rendere l'Italia autosufficiente. Diede dei risultati ma non sufficienti.
Nello stesso anno la  \textbf{Carta del lavoro} in cui vengono esposti i principi dell'economia e
della società in cui si rifiuta sia il liberalismo (era individuale, non valorizza la nazione) sia
il socialismo (sostiene la lotta di classe che divide). \textbf{Il lavoro è un dovere} non più solo
un diritto. L'economia è in funzione della nazione, tramite le \textbf{corporazioni} che erano
organizzazioni in cui i rappresentati dei lavoratori e dei datori di lavoro si incontrano, per ogni
settore. Una corporazione organizza tutto, riprende la Rerum Novarum ma con finalità nazionalistiche,
non sociali. Non fecero molto, diedero qualche posto di lavoro.\\
Nel \textbf{1939} il Parlamento viene rinominato a ``Camera dei fasci e delle corporazioni''. \\
La crisi del 1929 si fa sentire un po' meno forte perché l'economia aveva già di suo cominciato a 
rallentare. Nel 1932 c'è il picco con circa 1,5 milioni di disoccupati.

\subsection{Politica estera degli anni Trenta}
All'inizio Mussolini non adotta una politica estera aggressiva, aveva l'obiettivo di far accettare il
regime fascista agli altri paesi. Partecipa agli accordi di Locarno e ad altri accordi che miravano
al disarmo delle potenze appartenenti alla Società delle Nazioni. Una guerra viene condotta 
nonostante tutto, in Libia che fu persa durante la guerra.\\
Nel 1933 il nazismo sale al potere e Hitler fa uscire la Germania dalla Società delle Nazioni, nel
1934 comincia a voler conquistare l'Austria (con la scusa di un partito filo-nazista al suo interno)
e \textbf{Mussolini si oppone} dicendo che avrebbe garantito l'indipendenza austriaca. Hitler cede.\\
Nell'aprile del 1935, c'è la \textbf{conferenza di Stresa} in cui Italia, Francia e Inghilterra si
promettono di mantenere l'ordine dopo la prima guerra mondiale, è un accordo anti-tedesco.

\subsubsection{Guerra in Etiopia}
Dopo questo patto, Mussolini vuole conquistare l'\textbf{Etiopia}. Era convinto che Francia e 
Inghilterra si sarebbero opposte ma avevano bisogno dell'Italia per sconfiggere e fronteggiare i 
tedeschi. Inoltre si facevano sentire gli effetti della crisi del 1929 e una guerra si diceva avrebbe
portato posti di lavoro e ricchezza. Specialmente alcuni industriali ci avrebbero guadagnato. Infine
il fascismo aveva uno stampo nazionalista e dal 1922 non aveva ancora attuato una conquista, voleva
dimostrare la propria forza. \textbf{La guerra si combatte tra ottobre 1935 e maggio 1936} con la 
vittoria dell'Italia, Vittorio Emanuele \rom{3} diventa imperatore d'Etiopia. La Chiesa aveva 
appoggiato la guerra in quanto vista come di civiltà.\\ [\baselineskip]
Il regime lo vede come un assoluto successo, aveva dimostrato la sua forza e infatti aumenta il
consenso (arriva ad istituire ``l'oro della patria'' dove i cittadini regalavano allo stato i propri
possedimenti aurei o preziosi). Dal punto di vista economico la guerra è stata \textbf{un disastro}
in quanto era costata tantissimo e non aveva portato alcuna ricchezza o posto di lavoro. Infine
l'Etiopia era uno stato membro della Società delle Nazioni e quindi attaccarla viene vista come un
atto inaccettabile. Vengono imposte lievi sanzioni economiche all'Italia (non poteva commerciare con
gli stati membri qualsiasi materiale bellico) e infine viene espulsa. Questo viene usato per la
propaganda fascista in quanto Mussolini presenta l'Italia come \textbf{circondata} dagli altri stati
che impediscono all'Italia di essere una potenza plutocratica (basata sulla ricchezza).

\subsection{Rapporti con Hitler}
I rapporti con la Germania si infittiscono e nel 1936, durante la guerra civile spagnola sia Italia 
che Germania aiutano in modo ufficioso Franco (filo-fascista). \textbf{Ad ottobre 1936 si forma 
l'asse Roma-Berlino}, è un'alleanza politica. Dal 1938 Hitler comincia il programma di annessione
dell'Austria, avrebbe dovuto chiedere a Mussolini ma di fatto faceva e poi chiedeva. Questo genera
malcontento nelle gerarchie fasciste.\\
Nel 1939 Ciano era il ministro degli esteri (cognato di Mussolini) e riteneva che l'asse fosse 
pericoloso. \textbf{Nel maggio del 1939 si firma il Patto d'Acciaio} che è un'alleanza militare tra
Hitler e Mussolini. Se una delle due nazioni entra in guerra (qualunque tipo) l'altra entrava ad
aiutarla. L'Italia comunica a Hitler che non era nelle condizioni di sostenere un'altra guerra e
Hitler assicura che non avrebbe cominciato la guerra prima di 3 anni. In realtà aveva già deciso il
1 settembre come data d'inizio. Il re non voleva la guerra ma non si oppone.\\
Quando Hitler attacca la Polonia inizia la guerra, Mussolini dichiara l'Italia \textbf{non 
belligerante} fino al 1940. Il 10 giugno 1940 entra in guerra con Hitler perché dice ``Ho bisogno di
qualche migliaio di morti da mettere sul tavolo delle trattative''.

%!TEX ROOT=storia.tex

\section{La grande crisi e il New Deal}
Ad oltre 50 anni dalla grande depressione di fine 1800, negli anni Trenta si verificò un'altra crisi
di portata mondiale.

\subsection{Il giovedì nero, le cause della crisi}
Il \textbf{24 ottobre 1929} esplode la crisi economica con il crollo della Borsa di New York. Wall
Street era diventata la banca principale dove venivano effettuati i principali movimenti speculativi.
Lo scambio di titoli azionari era fatto senza controlli e questa libertà aveva portato a praticare
attività molto rischiose come \textbf{l'acquisto delle azioni a credito}. Questo ha provocato 
l'aumento dell'\textit{economia di carta} che è sempre più slegata da quella reale. Il sistema
funzionava nel seguente modo
\begin{itemize}
  \item Il piccolo risparmiatore chiedeva un prestito al mediatore di Borsa, per farlo depositava
    un margine (titoli in garanzia) pari al 30\%-50\% del prestito
  \item Il mediatore contraeva prestiti a breve termine da altri istituti
  \item Il risparmiatore contava di vendere le azioni ad un prezzo tale da ripagare i debiti
\end{itemize}
Questo funzionò fino al 1929. Per arginare il problema, la banca centrale americana aveva aumentato
il tasso d'interesse nei rapporti con altre banche con il fine di scoraggiare operazioni di credito
con altre banche. Ma già nel marzo 1929 la speculazione riprese e soltanto ad agosto alzò il tasso
al 6\%, una misura troppo tardiva.\\
Questa crisi finanziaria ebbe ripercussioni su tutta la società dato che erano nate aziende in ogni
settore sull'onda della speculazione. Ora queste aziende non riuscivano più a sostenersi e chiudevano
licenziando migliai di lavoratori. All'inizio del 1931 i disoccupati erano 8 milioni, dopo un anno 
13.\\
La \textbf{causa strutturale} del grande crollo fu l'eccesso di capacità produttiva. La diffusione
del taylorismo fu un altro forte motivo che produsse il crollo in quanto la sua espansione prevedeva
una riorganizzazione dei sistemi di produzione. La suddivisione del lavoro aumentava la quantità
di prodotti ma mancavano i mercati a sostenerli. Infatti, anche con l'aumento dell'export, i mercati
non riuscivano ad assorbire tutto il prodotto. Verso la metà degli anni Venti però l'Europa tornava
ad essere una grande produttrice. Quindi da una sola grande nazione, l'America, che gestiva tutto
il mercato mondiale, si è giunti ad un sistema policentrico che comportò un enorme produzione di
eccedenze.\\
Dato che la produzione di scala aveva aumentato il numero di prodotti, doveva aumentare il potere
d'acquisto dei cittadini e per fare ciò si diffusero le \textbf{agevolazioni creditizie}. Ma 
nonostante l'aumento generale dei redditi, non si riusciva ad assorbire quella quantità di merci
prodotte. Questo provocò la riduzione della produzione e il conseguente licenziamento.\\
Questa crisi, dagli Stati Uniti, si diffuse in tutto il mondo. Principalmente per gli stretti
rapporti che intercorrevano fra i vari stati, anche in Occidente si diffusero sovraproduzione e
stagnamento.

\subsection{Gli effetti della crisi}
La crisi ebbe l'effetto di far \textbf{sostenere i prezzi} e \textbf{abbassae la produzione}.
Questo comportò una forte disoccupazione. Ciò fu possibile perché ormai i prezzi non erano regolati
da rapporti di domanda e offerta, ma erano imposti dai grandi trust che, pur di tutelare i profitti,
decisero di ridurre la produzione e sotenere i prezzi. Gli stati risposero con il 
\textbf{protezionismo} che tutelava il mercato interno a discapito di quello estero. Gli Stati Uniti
con \textbf{Hoover} furono i primi ad adottare questa politica. Gli altri stati si adeguarono e 
quindi si ottennero tanti piccoli mercati nazionali.\\
Gran Bretagna e Germania hanno sofferto di più, quest'ultima fu resa di fatto dipendente dagli 
investitori statuinitensi. Così che quando furono ritirati a causa della crisi, la Germania sprofondò
ancora di più.\\
Oltre ai soliti dazi doganali, gli stati adottarono anche misure più razionali come accordi 
bilaterali. Questa crisi comportò anche una scossa del fondamento dell'economia monetaria del tempo:
\textbf{il valore della moneta non è più dipendente dal valore aureo} (evento chiave è il governo 
inglese nel 1931 decise di rendere inconvertibile la sterlina). Questo significava che dopo 
l'Inghilterra, gli altri paesi aumentarono le svalutazioni per non perdere la competitività in quanto
non si poteva più riscuotere il valore in oro della sterlina. Quest'economia produsse una 
\textbf{politica di potenza} tra i vari stati.

\subsection{Roosvelt e il New Deal}
Roosvelt venne eletto presidente nel 1932. La sua campagna si fondava su due principi fondamentali
\begin{description}
  \item[Rilancio dell'economia] sostenendo il mercato, rimuovendo la miseria ed aiutando la società
  \item[Mettere sotto controllo il sistema bancario] per impedire le grandi speculazioni di borsa
\end{description}
\textbf{Lo stato quindi interveniva nella vita economica} e ciò era qualcosa di nuovo, mai accaduto
prima. L'intervento economico americano era profondamente democratico in quanto si fondava sulla
redistribuzione del reddito.\\
Nel 1933, con \textbf{Emergence Banking Act} la Federal Reserve viene rafforzata e sulle banche,
holding e la Borsa vengono messe sotto più rigidi controlli. Venne introdotta una garanzia
statale sui piccoli depositi. Favorì la ripartizione delle quote di mercato limitando la
concorrenza sleale. Creò la \textbf{Work Progress Administration} per aprire cantieri pubblici al
fine di riassorbire la disoccupazione.\\
Le grandi \textit{corporations} come la General Motors erano contrarie e la stessa corte suprema 
dichiarò incostituzionale la manovra di Roosvelt. Approvò inoltre la \textbf{legge Wagner} nel 1935
che riconobbe pienamente i diritti sindacali dei lavoratori.\\
Il \textbf{Social Security Act} del 1935 fondò le basi dello stato sociale che per la prima volta
proteggeva il lavoratore con assicurazioni e sussidi. Infine si mise in atto una tassazione 
progressiva.

\subsection{Keynesismo}
John Maynard Keynes fu uno dei più grandi economisti del 1900. Modificò il liberismo tradizionale.
Nel suo ``Teoria generale dell'occupazione, dell'interesse e della moneta'' del 1936 \textbf{rifiuta
che il mercato venga lasciato libero} di raggiungere l'equlibrio spontaneamente e che \textbf{le 
risorse economiche vengano usate integralmente}. Nella crisi del 1929 il liberismo non funziona,
lo stato deve intervenire tramite lavori pubblici (commesse, profitti per le imprese, \ldots). Questo
però ha dei risvolti negativi
\begin{description}
  \item[Aumento del debito pubblico] Se è utile a far rinascere l'economia però è accettabile
  \item[Inflazione] Aumenta il denaro e aumenta la domanda
\end{description}
È importante che lo stato \textbf{raccolga le imposte} per gli investimenti pubblici. La 
\textbf{tassazione} deve essere \textbf{progressiva} (se è povero, il risparmio diventa domanda,
se è ricco diventa ancora più ricchezza con la flat-tax). Era inoltre favorevole a sussidi.\\
Gli anti-keynesiani lo criticano come socialista (per l'intervento dello stato). In realtà
``voleva salvare il capitalismo dai capitalisti'' senza perdere la democrazia.



\newpage
\listoftodos[Note]
\end{document}
