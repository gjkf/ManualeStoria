\documentclass[usenames,dvipsnames]{article}
\usepackage[italian]{babel}
\usepackage[latin1]{inputenc}
\usepackage{bm}
\usepackage{amsmath, amsfonts, amssymb, mathtools, calrsfs}
\usepackage{xfrac}
\usepackage[pdftex,dvipsnames]{xcolor}
\usepackage[colorinlistoftodos,prependcaption,textsize=tiny]{todonotes}
\usepackage{xtab, booktabs, array}
\usepackage[labelformat=empty]{caption}
\usepackage{tikz}
\usetikzlibrary{arrows,decorations,calc,intersections,shapes.geometric}
\usepackage{parskip} % To avoid indentation
\usepackage{textcomp} % To have \textcelsius and other symbols
\usepackage{multirow} % For nicer tables with split rows/columns
\usepackage{multicol}
\usepackage{cancel} % Cacelled equations and simplifications
\usepackage{hyperref}
\hypersetup{
    colorlinks,
    citecolor=black,
    filecolor=black,
    urlcolor=black,
    linkcolor=blue
}
\usepackage[all]{hypcap}

\usepackage[showframe=false, top=2cm, bottom=2.5cm, left=2.5cm, right=2.5cm]{geometry}

% Some useful TODO commands
\usepackage{xargs}
\newcommandx{\unsure}[2][1=]{\todo[linecolor=red,backgroundcolor=red!25,bordercolor=red,#1]{#2}}
\newcommandx{\change}[2][1=]{\todo[linecolor=blue,backgroundcolor=blue!25,bordercolor=blue,#1]{#2}}
\newcommandx{\info}[2][1=]{\todo[linecolor=OliveGreen,backgroundcolor=OliveGreen!25, bordercolor=OliveGreen,#1]{#2}}
\newcommandx{\improvement}[2][1=]{\todo[linecolor=Plum,backgroundcolor=Plum!25, bordercolor=Plum,#1]{#2}}
\newcommandx{\thiswillnotshow}[2][1=]{\todo[disable,#1]{#2}}
\setlength{\marginparwidth}{2cm}

% For an older but clearer root. Still \oldsqrt is valid
\usepackage{letltxmacro}
\makeatletter
\let\oldr@@t\r@@t
\def\r@@t#1#2{%
  \setbox0=\hbox{$\oldr@@t#1{#2\,}$}\dimen0=\ht0
  \advance\dimen0-0.2\ht0
  \setbox2=\hbox{\vrule height\ht0 depth -\dimen0}%
  {\box0\lower0.4pt\box2}}
\LetLtxMacro{\oldsqrt}{\sqrt}
\renewcommand*{\sqrt}[2][\ ]{\oldsqrt[#1]{#2} }
\makeatother

% Matrix spacing
\makeatletter
\renewcommand*\env@matrix[1][\arraystretch]{%
  \edef\arraystretch{#1}%
  \hskip -\arraycolsep
  \let\@ifnextchar\new@ifnextchar
  \array{*\c@MaxMatrixCols c}}
\makeatother

\newcommand\twodigits[1]{%
  \ifnum#1<10 0\number#1 \else #1\fi
}
\usepackage[yyyymmdd]{datetime}
\renewcommand{\dateseparator}{-}
\usepackage{fancyhdr}
\pagestyle{fancy}
\fancyhead{} % clear all header fields
\renewcommand{\headrulewidth}{0pt} % no line in header area
\fancyfoot{} % clear all footer fields
\fancyfoot[C,CO]{\thepage}% page number in "outer" position of footer line
\fancyfoot[R,RO]{Copyright \copyright 2017--\the\year$\,$Cossu Davide
} % other info in "inner" position of footer line
\fancyfoot[L,LO]{Version 0.0.1 \today
}

\DeclarePairedDelimiter\norm{\lVert}{\rVert} % ||v||  
\DeclarePairedDelimiter\abs{\lvert}{\rvert} % |v|
\newcommand\markangle[7][red]{% [color] origin A B radius radiusmark mark
  % fill red circle
  \begin{scope}
    \path[clip] (#2) -- (#3) -- (#4);
    \fill[color=#1,fill opacity=0.5,draw=#1,name path=circle]
    (#2) circle (#5);
  \end{scope}
  % middle calculation
  \path[name path=line one] (#2) -- (#3);
  \path[name path=line two] (#2) -- (#4);
  \path[%
    name intersections={of=line one and circle, by={inter one}},
    name intersections={of=line two and circle, by={inter two}}
  ] (inter one) -- (inter two) coordinate[pos=.5] (middle);
  % put mark
  \node at ($(#2)!#6!(middle)$) {#7};
}
\def\mathcolor#1#{\@mathcolor{#1}}
\def\@mathcolor#1#2#3{%
      \protect\leavevmode
      \begingroup
      \color#1{#2}#3%
      \endgroup
}
% For better visual in tables
\renewcommand*{\arraystretch}{2}
% To center with m{}
\newcolumntype{M}[1]{>{\centering\arraybackslash}m{#1}}
\newcommand{\divisor}{\rule{8.7cm}{0.4pt}}

\makeatletter
\newcommand*{\rom}[1]{\expandafter\@slowromancap\romannumeral #1@}
\makeatother

\begin{document}
{
\hypersetup{linkcolor=black}
\tableofcontents
}
%!TEX ROOT=storia.tex

\section{Seconda rivoluzione industriale}
La seconda rivoluzione industriale non ha dei limiti temporali definiti. La si può indicativamente
far andare dal \textbf{1870} al \textbf{1914} circa.\\
Una delle invenzioni che hanno caratterizzato questo periodo è stata quella del \textbf{motore
elettrico} e quella del \textbf{motore a scoppio}. Di conseguenza sono nate \textbf{dinamo},
\textbf{lampadine}, \textbf{aerei}, \textbf{telefoni} e \textbf{radio}.\\
A queste innovazioni si collega la nascita di molte industrie e aziende che producevano e sostenevano
queste innovazioni. Le più importanti furono aziende \textbf{chimiche}, \textbf{siderurgiche} ed
\textbf{elettriche}. Gli \emph{Stati Uniti} e la \emph{Germania} erano le più innovative nazioni,
superando persino l'\emph{Inghilterra} che però deteneva ancora il primato finanziario. Le altre 
nazioni stanno piano piano intraprendendo la strada dell'innovazione, l'Italia avrà il suo boom a 
fine '800.\\ [\baselineskip]
Il \textbf{Giappone} sta anch'esso industrializzandosi a poco a poco. Lì, è lo stato che decide di 
avere la stessa potenza dei paesi europei. Quindi lo stato invia ``spie'' a verificare cosa si fa
in Europa e il Giappone copia, e copia bene.\\
Anche l'\textbf{agricoltura} si comincia a modernizzare con l'uso di concimi chimici e macchine
agricole.\\ [\baselineskip]
Si cominciano a completare \textbf{reti ferroviarie} con locomotive a vapore che diventano elettiche,
acciaio per i binari. Viene inventata la \textbf{turbina} e l'\textbf{elica} e tutta la navigazione
diventa a motore, più sicura e rapida con costi minori. Ciò rendeva più conveniente i cibi americani
di quelli europei e si sviluppava la concorrenza. Così si cominciano anche a studiare metodi di
conservazione delle derrate alimentari.\\ [\baselineskip]
Lo sviluppo provoca una \textbf{forte deflazione} in quanto per la stessa domanda, l'offerta aumenta
considerevolemente. Viene questa definita la \textbf{\textit{Grande Depressione}}. Si sono attuate
3 diverse politiche per contrarstare questo fenomeno:
\begin{description}
  \item[Protezionismo] Gli imprenditori premono sui governi per aggiungere dazi e proteggere 
    l'industria interna. Nel 1873 la Germania introduce le prime tariffe, poi gli altri paesi si
    adegueranno. Da qui in poi lo stato interverrà sempre di più nella vita economica
  \item[Trust, cartelli e concentrazioni industriali] Si vengono a formare aziende frutto di fusione
    di altre più piccole
    \begin{description}
      \item[Cartelli] Accordi tra aziende che producono lo stesso bene per non farsi o ridurre la
        concorrenza (prezzi fissi, scelte di zone di vendita, \ldots). Genera prezzi più alti
      \item[Trust] Unione di aziende
        \begin{description}
          \item[Orizzontali] Che producono un bene e accorpano altre aziende del settore
          \item[Verticali] Che vanno dalla materia prima al bene finito. Sono le prime multinazionali
        \end{description}
    \end{description}
  \item[Commissioni statali] Lo stato alimenta direttamente alcune zone d'industria
\end{description}
Cambia anche il \textbf{rapporto tra aziende e banche}. Le più grandi aziende erano S.P.A. ma i fondi
non erano sufficienti, quindi chiedono dei prestiti alle banche con cui si indebiteranno. Le banche 
acquistano azioni dalle aziende finanziandole e diventandone co-proprietarie come forma di garanzia. 
La distinzione banca-azienda si fa sempre più debole. I consumatori sono danneggiati dall'aumento
dei prezzi, quindi si creano delle \textbf{norme anti-trust}.\\ [\baselineskip]
In campo sociale, c'è stata un'enorme espansione demografica, gli abitanti in Europa sono più che
raddoppiati in un secolo. Questo ha provocato un'eccedenza di mano d'opera nelle campagne che a sua
volta ha portato a una forte \textbf{emigrazione} dall'Europa verso l'America.\\ [\baselineskip]
In questo periodo si va anche a formare il \textbf{Taylorismo} ovvero l'\emph{organizzazione 
scientifica del lavoro}. Bisogna rendere il lavoro il più efficiente possibile, per fare ciò lo
si deve dividere, specializzare il lavoro in lavori più semplici e particolari. Questo porterebbe
a vantaggi per lavoratori (con salari più alti) e agli imprenditori. I sindacati erano contrari in
quanto il \textbf{lavoro era alienante}. Nel \textbf{1911} Ford crea la prima \textbf{catena di
montaggio}. La produzione era in serie, tutti i prodotti uguali con il lavoro suddiviso. Diventerà
un modello. Le industrie vanno sempre più verso la produzione di massa.

%!TEX ROOT=storia.tex

\section{Imperialismo}
In questo periodo di espansione economica si nota anche un'espansione coloniale. Pi� precisamente
avviene il fenomeno dell'\textbf{imperialismo}. Dalla fine dell'800 si attua una politica di
potenza coloniale che aveva come principali cause economiche (avere un mercato dove vendere i
propri prodotti, nuove materie prime, pi� mano d'opera, nuovi contratti statali, \ldots). Secondo
Lenin ``L'Imperialismo � la fase suprema del capitalismo''. L'imperialismo � quindi una causa 
dell'economia. Nascono da questo i movimenti \textbf{nazionalisti} non solo per questioni economiche
ma anche politiche (pi� territori si controllano, pi� si � prestigiosi) e militari. Alcuni movimenti
nazionalisti sfociano nel razzismo e nell'anti-semitismo.\\ [\baselineskip]
L'impero pi� grande era quello inglese (possedeva $\sfrac{1}{4}$ delle terre emerse e $\sfrac{1}{4}$
della popolazione). Quello francese era secondo ma meno ricco. Poi venivano tutti gli altri.\\
L'\textbf{Africa} era la nuova terra di conquista. Nel 1885 la spartizione era stata fatta a tavolino
su proposta di Bismark. Le spartizioni non tenevano minimamente conto delle popolazioni. 
L'Inghilterra voleva collegare Egitto e Sud Africa, la Francia voleva andare ad est (Marocco e 
Algeria), la Germania il Belgio e l'Italia quello che rimaneva. In \textbf{Asia} l'Inghilterra
ha l'India e la Birmania, la Francia l'Indonesia. La Cina non � stata conquistata perch� non ci
furono accordi a riguardo. Il Giapppone ha anche lui un impero (Corea). La Russia si espande verso
est fino al Giappone e a sud fino all'Afghanistan. Anche gli Stati Uniti, nati come stato coloniale
si espandono verso l'America centro-meridionale. Spacciavano le conquiste come ``liberazioni''. Gli
USA aiutano Cuba con l'indipendenza dalla Spagna per� scrivono loro la costituzione e tengono le
basi militari. Fanno lo stesso a Puerto Rico e nelle Filippine. Fanno nascere un movimento di rivolta
a Panama e nasce lo stato panamense. Gli USA hanno il controllo del canale per un secolo.

%!TEX ROOT=storia.tex

\section{Societ� di massa}
La societ� di massa � la societ� industrializzata di fine '800. L'agricoltura ha un'importanza sempre
minore, il settore terziario invece aumenta e si ingrandisce. Le citt� si ingrandiscono e diventa
una societ� \textbf{sempre pi� complessa}. Gli operai aumentano e si dividono in ruoli, la borghesia
aumenta il suo potere. La societ� si va stratificando sempre di pi�. I colletti bianchi (media 
borghesia) aumentano sempre di pi� di numero, aumentano i dipendenti pubblici (lo stato interviene
nella vita sociale). La piccola/media borghesia aveva un tenore di vita simile a quello degli operai
ma facevano di tutto pur di distinguersi (in questo clima di disagio nascono i partiti di estrema
destra).\\ [\baselineskip]
L'istruzione si diffonde sempre di pi�, piano piano. Pi� giornali vengono venduti, nascono i
giornali sportivi e si diffonde lo sport.\\ [\baselineskip]
Gli eserciti si rinforzano (leva obbligatoria) e gli ufficiali diffondono idee di patriottismo, 
\ldots. Favorivano lo studio delle lingue e la nascita di nuove idee.\\ [\baselineskip]
Il suffragio si allarga sempre di pi�. Il suffragio � universale maschile prima della WW1 e anche
in qualche paese femminile.

\subsection{Partiti socialisti}
I primi partiti sono quelli socialisti. La Seconda Internazionale si tiene a Parigi
nel 1889. Il pi� grande partito � quello \textbf{social-democratico tedesco}. L'obiettivo era di
coordinare i partiti per ottenere migliori condizioni lavorative per gli operai. Erano sostenitori
dell'\textbf{internazionalismo}. L'ideale di nazione � un ideale borghese, il proletariato non �
nazionale.\\ [\baselineskip]
Erano divisi in due correnti
\begin{description}
  \item[Rivoluzionari] Volevano i cambiamenti con violenza, senza riforme
  \item[Riformisti] Volevano i cambiamenti con graduali riforme, in modo pacifico
\end{description}
Tra i \emph{Riformisti}, \textbf{Bernstein} era uno dei pi� importanti. Nel 1899 pubblica 
\emph{``I presupposti del socialismo e i compiti della social-democrazia''}. I presupposti e gli 
ideali sono gli stessi di Marx ma lui ha commesso un errore: la situazione non sta peggiorando e la 
borghesia non si sta proletarizzando. Il crollo del capitalismo non � quindi vicino, � necessario 
migliorare la situazione dei lavoratori tramite riforme.\\ [\baselineskip]
Agli inizi del '900, si formano piccoli gruppi di rivoluzionari (estrema sinistra). Il primo era
guidato da \textbf{Lenin}. Il proletariato da solo non pu� fare la rivoluzione, ha bisogno del
partito come guida perch� non ha la coscienza di classe. Il partito � fatto da intellettuali che
pensano di capire l'economia. � composto da un'el�te di rivoluzionari per professione.\\ 
[\baselineskip]
Nel 1903 si tiene il congresso del PSD, a Londra. Lenin ottiene la maggioranza. Tra queste correnti
c'erano anche dei \emph{sindacalisti rivoluzionari} tra cui \textbf{Sorel}. Pubblica nel 1905
\emph{``Considerazioni sulla violenza''}. Erano critici contro i partiti socialisti che tendevano ad
allontanarsi dal proletariato ed erano guidati da chi viveva come un borghese. Il sindacato invece
era fatto da lavoratori che erano a stretto contatto con i proletari. L'azione spontanea � esaltata.
L'inizio della rivoluzione sarebbe stato uno sciopero generale che metter� in crisi l'economia
capitalista. � una forma di \emph{anarco-sindacalismo}.

\subsection{Partiti nazionalisti}
In questa societ� di massa si vengono a formare anche dei partiti nazionalisti. L'idea di fondo era
di valorizzare la forza e la potenza della nazione. Sono \textbf{interclassisti} in quanto tutte le
classi sociali devono collaborare per la forza della nazione. Il modello � l'esercito e la sua 
gerarchia. Sono a favore del protezionismo e dell'imperialismo. Le idee democratiche sono pericolose,
al potere deve starci chi ha veramente l'abilit�. La libert� deve essere ridotta.\\
C'erano alcuni partiti di spicco
\begin{description}
  \item[Pangermanesimo] nazionalismo tedesco che voleva riunire tutti i tedeschi in un unico stato
  \item[Revanescismo] nazionalismo francese che voleva la rivincita contro i tedeschi
  \item[Panslavismo] nazionalismo slavo per riunire tutti gli stati sotto la Russia
  \item[Inglese] per il colonialismo e l'impero
  \item[Italiano] \textbf{Enrico Corradini} � il primo ideologo. Usava un linguaggio marxista con
    significato nazionalista. Ci sono due tipi di nazioni: \emph{borghesi} (ricche, coloniali, 
    \ldots) e \emph{proletarie} (giovani, povere, sovrappopolate). L'Italia rientra in queste ultime.
\end{description}
Il \textbf{razzismo} � anche un fenomeno che � collegato al nazionalismo. La societ� umana � divisa 
in \emph{razze} che si differenziano non solo per le qualit� fisiche, ma anche per quelle morali e 
culturali che dipendono da fattori biologici.\\
\textbf{De Gobineau} � uno degli esponenti. Pubblica \emph{``Saggio sull'inuguaglianza delle razze
umane''}. Ci sono 3 razze: gialla, nera e bianca. La bianca (ariana = Europa centro-nord) � la
superiore sia sul piano fisico che intellettuale. Ha creato la cultura e solo quella ha i veri
valori. Il razzismo teme l'\textbf{ibridazione} ovvero la mescolanza fra razze. Il sangue non deve 
contaminarsi. Secondo De Gobineau sopratutto le classi superiori (di cui lui fa parte) rappresentano 
la razza ariana.\\
Legato al razzismo c'� anche l'\textbf{antisemitismo}. Comunit� ebraiche ci sono sempre state in 
Europa. Nel Medioevo erano \emph{infedeli}, dal '500 in avanti vivono in ghetti, solo nel '700 
cominciano ad integrarsi meglio. L'antisemitismo non � scomparso ma modificato. Dopo l'emancipazione,
gli ebrei si sono assimilati alla societ� e alcuni hanno anche avuto successo. Gli ebrei erano una
razza che si contrapponeva a quella ariana, anche dopo la conversione si rimaneva ebrei. La loro
pericolosit� deriva dalla loro somiglianza a noi (Chamberlain, \emph{``Fondamenti del \rom{19} 
secolo''}).
Due sono i casi-esempio di anti-semitismo che vanno ricordati
\begin{description}
  \item[Caso Dreyfuss] Dreyfuss era un capitano francese che faceva parte dello Stato Maggiore. Era
    ebreo. Nel 1894 i servizi segreti francesi scoprono che nello Stato Maggiore c'era una spia.
    Essendo l'unico ebreo, Dreyfuss fu sospettato e messo sotto processo. Vengon create prove false
    e poi condannato. Emergono ora due correnti di pensiero
    \begin{description}
      \item[Dreyfussardi] A favore di Dreyfuss (democratici, socialisti)
      \item[Anti-dreyfussardi] Contro Dreyfuss (nazionalisti, Chiesa)
    \end{description}
    Dopo qualche anno il processo viene rivisto e nel 1906 Dreyfuss � stato reintegrato.
  \item[I (falsi) protoclli dei Savi di Sion] � un libro in cui si descrive un complotto ideato
    dai rabbini per fare in modo che gli ebrei governino il mondo. Fu considerata la dimostrazione
    della pericolosit� degli ebrei. Si scopr� poi che in realt� era un falso scritto dai servizi
    segreti zaristi in quanto alcune parti erano ricopiate da romanzi di bassa lega dell'800.
    Nonostante questo c'� chi ancora crede siano veri.
\end{description}
Sotto questi influssi nasce il \textbf{Sionismo} overo il nazionalismo ebraico. 
\emph{Theodor Herzl} era un ungherese, fondatore del sionismo. Era il tipico ebreo assimilato, 
socialista e non religioso. Va a Parigi a seguire il caso Dreyfuss e nota che gli ebrei vogliono 
assimilarsi ai cristiani ma non possono perch� l'antisemitismo � troppo forte. Serve uno stato 
ebraico. Nel 1896 scrive \emph{``Lo Stato Ebraico''}. Deve nascere per accordi internazionali ed 
essere neutrale. Lo stato non � necesssariamente la Palestina. Viene creata un'organizzazione 
sionista che si riunisce la prima volta a Basilea nel 1897. L'unico territorio che avesse senso er
la Palestina che era sotto l'impero Ottomano. Non ottennero nulla. Nel '900 cambiano strategia 
facendo emigrare gli ebrei verso la Palestina dove avrebbero comprato terra e fatto i contadini. 
Dopo la WW1 cominciano i problemi in quanto gli arabi non volevano si costituisse uno stato.

\subsection{Partiti cattolici}
Nel parlamento a sinistra c'erano i socialisti, a destra i nazionalisti e in centro i cattolici.
In Germania nasce la CDU. Erano partiti sotenuti dalla Chiesa.\\
Pio \rom{9} era molto conservatore politicamente, invitava a non impegnarsi politicamente. Muore 
nel '76. Il successore Leone \rom{13} cambia atteggiamento. Nel 1891 pubblica 
\textbf{\emph{``Rerum Novarum''}} che non � altro che la dottrina sociale della Chiesa. Interviene
per la prima volta sulla "questione operaia". I lavoratori hanno dei doveri nei confronti del 
datore di lavoro (impegno, fedelt�, rispetto) ma anche dei diritti (giusto stipendio, corretti
trattamenti). Non � una riforma socialista, anzi, critica i socialisti (sono atei, senza propriet�
privata, crea lotta di classe). Voleva evitare una perdita di contatto con i lavoratori. Non �
nemmeno liberista (critica l'individualismo, esclude lo stato dalla vita economica). Rifiuta i
sindacati ma promuove le corporazioni. In Italia esistevano sindacati ma non corporazioni, non
sarebbero pi� stati credibili. I sindacati cattolici non sempre seguivano il Papa, se avessero 
rinunciato agli scioperi, sarebbero sembrati deboli.\\
Dopo la Rerum Novarum molti cattolici si sentono spinti verso la vita sociale. Agli inizi del '900
comincia a nascere la \textbf{Democrazia Cristiana}: Romolo Marri e Luigi Sturzo sono sacerdoti che
volevano fare un partito. Pio \rom{10} era pi� conservatore del predecessore e blocca l'iniziativa.
Far nascere in Italia un partito significherebbe riconoscere lo stato Italiano. Sturzo abbandona,
Marri invece continua, abbandona la Chiesa. Proprio Sturzo per� nel '19 con il sostegno della 
Chiesa fonder� il partito Cattolico.

%!TEX ROOT=storia.tex

\section{Et� Giolittiana}
L'et� giolittiana va dal 1900 al 1914.

\subsection{Economia}
Il periodo � caratterizzato da generale crescita economica. L'industrializzazione cresce ma solo in
alcune zone (Lombardia, Piemonte, Liguria) e solo alcuni settori.

\subsection{Societ�}
C'� una fortissima emigrazione (circa 500k all'anno).

\subsection{Politica}
Giolitti era un piemontese liberale. La formazione dei sindacati era inevitabile in quanto � una 
tendenza causata dall'industrializzazione. Lo stato \textbf{non deve impedire l'organizzazione}
perch� altrimenti si organizzano clandestinamente contro lo Stato. Non deve reprimere manifestazioni
pacifiche.\\
Giolitti cerc� alleanze con socialisti (offre a Turati un ministero se il PSI si fosse alelato con
il governo, rifiuta per non dividere il partito). Quando Giolitti era presidente del Consiglio, era
anche ministro dell'Interno.\\
Nel 1906 � fondata la \textbf{CGL} legata al partito socialista. Guidata da socialisti riformisti.\\
Nel 1910 � fondata la \textbf{Confindustria}.\\
Giolitti ha portato avanti importanti riforme tra cui le \textbf{prime leggi per regolamentare il 
lavoro} (obbligo del riposo festivo, vietato per donne e bambini il lavoro notturno). Nel 1911 viene 
creata l'INA (Istituto Nazionale Assicurazioni) a cui � dato il monopolio delle assicurazioni sulla 
vita. In questo modo i lavoratori erano pi� sicuri e i fondi andavano a formare un sistema 
previdenziale. Nel 1913 viene data una pensione agli infortunati sul lavoro.\\
Vengono \textbf{nazionalizzate le ferrovie} cos� si sarebbe risparmiato e si sarebbero collegati 
anche i punti pi� sfavorevoli. Fa anche riforme per il sud che dovevano favorire lo sviluppo
(costruito un acquedotto in Puglia, \ldots), in realt� non ebbero grandi risultati in quanto i
provvedimenti erano clientelari (favoritismi, \ldots).\\ [\baselineskip]
Nel 1912 � stata varata una \textbf{riforma elettorale} che permetteva il suffragio universale 
maschile per chi avesse avuto 21 anni e fato la leva militare o 30 altrimenti. Circa 9 milioni di
elettori. Uninominale a doppio turno (1 deputato per collegio, 50\% dei voti al primo turno, 
ballottaggio dei primi due). Nel 1913 le prime elezioni di massa. I socialisti erano organizzati per 
massa, non i liberali. Cos� si form� il \textbf{Patto Gentiloni} che sanciva che i candidati liberali
cattolici sarebbero stati sostenuti dalla Chiesa se poi in parlamento non avessero sostenuto 
provvedimenti che la Chiesa riteneva scomodi (divorzio, scuole cattoliche, \ldots). La Chiesa temeva
i socialisti, viene cos� eliminato il ``Non Expedit'' e i cattolici entrano nella vita dello stato
italiano.

\subsection{Esteri}
Tra il 1911 e il 1912 Giolitti intraprende una guerra coloniale. Furono presi accordi segreti con la 
Francia: l'Italia permette concede il Marocco alla Francia, la Francia non ostacola l'Italia. La
Chiesa sosteneva la guerra come fosse di civilt�. La guerra fu durissima, quasi barbara 
(avvelenamenti, capi di concentramento).\\
Nel 1912 si stipula la \textbf{Pace di Losanna}. La Libia ora � colonia Italiana. La Libia era 
allettante per l'economia secondo Giolitti, non tutti erano d'accordo (Sanvemini disse che la Libia
era una ``Scatola di Sabbia'').


\newpage
\listoftodos[Note]
\end{document}
