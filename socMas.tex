%!TEX ROOT=storia.tex

\section{Società di massa}
La società di massa è la società industrializzata di fine '800. L'agricoltura ha un'importanza sempre
minore, il settore terziario invece aumenta e si ingrandisce. Le città si ingrandiscono e diventa
una società \textbf{sempre più complessa}. Gli operai aumentano e si dividono in ruoli, la borghesia
aumenta il suo potere. La società si va stratificando sempre di più. I colletti bianchi (media 
borghesia) aumentano sempre di più di numero, aumentano i dipendenti pubblici (lo stato interviene
nella vita sociale). La piccola/media borghesia aveva un tenore di vita simile a quello degli operai
ma facevano di tutto pur di distinguersi (in questo clima di disagio nascono i partiti di estrema
destra).\\ [\baselineskip]
L'istruzione si diffonde sempre di più, piano piano. Più giornali vengono venduti, nascono i
giornali sportivi e si diffonde lo sport.\\ [\baselineskip]
Gli eserciti si rinforzano (leva obbligatoria) e gli ufficiali diffondono idee di patriottismo, 
\ldots. Favorivano lo studio delle lingue e la nascita di nuove idee.\\ [\baselineskip]
Il suffragio si allarga sempre di più. Il suffragio è universale maschile prima della WW1 e anche
in qualche paese femminile.

\subsection{Partiti socialisti}
I primi partiti sono quelli socialisti. La Seconda Internazionale si tiene a Parigi
nel 1889. Il più grande partito è quello \textbf{social-democratico tedesco}. L'obiettivo era di
coordinare i partiti per ottenere migliori condizioni lavorative per gli operai. Erano sostenitori
dell'\textbf{internazionalismo}. L'ideale di nazione è un ideale borghese, il proletariato non è
nazionale.\\ [\baselineskip]
Erano divisi in due correnti
\begin{description}
  \item[Rivoluzionari] Volevano i cambiamenti con violenza, senza riforme
  \item[Riformisti] Volevano i cambiamenti con graduali riforme, in modo pacifico
\end{description}
Tra i \emph{Riformisti}, \textbf{Bernstein} era uno dei più importanti. Nel 1899 pubblica 
\emph{``I presupposti del socialismo e i compiti della social-democrazia''}. I presupposti e gli 
ideali sono gli stessi di Marx ma lui ha commesso un errore: la situazione non sta peggiorando e la 
borghesia non si sta proletarizzando. Il crollo del capitalismo non è quindi vicino, è necessario 
migliorare la situazione dei lavoratori tramite riforme.\\ [\baselineskip]
Agli inizi del '900, si formano piccoli gruppi di rivoluzionari (estrema sinistra). Il primo era
guidato da \textbf{Lenin}. Il proletariato da solo non può fare la rivoluzione, ha bisogno del
partito come guida perché non ha la coscienza di classe. Il partito è fatto da intellettuali che
pensano di capire l'economia. È composto da un'elìte di rivoluzionari per professione.\\ 
[\baselineskip]
Nel 1903 si tiene il congresso del PSD, a Londra. Lenin ottiene la maggioranza. Tra queste correnti
c'erano anche dei \emph{sindacalisti rivoluzionari} tra cui \textbf{Sorel}. Pubblica nel 1905
\emph{``Considerazioni sulla violenza''}. Erano critici contro i partiti socialisti che tendevano ad
allontanarsi dal proletariato ed erano guidati da chi viveva come un borghese. Il sindacato invece
era fatto da lavoratori che erano a stretto contatto con i proletari. L'azione spontanea è esaltata.
L'inizio della rivoluzione sarebbe stato uno sciopero generale che metterà in crisi l'economia
capitalista. È una forma di \emph{anarco-sindacalismo}.

\subsection{Partiti nazionalisti}
In questa società di massa si vengono a formare anche dei partiti nazionalisti. L'idea di fondo era
di valorizzare la forza e la potenza della nazione. Sono \textbf{interclassisti} in quanto tutte le
classi sociali devono collaborare per la forza della nazione. Il modello è l'esercito e la sua 
gerarchia. Sono a favore del protezionismo e dell'imperialismo. Le idee democratiche sono pericolose,
al potere deve starci chi ha veramente l'abilità. La libertà deve essere ridotta.\\
C'erano alcuni partiti di spicco
\begin{description}
  \item[Pangermanesimo] nazionalismo tedesco che voleva riunire tutti i tedeschi in un unico stato
  \item[Revanescismo] nazionalismo francese che voleva la rivincita contro i tedeschi
  \item[Panslavismo] nazionalismo slavo per riunire tutti gli stati sotto la Russia
  \item[Inglese] per il colonialismo e l'impero
  \item[Italiano] \textbf{Enrico Corradini} è il primo ideologo. Usava un linguaggio marxista con
    significato nazionalista. Ci sono due tipi di nazioni: \emph{borghesi} (ricche, coloniali, 
    \ldots) e \emph{proletarie} (giovani, povere, sovrappopolate). L'Italia rientra in queste ultime.
\end{description}
Il \textbf{razzismo} è anche un fenomeno che è collegato al nazionalismo. La società umana è divisa 
in \emph{razze} che si differenziano non solo per le qualità fisiche, ma anche per quelle morali e 
culturali che dipendono da fattori biologici.\\
\textbf{De Gobineau} è uno degli esponenti. Pubblica \emph{``Saggio sull'inuguaglianza delle razze
umane''}. Ci sono 3 razze: gialla, nera e bianca. La bianca (ariana = Europa centro-nord) è la
superiore sia sul piano fisico che intellettuale. Ha creato la cultura e solo quella ha i veri
valori. Il razzismo teme l'\textbf{ibridazione} ovvero la mescolanza fra razze. Il sangue non deve 
contaminarsi. Secondo De Gobineau sopratutto le classi superiori (di cui lui fa parte) rappresentano 
la razza ariana.\\
Legato al razzismo c'è anche l'\textbf{antisemitismo}. Comunità ebraiche ci sono sempre state in 
Europa. Nel Medioevo erano \emph{infedeli}, dal '500 in avanti vivono in ghetti, solo nel '700 
cominciano ad integrarsi meglio. L'antisemitismo non è scomparso ma modificato. Dopo l'emancipazione,
gli ebrei si sono assimilati alla società e alcuni hanno anche avuto successo. Gli ebrei erano una
razza che si contrapponeva a quella ariana, anche dopo la conversione si rimaneva ebrei. La loro
pericolosità deriva dalla loro somiglianza a noi (Chamberlain, \emph{``Fondamenti del \rom{19} 
secolo''}).
Due sono i casi-esempio di anti-semitismo che vanno ricordati
\begin{description}
  \item[Caso Dreyfuss] Dreyfuss era un capitano francese che faceva parte dello Stato Maggiore. Era
    ebreo. Nel 1894 i servizi segreti francesi scoprono che nello Stato Maggiore c'era una spia.
    Essendo l'unico ebreo, Dreyfuss fu sospettato e messo sotto processo. Vengon create prove false
    e poi condannato. Emergono ora due correnti di pensiero
    \begin{description}
      \item[Dreyfussardi] A favore di Dreyfuss (democratici, socialisti)
      \item[Anti-dreyfussardi] Contro Dreyfuss (nazionalisti, Chiesa)
    \end{description}
    Dopo qualche anno il processo viene rivisto e nel 1906 Dreyfuss è stato reintegrato.
  \item[I (falsi) protoclli dei Savi di Sion] È un libro in cui si descrive un complotto ideato
    dai rabbini per fare in modo che gli ebrei governino il mondo. Fu considerata la dimostrazione
    della pericolosità degli ebrei. Si scoprì poi che in realtà era un falso scritto dai servizi
    segreti zaristi in quanto alcune parti erano ricopiate da romanzi di bassa lega dell'800.
    Nonostante questo c'è chi ancora crede siano veri.
\end{description}
Sotto questi influssi nasce il \textbf{Sionismo} overo il nazionalismo ebraico. 
\emph{Theodor Herzl} era un ungherese, fondatore del sionismo. Era il tipico ebreo assimilato, 
socialista e non religioso. Va a Parigi a seguire il caso Dreyfuss e nota che gli ebrei vogliono 
assimilarsi ai cristiani ma non possono perché l'antisemitismo è troppo forte. Serve uno stato 
ebraico. Nel 1896 scrive \emph{``Lo Stato Ebraico''}. Deve nascere per accordi internazionali ed 
essere neutrale. Lo stato non è necesssariamente la Palestina. Viene creata un'organizzazione 
sionista che si riunisce la prima volta a Basilea nel 1897. L'unico territorio che avesse senso er
la Palestina che era sotto l'impero Ottomano. Non ottennero nulla. Nel '900 cambiano strategia 
facendo emigrare gli ebrei verso la Palestina dove avrebbero comprato terra e fatto i contadini. 
Dopo la WW1 cominciano i problemi in quanto gli arabi non volevano si costituisse uno stato.

\subsection{Partiti cattolici}
Nel parlamento a sinistra c'erano i socialisti, a destra i nazionalisti e in centro i cattolici.
In Germania nasce la CDU. Erano partiti sotenuti dalla Chiesa.\\
Pio \rom{9} era molto conservatore politicamente, invitava a non impegnarsi politicamente. Muore 
nel '76. Il successore Leone \rom{13} cambia atteggiamento. Nel 1891 pubblica 
\textbf{\emph{``Rerum Novarum''}} che non è altro che la dottrina sociale della Chiesa. Interviene
per la prima volta sulla "questione operaia". I lavoratori hanno dei doveri nei confronti del 
datore di lavoro (impegno, fedeltà, rispetto) ma anche dei diritti (giusto stipendio, corretti
trattamenti). Non è una riforma socialista, anzi, critica i socialisti (sono atei, senza proprietà
privata, crea lotta di classe). Voleva evitare una perdita di contatto con i lavoratori. Non è
nemmeno liberista (critica l'individualismo, esclude lo stato dalla vita economica). Rifiuta i
sindacati ma promuove le corporazioni. In Italia esistevano sindacati ma non corporazioni, non
sarebbero più stati credibili. I sindacati cattolici non sempre seguivano il Papa, se avessero 
rinunciato agli scioperi, sarebbero sembrati deboli.\\
Dopo la Rerum Novarum molti cattolici si sentono spinti verso la vita sociale. Agli inizi del '900
comincia a nascere la \textbf{Democrazia Cristiana}: Romolo Marri e Luigi Sturzo sono sacerdoti che
volevano fare un partito. Pio \rom{10} era più conservatore del predecessore e blocca l'iniziativa.
Far nascere in Italia un partito significherebbe riconoscere lo stato Italiano. Sturzo abbandona,
Marri invece continua, abbandona la Chiesa. Proprio Sturzo però nel '19 con il sostegno della 
Chiesa fonderà il partito Cattolico.
