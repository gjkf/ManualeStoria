%!TEX ROOT=storia.tex

\section{Societ� di massa}
La societ� di massa � la societ� industrializzata di fine '800. L'agricoltura ha un'importanza sempre
minore, il settore terziario invece aumenta e si ingrandisce. Le citt� si ingrandiscono e diventa
una societ� \textbf{sempre pi� complessa}. Gli operai aumentano e si dividono in ruoli, la borghesia
aumenta il suo potere. La societ� si va stratificando sempre di pi�. I colletti bianchi (media 
borghesia) aumentano sempre di pi� di numero, aumentano i dipendenti pubblici (lo stato interviene
nella vita sociale). La piccola/media borghesia aveva un tenore di vita simile a quello degli operai
ma facevano di tutto pur di distinguersi (in questo clima di disagio nascono i partiti di estrema
destra).\\ [\baselineskip]
L'istruzione si diffonde sempre di pi�, piano piano. Pi� giornali vengono venduti, nascono i
giornali sportivi e si diffonde lo sport.\\ [\baselineskip]
Gli eserciti si rinforzano (leva obbligatoria) e gli ufficiali diffondono idee di patriottismo, 
\ldots. Favorivano lo studio delle lingue e la nascita di nuove idee.\\ [\baselineskip]
Il suffragio si allarga sempre di pi�. Il suffragio � universale maschile prima della WW1 e anche
in qualche paese femminile.

\subsection{Partiti socialisti}
I primi partiti sono quelli socialisti. La Seconda Internazionale si tiene a Parigi
nel 1889. Il pi� grande partito � quello \textbf{social-democratico tedesco}. L'obiettivo era di
coordinare i partiti per ottenere migliori condizioni lavorative per gli operai. Erano sostenitori
dell'\textbf{internazionalismo}. L'ideale di nazione � un ideale borghese, il proletariato non �
nazionale.\\ [\baselineskip]
Erano divisi in due correnti
\begin{description}
  \item[Rivoluzionari] Volevano i cambiamenti con violenza, senza riforme
  \item[Riformisti] Volevano i cambiamenti con graduali riforme, in modo pacifico
\end{description}
Tra i \emph{Riformisti}, \textbf{Bernstein} era uno dei pi� importanti. Nel 1899 pubblica 
\emph{``I presupposti del socialismo e i compiti della social-democrazia''}. I presupposti e gli 
ideali sono gli stessi di Marx ma lui ha commesso un errore: la situazione non sta peggiorando e la 
borghesia non si sta proletarizzando. Il crollo del capitalismo non � quindi vicino, � necessario 
migliorare la situazione dei lavoratori tramite riforme.\\ [\baselineskip]
Agli inizi del '900, si formano piccoli gruppi di rivoluzionari (estrema sinistra). Il primo era
guidato da \textbf{Lenin}. Il proletariato da solo non pu� fare la rivoluzione, ha bisogno del
partito come guida perch� non ha la coscienza di classe. Il partito � fatto da intellettuali che
pensano di capire l'economia. � composto da un'el�te di rivoluzionari per professione.\\ 
[\baselineskip]
Nel 1903 si tiene il congresso del PSD, a Londra. Lenin ottiene la maggioranza. Tra queste correnti
c'erano anche dei \emph{sindacalisti rivoluzionari} tra cui \textbf{Sorel}. Pubblica nel 1905
\emph{``Considerazioni sulla violenza''}. Erano critici contro i partiti socialisti che tendevano ad
allontanarsi dal proletariato ed erano guidati da chi viveva come un borghese. Il sindacato invece
era fatto da lavoratori che erano a stretto contatto con i proletari. L'azione spontanea � esaltata.
L'inizio della rivoluzione sarebbe stato uno sciopero generale che metter� in crisi l'economia
capitalista. � una forma di \emph{anarco-sindacalismo}.

\subsection{Partiti nazionalisti}
In questa societ� di massa si vengono a formare anche dei partiti nazionalisti. L'idea di fondo era
di valorizzare la forza e la potenza della nazione. Sono \textbf{interclassisti} in quanto tutte le
classi sociali devono collaborare per la forza della nazione. Il modello � l'esercito e la sua 
gerarchia. Sono a favore del protezionismo e dell'imperialismo. Le idee democratiche sono pericolose,
al potere deve starci chi ha veramente l'abilit�. La libert� deve essere ridotta.\\
C'erano alcuni partiti di spicco
\begin{description}
  \item[Pangermanesimo] nazionalismo tedesco che voleva riunire tutti i tedeschi in un unico stato
  \item[Revanescismo] nazionalismo francese che voleva la rivincita contro i tedeschi
  \item[Panslavismo] nazionalismo slavo per riunire tutti gli stati sotto la Russia
  \item[Inglese] per il colonialismo e l'impero
  \item[Italiano] \textbf{Enrico Corradini} � il primo ideologo. Usava un linguaggio marxista con
    significato nazionalista. Ci sono due tipi di nazioni: \emph{borghesi} (ricche, coloniali, 
    \ldots) e \emph{proletarie} (giovani, povere, sovrappopolate). L'Italia rientra in queste ultime.
\end{description}
Il razzismo � anche un fenomeno che � collegato al nazionalismo. La societ� umana � divisa in
\emph{razze} che si differenziano non solo per le qualit� fisiche, ma anche per quelle morali e 
culturali che dipendono da fattori biologici.\\
\textbf{De Gobineau} � uno degli esponenti. Pubblica \emph{``Saggio sull'inuguaglianza delle razze
umane''}. Ci sono 3 razze: gialla, nera e bianca. La bianca (ariana = Europa centro-nord) � la
superiore sia sul piano fisico che intellettuale. Ha creato la cultura e solo quella ha i veri
valori.
