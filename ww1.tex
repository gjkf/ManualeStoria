%!TEX ROOT=storia.tex

\section{La Prima Guerra Mondiale}
Come ogni fenomeno complesso, la guerra non ha avuto una sola causa. Forse nessuno dei fattori, presi
singolarmente, sarebbe bastato.\\
Gli storici marxisti sottolineavano le \textbf{cause economiche} (concorrenza industriale, 
protezionismo e guerre doganali, concorrenza coloniale).\\

\subsection{Bismarck e la Germania}
Bismarck è rimasto cancelliere fino al 1890. Fino ad allora non aveva fatto una politica coloniale
in quanto sarebbe entrato in conflitto con l'Inghilterra e doveva mantenere buoni rapporti con la
Russia (il suo principale obiettivo era isolare politicamente la Francia). Bismarck si dimette nel
1890.\\
Guglielmo \rom{2} voleva una politica più aggressiva, coloniale. Quindi minaccia gli Inglesi creando
una flotta che possa competere con la loro. \textbf{Francia e Inghilterra si accordano sulle
colonie}.\\
I tedeschi ottengono un appalto per la costruzione di una ferrovia da Istanbul a Baghdad. Favoriscono
così il commercio con l'impero Ottomano delle merci tedesche.\\
Il Marocco era diviso a metà tra Spagnoli e Ottomani. Sia la Francia che la Germania lo volevano.
\textbf{Due crisi Marocchine}: 1905--1906, vinta dall'alleanza Inghilterra-Francia e 1911. 
\textbf{Francia e Germania sono sull'orlo della guerra}.\\
La Germania aveva solo l'Austria come alleata ma era continentale, senza sbocchi sul mare. L'Italia
aveva accordi con la Francia. Ormai la guerra pareva come l'unica maniera per realizzare i piani
tedeschi.\\ [\baselineskip]
Negli anni '90 Francia e Russia fanno un'alleanza militare, così come anche Inghilterra e Russia.
Ci sono ora due schieramenti
\begin{description}
  \item[Triplice Alleanza] Germania, Austria, Italia
  \item[Triplice Intesa] Francia, Inghilterra, Russia (accordi bilaterali)
\end{description}

\subsection{L'inizio della Guerra}
Nei Balcani c'era un contrasto fra Austria e Russia. Molti stati ottengono l'indipendenza, tra cui
la Serbia (che conteneva anche Croazia e Slovenia). Si forma così la \textbf{Iugoslavia} ovvero lo
Stato degli Slavi del Sud. Alcune delle popolazioni erano sotto l'Austria però. La Russia era alleata
della Serbia. La guerra era ormai scontata anche per i movimenti nazionalisti che si andavano
diffondendo che offrivano una visione della guerra come modo per dimostrare la forza.\\ 
[\baselineskip]
Il \textbf{28 giugno 1914} Gavrilo Princip assasina Franecsco Ferdinando per protesta dell'annessione
della Bosnia all'Austria. Lo fa con il sostegno dei servizi segreti Serbi. Il \textbf{28 luglio}
scoppia la guerra. Fra il 28 e il 4 agosto si attivano le alleanze: \textbf{Austria e Germania} 
contro \textbf{Russia, Francia, Inghilterra e Serbia}.

\subsubsection{La questione Italiana}
\textbf{Antonio Salandra} guida il governo in modo liberale, di Destra. \textbf{Sonnino} è il
ministro degli esteri. Dichiara la
\textbf{neutralità} dicendo che l'alleanza era difensiva. La popolazione si divide in due: 
\textbf{Neutralisti} e \textbf{Interventisti}.\\
\textbf{Giolitti} voleva la neutralità in quanto non sarebbe stata sostenibile un'altra guerra dopo
quella in Libia. Contrattando la neutralità invece si sarebbe potuto ottenere molto. La Chiesa
condivideva. \textbf{La Chiesa} esprimeva i pensieri dei contadini: una guerra contro gli Austriaci,
cattolici, non era vista bene (Benedetto \rom{15}) era il nuovo papa. Anche i \textbf{socialisti} 
erano neutralisti in quanto si sarebbe intaccato l'internazionalismo.\\
\textbf{I Nazionalisti} invece erano interventisti, per dimostrare la propria forza, i 
\textbf{democratici} si ricollegavano a Mazzini e all'idea di un completamento del Risorgimento
italiano con l'annessione delle terre irredente. Anche i \textbf{sindacati rivoluzionari} erano 
favorevoli in quanto ritenevano che la guerra avrebbe scosso il capitalismo e fatto crollare, creando
i presupposti per una rivoluzione. Infine anche i \textbf{liberali conservatori}.\\ [\baselineskip]
\textbf{Dopo mesi, entriamo in guerra contro Austria e Germania}. Per convenienza. Il 26 aprile 1915
fu stipulato segretamente il \textbf{Patto di Londra} tra l'Italia e l'Intesa. Entro un mese l'Italia
sarebbe dovuta entrare in guerra contro l'Austria, in cambio avrebbe ricevuto
\begin{enumerate}
  \item Le terre irredente
  \item L'Alto-Adige
  \item L'Istria
  \item La Dalmazia e il porto di Valona
  \item Il controllo della politica estera dell'Albania
  \item Il Dodecaneso
  \item Un bacino di carbone in Turchia
  \item Alcune colonie tedesche in Africa
\end{enumerate}
L'Italia ora doveva entrare in guerra ma i neutralisti erano in maggioranza in parlamento e tra il
popolo. \textbf{Salandra si dimette}. Ci furono molte manifestazioni causate da questa crisi di
governo (studenti, borghesi, socialisti, \ldots). Il governo allora \textbf{lascia liberi gli
interventisti} e \textbf{asseconda i socialisti} per dimostrare che l'Italia voleva la guerra.\\
Vittorio Emanuele \rom{3} chiama Giolitti e lo informa sul patto di Londra. Giolitti, temendo una
crisi istituzionale della monarchia ed essendo comunque piemontese, abbandona Roma (ovvero rinuncia
a tenere l'Italia fuori dalla guerra). Richiama Salandra al Quirinale e gli conferisce poteri 
speciali (20 maggio) e finanziamenti per sostenere la guerra. \textbf{Il parlamento vota l'entrata in
guerra con il sostegno anche dei liberali giolittiani} (non dei socialisti). Il 
\textbf{24 maggio 1915} l'Italia entra in guerra contro l'Austria.\\
L'entrata in guerra è importante anche per la politica interna in quanto Salandra, Sonnino e il re 
sono riusciti a togliere il potere al parlamento e a dare al re il governo.

\subsection{Lo spirito del combattente}
Perché combatte un soldato?
\begin{description}
  \item[Per solidarietà] nei confronti dei compagni
  \item[Per rassegnazione] dopo il primo inverno e dopo l'abitudine
\end{description}
Nonostante le nobili intenzioni, i fenomeni di diserzione e ribellione non erano infrequenti. 
L'istinto di sopravvivenza e il sottrarsi alla morte avevano la meglio.\\
Al polo opposto stava un'idologia ``bellicista'', secondo la quale la guerra è la massima 
esaltazione ed espressione più alta dell'esperienza umana. Ernst Jünger riteneva che la guerra 
fosse un momento costitutivo di una nuova razza superiore alle precedenti. In Italia l'equivalente
di questi erano gli ``arditi'', capaci di rovesciare le regole tradizionali di combattimento (secondo
Giorgio Rochat).

\subsection{La partecipazione delle masse nella guerra}
Oltre alle innovazioni tecnologiche, il vero motore della guerra era la forza d'urto delle masse
di uomini mandati al fronte. Quelli che combattevano non erano altro che contadini, operai, impiegati
pubblici o privati che avevano alimentato l'intervento Italiano. La guerra dunque fu un modo per
\textbf{rafforzare lo spirito delle masse} che ora diventavano le vere protagoniste. Omogeneizzò
inoltre tutti gli strati sociali accomunandoli con il concetto di ``nazione''.\\
In Italia in particolar modo la guerra fu vista come un completamento del Risorgimento. Nei primi
50 anni dell'unità buona parte della massa non si sentiva unita, non si sentiva ``Italiana''.
Al fronte i ceti più bassi avevano scoperto l'ideale di nazione più grande del loro paese d'origine.
Il destino comune, la lingua comune, amalgamavano tutti gli strati sociali.

\subsection{Lo stallo del 1915--1916}
Con l'entrata in guerra dell'Italia si apre un nuovo fronte, meridionale. La guerra era lenta, così
detta di \textbf{logoramento} in trincea. Nessuno dei due fronti otteneva vere e proprie vittorie
e riusciva ad avanzare. Questo stallo danneggia prevalentemente gli \textbf{Imperi centrali} in 
quanto sono isolati dal resto del mondo e il blocco commerciale li danneggiava.\\
Per smuovere la situazione, gli imperi tedeschi concentrano le forze a \textbf{Verdun} dove la
battaglia si protrasse per oltre 5 mesi con più di mezzo milione di morti. I tedeschi falliscono
in questo tentativo.\\
Provano anche a sfondare via mare, con una battaglia contro gli Inglesi nello \textbf{Jutland},
vicino allo Skagerrak.\\
Tentarono quindi la guerra \textbf{sottomarina} per affondare le navi inglesi. Ottenne buoni 
risultati ma vide anche l'entrata in guerra degli Stati Uniti.\\ [\baselineskip]
Sul fronte meridionale i tedeschi fecero una \textbf{spedizione punitiva} contro l'Italia che
vide l'occupazione di Asiago come risultato. L'impreparazione dell'esercito Italiano portò Salandra
a dimettersi. Dopo una cruenta battaglia le truppe italiane furono capaci di \textbf{prendere 
Gorizia} (9 agosto 1916).\\ [\baselineskip]
Per fronteggiare il malcontento, molti governi formarono dei \textbf{governi di unità nazionale}, 
ovvero dei governi di grandi alleanze. Dopo le dimissioni di Salandra, in Italia si formò il governo
\textbf{Boselli}, in Francia \textbf{Briand} e in Gran Bretagna \textbf{David Lloyd George}. In
Germania tutto il potere fu concentrato nell'imperatore e nelle più alte gerarchie militari.\\
Questo accentramento porta i governi a pianificare e dirigere direttamente la guerra. Infatti ci fu
l'influsso sulle aziende di \textbf{innovare} e migliorare i propri prodotti. Queste modifiche
portarono anche alla \textbf{militarizzazione del lavoro in fabbrica} e alla limitazione delle
libertà sindacali. La guerra inoltre era costosa e questo portò all'introduzione di nuove tasse
coon conseguente aumento del debito pubblico e inflazione.

\subsection{La guerra ``Mondiale''}
Nel 1917 la \textbf{Russia} si ritira dal conflitto a causa della rivoluzione bolscevica. Così entra
una nova nazione: gli \textbf{Stati Uniti}. Lo zar a causa della rivoluzione che ne esce, è
costretto ad abdicare (rivolte di operai e soldati nella capitale) e si forma un governo repubblicano
provvisorio. \textbf{Karenskij} era a capo del governo e decise, dopo una sconfitta militare, di
uscire definitivamente dalla guerra. Gli Stati Uniti entrano in guerra principalmente a causa della
guerra sottomarina che stava flagellando gli inglesi.\\ [\baselineskip]
Se all'inizio la popolazione aveva preso di buon grado l'entrata in guerra, ora si diffondeva
malcontento, stanchezza e insofferenza. I soldati erano abbandonati a sè stessi, malnutriti e quasi
sommersi dalle trincee. L'utilizzo di nuove armi (bombe a mano, gas, lanciafiamme) e le pessime
condizioni, favorivono \textbf{diserzioni di massa} e ammutinamenti. Il ``disfattismo'' ormai dilgava
in tutte le fasce della gerarchia.\\
Per fermare questa ondata, i governi agiscono con battente propaganda e devono anche arginare il 
problema del ``fronte interno'', ovvero di tutti quegli strati sociali che per varie ragioni si
opponevano alla guerra. La scarsità di cibo e di beni era anche aumentata dai prezzi esorbitanti che
i proprietari delle aziende che avevano avuto il monopolio, fissavano.\\
In Francia si cambiò gli uomini al governo: \textbf{Pétain} è il nuovo generale e \textbf{Clemenceau}
è il nuovo primo ministro, determinato alla guerra.\\
In Germania si fece qualcosa di simile: si \textbf{militarizzarono le industrie} e il potere si
concentrò nel capo di stato maggiore \textbf{von Hindenburg}.\\ [\baselineskip]
Nel 1917, gli imperi centrali provano uno \textbf{sforzo offensivo eccezionale} nella speranza
di risolvere il conflitto in breve. Sferrarono un attacco nell'Isonzo. Il generale Italiano era
\textbf{Luigi Cadorna} che con il suo esercito non resistette all'urto. Fu la \textbf{Disfatta di
Caporetto} che portò una ritirata fino al Piave. Si forma così un \textbf{nuovo governo} e l'esercito
fu affidato a \textbf{Armando Diaz}.

\subsection{La disfatta di Caporetto}
\textbf{Il 24 ottobre 1917 un attacco austro-tedesco sfonda la linea italiana}. L'esercito italiano
fugge verso ovest con circa 300 mila prigionieri e altrettanti sbandati. I tedeschi sono fermati sul
Piave. 11 mesi dopo, con la battaglia di Vittorio Veneto si riconquista il Veneto e il Friuli.\\
\textbf{Cadorna sapeva ci sarebbe stato un attacco} eppure non ha fatto niente. Sul campo di 
battaglia in montagna, per ogni soldato c'era bisogno di 4 uomini. Dopo la sconfitta, Cadorna
\textbf{accusa} i soldati di essere stati vili ed aver abbandonato il campo di battaglia. \textbf{Un
generale non dovrebbe mai condannare i propri soldati}.\\
Le truppe italiane si sentivano ``distaccate'' dalla guerra, ecco perché molti disertavano. O almeno
così si diceva. In realtà \textbf{quasi nessuno disertò e fuggì dal campo di battaglia} tranne i
colonnelli e i generali. Tutti i reparti continuarono a combattere. Alcuni generali rimasero
al fronte e combatterono assieme ai soldati, molti però abbandonarono i propri reparti.\\
La prima guerra mondiale è stata una \textbf{guerra di popoli} in cui si voleva solo l'annientamento
del nemico. I reparti italiani avevano 600 cannoni, quelli tedeschi più di 1200 (li avevano nascosti
e sapeva dove fossero quelli italiani).\\
La leggenda dello ``sciopero militare'' di Caporetto nacque da un libro di \textbf{Alberti}. Infatti
circa 7000 prigionieri descrivevano le azioni eroiche del battaglione che continuava a combattere
nonostante tutto.

\subsection{La fine della guerra}
I tedeschi spinti dalla vittoria di Caporetto, sferrarono un attacco sul fronte occidentale vicino a
\textbf{San Quintino} in cui l'Intesa fu sfondata fino alla \textbf{Marna}. La battaglia però
riprese con l'uso di nuove tecnologie (cannoni tedeschi e arerei e carri armati inglesi).\\
A luglio con l'arrivo degli americani il fronte fu sfondato verso \textbf{Amiens} e cominciò 
l'avanzata anche a sud, in Italia fino a Vittorio Veneto. \textbf{Il 4 novembre 1918 fu firmato
l'armistizio tra Austria e Italia}. L'impero asburgico si stava disgregando e la Germania deve
arrendersi anche lei, l'11 novembre.\\ [\baselineskip]
Nel gennaio del 1919 a \textbf{Versailles} si ritrovano i paesi vincitori (Francia, Gran Pretagna,
Stati Uniti e Italia). C'erano due schieramenti diversi
\begin{description}
  \item[Europa] capitanata da \textbf{Clemenceau} voleva mantenere le tradizionali annessioni
    territoriali, incentrate sull'egemonia della Francia e della Gran Bretagna in Europa.
  \item[Stati Uniti] con \textbf{Wilson} voleva affermare il principio dell'autodeterminazione,
    ovvero i vincitori dovevano solo ridisegnare la mappa geo-politica dell'Europa.
\end{description}
L'Italia no riuscì ad ottenere l'annessione di Fiume e della Dalmazia conducendo ad un'insistita
campagna nazionalistica per la vittoria mutilata.\\ [\baselineskip]
Prevalse la linea dura di Clemenceau che \textbf{impose} il trattato di Versailles ai tedeschi.
Esso stabiliva
\begin{itemize}
  \item La restituzione alla Francia dell'\textbf{Alsazia e Lorena}, lo smembramento dei possedimenti
    coloniali e il pagamento dei danni di guerra
  \item La creazione di \textbf{Austria}, \textbf{Ungheria} e \textbf{Cecoslovacchia}, 
    \textbf{Iugoslavia}
  \item La \textbf{Polonia} fu ricostruita
  \item L'Italia ottenne l'\textbf{Alto Adige}, \textbf{Trieste} e l'\textbf{Istria} 
  \item Gli altri territori furono messi sotto il controllo Francese o Inglese
\end{itemize}
Infine, su volere di Wilson, fu istituita la \textbf{Società delle Nazioni} che aveva lo scopo di 
tutelare la pace facendo da arbitro nelle controversie internazionali. Questo progetto però
non andò molto lontano in quanto Russia, Germania e gli stessi Stati Uniti restarono fuori. Divenne
quindi un mezzo per Francia e Gran Bretagna per esercitare il potere.
