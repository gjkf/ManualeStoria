%!TEX ROOT=storia.tex

\section{La Prima Guerra Mondiale}
Come ogni fenomeno complesso, la guerra non ha avuto una sola causa. Forse nessuno dei fattori, presi
singolarmente, sarebbe bastato.\\
Gli storici marxisti sottolineavano le \textbf{cause economiche} (concorrenza industriale, 
protezionismo e guerre doganali, concorrenza coloniale).\\

\subsection{Bismarck e la Germania}
Bismarck � rimasto cancelliere fino al 1890. Fino ad allora non aveva fatto una politica coloniale
in quanto sarebbe entrato in conflitto con l'Inghilterra e doveva mantenere buoni rapporti con la
Russia (il suo principale obiettivo era isolare politicamente la Francia). Bismarck si dimette nel
1890.\\
Guglielmo \rom{2} voleva una politica pi� aggressiva, coloniale. Quindi minaccia gli Inglesi creando
una flotta che possa competere con la loro. \textbf{Francia e Inghilterra si accordano sulle
colonie}.\\
I tedeschi ottengono un appalto per la costruzione di una ferrovia da Istanbul a Baghdad. Favoriscono
cos� il commercio con l'impero Ottomano delle merci tedesche.\\
Il Marocco era diviso a met� tra Spagnoli e Ottomani. Sia la Francia che la Germania lo volevano.
\textbf{Due crisi Marocchine}: 1905--1906, vinta dall'alleanza Inghilterra-Francia e 1911. 
\textbf{Francia e Germania sono sull'orlo della guerra}.\\
La Germania aveva solo l'Austria come alleata ma era continentale, senza sbocchi sul mare. L'Italia
aveva accordi con la Francia. Ormai la guerra pareva come l'unica maniera per realizzare i piani
tedeschi.\\ [\baselineskip]
Negli anni '90 Francia e Russia fanno un'alleanza militare, cos� come anche Inghilterra e Russia.
Ci sono ora due schieramenti
\begin{description}
  \item[Triplice Alleanza] Germania, Austria, Italia
  \item[Triplice Intesa] Francia, Inghilterra, Russia (accordi bilaterali)
\end{description}

\subsection{L'inizio della Guerra}
Nei Balcani c'era un contrasto fra Austria e Russia. Molti stati ottengono l'indipendenza, tra cui
la Serbia (che conteneva anche Croazia e Slovenia). Si forma cos� la \textbf{Iugoslavia} ovvero lo
Stato degli Slavi del Sud. Alcune delle popolazioni erano sotto l'Austria per�. La Russia era alleata
della Serbia. La guerra era ormai scontata anche per i movimenti nazionalisti che si andavano
diffondendo che offrivano una visione della guerra come modo per dimostrare la forza.\\ 
[\baselineskip]
Il \textbf{21 giugno 1914} Gavrilo Princip assasina Franecsco Ferdinando per protesta dell'annessione
della Bosnia all'Austria. Lo fa con il sostegno dei servizi segreti Serbi. Il \textbf{21 luglio}
scoppia la guerra. Fra il 28 e il 4 agosto si attivano le alleanze: \textbf{Austria e Germania} 
contro \textbf{Russia, Francia, Inghilterra e Serbia}.

\subsubsection{La questione Italiana}
\textbf{Antonio Salandra} guida il governo in modo liberale, di Destra. Dichiara la 
\textbf{neutralit�} dicendo che l'alleanza era difensiva. La popolazione si divide in due: 
\textbf{Neutralisti} e \textbf{Interventisti}.\\
\textbf{Giolitti} voleva la neutralit� in quanto non sarebbe stata sostenibile un'altra guerra dopo
quella in Libia. Contrattando la neutralit� invece si sarebbe potuto ottenere molto. La Chiesa
condivideva. \textbf{La Chiesa} esprimeva i pensieri dei contadini: una guerra contro gli Austriaci,
cattolici, non era vista bene (Benedetto \rom{15}) era il nuovo papa. Anche i \textbf{socialisti} 
erano neutralisti in quanto si sarebbe intaccato l'internazionalismo.\\
\textbf{I Nazionalisti} invece erano interventisti, per dimostrare la propria forza, i 
\textbf{democratici} si ricollegavano a Mazzini e all'idea di un completamento del Risorgimento
italiano con l'annessione delle terre irredente. Anche i \textbf{sindacati rivoluzionari} erano 
favorevoli in quanto ritenevano che la guerra avrebbe scosso il capitalismo e fatto crollare, creando
i presupposti per una rivoluzione. Infine anche i \textbf{liberali conservatori}.\\ [\baselineskip]
\textbf{Dopo mesi, entriamo in guerra contro Austria e Germania}. Per convenienza. Il 26 aprile 1915
fu stipulato segretamente il \textbf{Patto di Londra} tra l'Italia e l'Intesa. Entro un mese l'Italia
sarebbe dovuta entrare in guerra contro l'Austria, in cambio avrebbe ricevuto
\begin{enumerate}
  \item Le terre irredente
  \item L'Alto-Adige
  \item L'Istria
  \item La Dalmazia e il porto di Valona
  \item Il controllo della politica estera dell'Albania
  \item Il Dodecaneso
  \item Un bacino di carbone in Turchia
  \item Alcune colonie tedesche in Africa
\end{enumerate}
L'Italia ora doveva entrare in guerra ma i neutralisti erano in maggioranza in parlamento e tra il
popolo. \textbf{Salandra si dimette}. Ci furono molte manifestazioni causate da questa crisi di
governo (studenti, borghesi, socialisti, \ldots). Il governo allora \textbf{lascia liberi gli
interventisti} e \textbf{asseconda i socialisti} per dimostrare che l'Italia voleva la guerra.\\
Vittorio Emanuele \rom{3} chiama Giolitti e lo informa sul patto di Londra. Giolitti, temendo una
crisi istituzionale della monarchia ed essendo comunque piemontese, abbandona Roma (ovvero rinuncia
a tenere l'Italia fuori dalla guerra). Richiama Salandra al Quirinale e gli conferisce poteri 
speciali (20 maggio) e finanziamenti per sostenere la guerra. \textbf{Il parlamento vota l'entrata in
guerra con il sostegno anche dei liberali giolittiani} (non dei socialisti). Il 
\textbf{24 maggio 1915} l'Italia entra in guerra contro l'Austria.\\
L'entrata in guerra � importante anche per la politica interna in quanto Salandra, Sonnino e il re 
sono riusciti a togliere il potere al parlamento e a dare al re il governo.
