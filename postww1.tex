%!TEX ROOT=storia.tex

\section{Dopo guerra}
In totale ci furono circa 65 milioni di soldati coinvolti, 10 milioni sono morti al fronte, 20 a 
causa dell'influenza spagnola. Ci furono inoltre milioni e milioni di feriti ed invalidati. Per 
sostenere i veterani del fronte si creano \textbf{pensioni} di guerra, d'invalidità, \dots

\subsection{Economia e società}
In campo economico la guerra fu distruttiva. Oltre alle enormi spese di ricostruzione, gli stati 
(tranne gli USA) uscirono \textbf{indebitati}, gli USA avevano da soli metà delle riserve auree del 
pianeta. I prezzi di conseguenza erano regolati dallo stato, anche se la regolazione era considerata
solamente provvisoria. Nel 1919 vengono eliminate le leggi e si torna al \textbf{libero mercato} e 
questo provoca un'enorme inflazione e il conseguente aumento dei prezzi. Vengono danneggiati quelli
con lo stipendio fisso.\\
È necessario inoltre ristabilire la produzione industriale e tornare a trasformarla in quella 
originale dopo la guerra. Le grandi aziende avevano guadagnato molto e ora devono riconvertirsi,
però ciò richiede tempo e \textbf{molti operai furono licenziati}. Questo genera malcontento e i 
sindacati e i partiti socialisti vedono un boom di iscritti. Tra il 1919 e il 1920 ci furono molti
scioperi e perciò venne definito il \textbf{biennio rosso}, sulla spinta socialista.\\
Ai contadini era stata promessa la terra, una riforma agraria che però non fu mai attuata. Infatti
durante il biennio roso molti contadini occupavano da sè la terra tentando una propria rivoluzione.\\
La grande borghesia (industriali) era molto più ricca di prima, la media-piccola borghesia invece no.
Molti erano uffciali in guerra, si erano abituati al potere, vedevano la guerra come \textit{``igiene
del mondo''}. Tornavano dalla guerra in una vita tra difficoltà economiche e una vita anonima. Da
una parte vedevano gli industriali, ricchi, dall'altra i contadini, poveri. Loro erano in mezzo e
da questo ambiente scaturiranno i \textbf{movimenti di estrema destra}. Il piccolo borghese dal
suo canto non voleva uniformarsi al proletario. Questo è il fenomeno del \textbf{reducismo}, il
sentimento di delusione dei reduci dalla guerra che avendo combattuto per la patria si aspettavano
qualcosa di diverso dalla guerra.\\
La guerra aveva quasi ``normalizzato'' la violenza e questo lo si vede anche in politica dove gli
scontri non sono più solo verbali.

\subsection{Politica}
La prima guerra mondiale a avuto il pregio di aver contribuito all'emancipazione femminile. Infatti
quando il marito era al fronte erano le donne a gestire la casa. Molte donne inoltre furono assunte
nelle industrie durante la guerra e prendendo anche dei posti di comando. Molti paesi, dopo il
conflitto daranno diritto di voto alle donne.

\subsection{Repubblica di Weimar}
A novembre del 1918 la Germania non è più capace di sostenere la guerra, non ha più risorse da 
spendere e il malcontento cresce (si vengono a formare dei consigli di fabbrica simili ai soviet)
sostenuto dalla ``Lega di Spartaco'' (partito comunista che aveva a capo Rosa Luxemburg, una donna,
ebrea, colta). Sempre nel novembre del 1918 alcuni marinai si ammutinano e l'imperatore è costretto
a fuggire in Olanda. Ci sono quindi due forze al potere: il \textbf{PSD} con a capo Ebert e 
l'esercito. Il governo era in mano ad Ebert e come in Russia si definiva ``Consiglio dei commissari 
del popolo''. Voleva ricordare la Russia ma non voleva la rivoluzione, voleva un regime democratico 
ma con quel nome si portava vicino gli operai. L'esercito (guidato da Hindemburg e Ludendorf) accetta
il governo anche se erano contrari al PSD in quanto il governo era il male minore. A gennaio il
partito comunista tenta una rivoluzione ma fallisce, vengono allora creati i \textbf{Freicorps} in
comune accordo tra PSD ed esercito. Erano formati da ex arditi, erano nazionalisti opposti alla
rivoluzione. Rosa Lindemburg viene uccisa.

\subsubsection{Costituzione}
La costituzione che viene creata è \textbf{democratica ed egalitaria}. Formava una repubblica 
federale con 17 Länder che avevano molta autonomia. Era una repubblica parlamentare, il presidente 
era eletto a suffragio universale direttamente, rimaneva in carica 7 anni. Il presidente sceglieva
il cancelliere ed il governo che doveva avere la fiducia del parlamento. \textbf{Ebert è il primo
presidente}. Fino al 1933 il governo sarà conteso tra PSD e CDU, quando il nazismo prenderà il 
potere.\\
Come può uno stato con una costituzione del genere eleggere il nazismo? La repubblica non era forte,
la borghesia aveva nostalgia dell'impero. Inoltre l'Art.\ 48 dice che in caso di pericolo il 
presidente può sospendere le libertà e i diritti del popolo. Con il nazismo sarà sempre un caso di
pericolo. Molti cittadini vedevano la repubblica come un ripiego, meglio del socialismo ma non il
meglio che si potesse avere. Non tutti i partiti sostenevano questo governo (quelli di estrema destra
e sinistra volevano abbatterlo). Quelli di \textbf{estrema destra} erano antidemocratici, razzisti. 
La repubblica secondo loro nasce dalla sconfitta e a guidarla è chi ha portato a questa sconfitta
(è una pugnalata alle spalle da parte del PSD, della CDU e degli ebrei). \textbf{L'estrema sinistra}
era comunista, la repubblica è un governo borghese.

\subsubsection{Origine del partito Nazista}
Hitler nasce in Austria nel 1889 in una famiglia piccolo borhese. Dopo gli studi va a Vienna dove
voleva iscriversi all'Accademia delle Belle Arti. Tira avanti facendo lavoretti. Era molto 
interessato alla politica ma non vi partecipava attivamente. Impara le dinamiche della società di
massa. Nel 1913 si trasferisce a Monaco di Baviera. Avrebbe dovuto andare a combattere per l'esercito
Austro-Ungarico ma voleva combattere per l'esercito tedesco in Francia, diventa caporale e viene
smobilitato dall'esercito alla fine. Continuerà a collaborare come informatore: si doveva infiltrare
nelle manifestazioni politiche ed informare i sueriori. Così Hitler entra in contatto con il
\textbf{partito dei Lavoratori Tedeschi} di Drexler (estrema destra). Abbandona l'esercito e milita
per questo partito. Tra il 1919 e il 1920 diventa leader e nel 1920 il partito diventa 
\textbf{partito Nazional-socialista dei Lavoratori Tedeschi}. Era ancora un piccolo partito.\\
[\baselineskip]
Il programma prevedeva
\begin{description}
  \item[Riunione dei Tedeschi in un'unica Germania] Pangermanesimo, andava contro i cattolici
  \item[Abolizione ufficiale dei trattati internazionali]
  \item[Terra e suolo per le popolazioni in eccedenza] ad esempio gli slavi
  \item[Cittadino dello stato solo chi è di sangue tedesco] senza problemi di religione, 
    \textbf{no ebrei}, non per la religione o la cultura, ma per il sangue
  \item[Tutti i non-cittadini sono stranieri] e quindi ospiti
  \item[I diritti politici li hanno solo i cittadini]
  \item[Anti-parlamentarismo] 
  \item[Lo stato deve assicurare lavoro e assistenza ai cittadini] Se non si arriva a sostenere, si
    espellano gli ospiti
  \item[Espulsione degli immigrati dopo il 2 agosto 1914]
  \item[Dovere è produrre per il bene di tutti] Abolizione dei redditi di chi non fattura, confisca
    integrale dei profitti di guerra
  \item[Statalizzazione dei trust] Gli operai partecipano alla ridistribuzione degli utili
  \item[Conservata la classe media]
  \item[Comunizzati i grandi magazzini] Si affittano ai privati, si avvicina al socialismo e cerca il
    consenso presso gli operai, ciò che conta è la Germania, si avvicina al nazionalismo
  \item[La scuola insegna il nazionalismo]
  \item[Lotta legale contro le menzogne politica] Controllo della stampa
  \item[I giornalisti sono solo tedeschi]
  \item[Libertà di religione finché non si danneggia la razza germanica] No ebrei
  \item[Forte potere centrale nel Reich] Quasi dittatoriale
\end{description}

\subsubsection{Le riparazioni}
Nel 1921 arrivarono le spese di riparazione di guerra. Ammontavano a circa 132 miliardi di marchi in
42 anni (era il 25\% del PIL tedesco). I governi del 1921 e del 1922 pagarono senza aumentare le 
tasse vendendo le riserve auree e aumentando carta moneta. Questo però portò ad una forte inflazione.
Nel 1923 non riuscendo a pagare la Francia e il Belgio, i loro eserciti occupano la Ruhr, una zona
ricca di trust, miniere e fabbriche per far lavorare le industrie per la Francia. La Germania non
potè opporsi. Il governo della CDU chiese ai cittadini di non lavorare per i francesi, il governo
avrebbe pagato loro lo stipendio. Nel 1929 ci fu un'altra enorme inflazione e il marco crolla.\\
I nazionalisti tentano l'insurrezione, i socialisti la rivoluzione. Anche i nazisti ci provano
a novembre a Putch e a Monaco. Ludendorf e Hitler volevano prendere il potere in Baviera e poi a 
Berlino, sul modello di Mussolini. Ludendorf fu libero, Hitler invece venne imprigionato e processato
nel 1924. Hitler sfruttò il processo per farsi conoscere. Passò un anno in carcere, dove scrive il
``Mein Kampf'' che contiene il programma del 1920 e il rifiuto del parlamentarismo con un'aggiunta 
riguardo allo spazio vitale.\\ [\baselineskip]
Nella seconda metà del 1923 il governo fu di Gro{\ss}e Koalition (CDU+PSD). \textbf{Streussman} guida
il governo, era un liberale. La Germania doveva prendere accordi internazionali con la mediazione
degli USA, vengono così allargate le rate dei danni di guerra. Gli Stati Uniti temevano la
sovraproduzione quindi \textbf{le banche americane si impegnarono ad investire in Germania}. Avevano 
interessi che l'economia tedesca si riprendesse per avere un mercato. Venne creato un nuovo marco
garantito dal suolo tedesco, non da riserve auree. \textbf{Dal 1924 al 1929 l'economia si riprende}.
Pagano le sanzioni e in generale l'età di Streussman è un'epoca di stabilità e sviluppo economico.
Infatti voleva inserire la Germania in un piano di parità con gli altri stati così che firma gli
\textbf{Accordi di Locarno} riaffermando i confini (la Germania perde ufficialmente l'Alsazia e la 
Lorena).\\ [\baselineskip]
Dopo il processo il partito Nazista e Hitler sono molto conosciuti. Fino al 1929 il suo partito
prenderà sempre attorno al 2\%, principalmente al sud. Aveva formato le Squadre di Assalto di 
partito, combattenti contro gli avversari politici. Le due più illustri vittime furono Erzberger 
(CDU, aveva firmato l'armistizio) e Ruthermau (liberale, organizzò l'economia nella guerra, diventato
ministro degli esteri voleva pagari le riparazioni di guerra).

\subsection{Italia}
Anche se l'Italia ufficialmente aveva vinto la guerra, rimaneva devastata. Con un debito pubblico
enorme, inflazione e malcontento.\\
Fino al 1922 i liberali rimasero al potere ma si erano indeboliti sempre di più, non erano stati
capaci di gestire la crisi.\\
Il PSI aumenta notevolmente il numero di iscritti, però era diviso in due correnti
\begin{description}
  \item[Massimalisti] Rivoluzionari. \textbf{Serrati} era il capo e anche dirigente di partito. Si
    aspettavano come imminente il crollo del capitalismo ma volevano che la rivoluzione arrivasse da
    sè, non fare come Lenin. Non aderirono alla Terza Internazionale. Il PSI secondo loro non poteva
    fare accordi con altri partiti in quanto borghesi
  \item[Riformisti] \textbf{Turati} era a capo. La rivoluzione non era così vicina, erano disposti a
    fare alleanze se necessario
  \item[Ordine nuovo] Una rivista torinese. \textbf{Gramsci} era l'esponente e direttore, 
    \textbf{Togliatti} collaborava. L'URSS era il modello da seguire, i consigli di fabbrica, 
    democrazia dal basso. Erano un piccolo gruppo di giovani che nel 1921 fonderanno il PCI
\end{description}
Nel \textbf{1919} nasce il \textbf{Partito popolare Italiano}, il primo partito cattolico. 
\textbf{Sturzo} è a capo. Benedetto \rom{15} era meno conservatore di Pio \rom{10} e la Chiesa era
preoccupata che i socialisti potessero prendere il sopravvento. Sturzo inoltre voleva avere una
propria rappresentnza politica. \textbf{Diventa subito partito di massa}. Sturzo presenta il PPI
come \textit{a-confessionale}, ovvero che non era necesasrio essere cattolici per aderire. Il 
programma conteneva elementi chiave della religione cristiana:
\begin{itemize}
  \item No divorzio
  \item Libertà d'azione per la Chiesa
  \item Libertà per scuole private
  \item Suffragio universale anche femminile
  \item Legge elettorale proporzionale
  \item Introdurre le regioni (decentramento del potere e maggiori autonomie locali)
  \item Riforma agraria (terre incolte espropriate con indennizzo e date alle singole famiglie)
\end{itemize}
Si presentava come non conservatore. È un partito interclassista che si interessava pricipalmente 
alle classi più deboli. La nascita del PPI indebolisce ancora di più i liberali.

\subsubsection{Trattati ed elezioni del 1919}
Nel 1919 Orlando era primo ministro e Sonnino era ministro degli esteri. Trattano in Francia e
\textbf{chiedono che venga rispettato integralmente il Patto di Londa} ma anche che, per il principio
di autodeterminazione dei popoli, \textbf{Fiume venga annessa}. Si contraddicono in questo modo.
Non ottengono né Fiume né la Dalmazia che finisce alla Iugoslavia. Tornano a Roma come gesto di 
sdegno con manifestazioni nazionaliste che insorgono. I trattati vanno avanti senza l'Italia.
L'Italia firma quello che gli altri avevano scelto (no Fiume, no Dalmazia e no colonie). Fiume era
una città libera e \textbf{Orlando si dimette} nel giugno del 1919. \textbf{Nitti} prende il suo
posto (liberale democratico).\\
A novembre si tengono nuove elezioni con una legge proporzionale:
\begin{description}
  \item[Liberali alla maggioranza relativa] circa 200 deputati
  \item[PSI] circa 150 deputati
  \item[PPI] circa 100 deputati
  \item[FIC] solo qualche voto a Milano
\end{description}
Liberali e PPI governeranno ma Sturzo non voleva fare la stampella e quindi il governo sarà molto 
instabile.

\subsubsection{Impresa di Fiume}
D'Annunzio ed alcuni nazionalisti partono da Ronchi e raggiungono Fiume che era sotto la società 
delle nazioni. Fiume cade nelle mani di D'Annunzio. Questa è una \textbf{violazione degli accordi
internazionali}. Molti dei seguaci erano militari che ora sono quasi disertori. Questo avvenimento
scredita ancora di più il governo italiano. \textbf{Badoglio} era il generale in Friuli e avrebbe
dovuto ordinare ai soldati di cacciare D'Annunzio ma non era sicuro l'avrebbero ascoltato. L'Italia
non è capace di governare.\\
L'impresa di Fiume suscitava molta simpatia tra gli aderenti ai FIC ma Mussolini temeva che 
D'Annunzio prendesse troppo successo. A questo punto \textbf{Mussolini capisce che la direzione da
seguire è quella nazionalista}. A luglio del 1920 a Trieste c'è una manifestazione a sostegno di
D'Annunzio (a Trieste c'erano anche i primi non-milanesi iscritti ai FCI). Un corteo parte da Piazza
Unità fino all'Hotel \textbf{Balkan} (simbolo della popolazione slava). \textbf{Viene incendiato} e
le forze dell'ordine collaborano. Questo è il primo esempio di \textbf{squadrismo}.\\
Il governo di Nitti era sia colpito dall'impresa di Fiume che dai sindacati. Nitti si dimette nel
giugno del 1920 e \textbf{Giolitti sale al potere}. Mette fine all'impresa di Fiume facendo accordi
con l'Inghilterra (\textbf{accordi di Rapallo}). \textbf{Fiume è una città libera}. Giolitti intimò
con poche cannonate di andarsene a D'Annunzio.

\subsubsection{L'occupazione delle fabbriche}
Tra agosto e settembre del 1920 la FIOM (facente parte della CGIL) e Confindustria discuterono sugli
orari di lavoro. La FIOM indisse quindi uno \textbf{sciopero bianco}, ovvero lavorare in modo da
rallentare la produzione. La Confindustria di rimando indisse la \textbf{serrata} ovvero la chiusura
delle fabbriche. A Milano, la Romeo era occpuata dagli operai. Successivamente altre fabbriche 
seguirono fino ad arrivare a 500 mila operai che cercano di mandare avanti il lavoro, dimostrando
forza e autonomia. \textbf{I consigli di fabbrica gestiscono gli operai}.\\
I socialisti \textbf{riformisti} non credevano ci fossero le condizioni per una rivoluzione, 
bisognava continuare l'occpuazione solo per trattare. I \textbf{massimalisti} non credevano ci
fossero le condizione per la rivoluzione e non era questo il modo di cominciarla. Quelli 
dell'\textbf{Ordine nuovo} volevano invece la rivoluzione in quanto il PSI deve prendere il potere
politico dalle fabbriche.\\
Confindustria voleva che il governo prendesse provvedimenti, anche con la forza.\\
Giolitti non voleva usare l'esercito, sarebbe stato l'inizio della rivoluzione, voleva far passare il
tempo favorendo un accordo sindacati-Confindustria. Dopo un mese FIOM e Confindustria firmano un 
accordo molto favorevole per gli operai: 8 ore di lavoro per tutti, aumenti salariali e diritti di 
ferie retribuite.\\
Questo evento rese ancora più visibili le correnti nel PSI tanto che a gennaio del 1921 c'è la 
scissione. A Livorno c'era il congresso che discuteva se aderire o meno alla Terza Internazionale
che implicava avere l'URSS come modello, cambiare nome di partito ed espellere i riformisti. Né
i massimalisti né i riformisti volevano aderire, solo Gramsci. \textbf{A gennaio del 1921} nasce
il \textbf{PCI} che aderisce alla Terza Internazionale.\\
I liberali vedevano l'occpuazione delle fabbriche come debolezza e la borghesia era impaurita dalla
rivoluzione. Alcuni membri del movimento liberale si avvicinano al fascismo. Infatti Mussolini sarà
bravo ad approfittarne. Promosse lo squadrismo anti socialista: nel 1920 ci sono le elezioni comunali
e il PSI vince in molte città. Gli squadristi manifestano contro di essi. Tra il 1921 e il 1922 ci 
sono oltre 1000 morti a casua dei fenomeni di squadrismo. Quindi il fascismo si diffonde e si
allarga fino ad esseere finnanziato dai propietari terrieri e gli industriali. Questo è il
\textbf{fascismo organico} ovvero dalel città si diffonde. Alla fine del 1921 avrà più iscritti del
Partito Socialista.


