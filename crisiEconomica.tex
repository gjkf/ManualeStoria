%!TEX ROOT=storia.tex

\section{La grande crisi e il New Deal}
Ad oltre 50 anni dalla grande depressione di fine 1800, negli anni Trenta si verificò un'altra crisi
di portata mondiale.

\subsection{Il giovedì nero, le cause della crisi}
Il \textbf{24 ottobre 1929} esplode la crisi economica con il crollo della Borsa di New York. Wall
Street era diventata la banca principale dove venivano effettuati i principali movimenti speculativi.
Lo scambio di titoli azionari era fatto senza controlli e questa libertà aveva portato a praticare
attività molto rischiose come \textbf{l'acquisto delle azioni a credito}. Questo ha provocato 
l'aumento dell'\textit{economia di carta} che è sempre più slegata da quella reale. Il sistema
funzionava nel seguente modo
\begin{itemize}
  \item Il piccolo risparmiatore chiedeva un prestito al mediatore di Borsa, per farlo depositava
    un margine (titoli in garanzia) pari al 30\%-50\% del prestito
  \item Il mediatore contraeva prestiti a breve termine da altri istituti
  \item Il risparmiatore contava di vendere le azioni ad un prezzo tale da ripagare i debiti
\end{itemize}
Questo funzionò fino al 1929. Per arginare il problema, la banca centrale americana aveva aumentato
il tasso d'interesse nei rapporti con altre banche con il fine di scoraggiare operazioni di credito
con altre banche. Ma già nel marzo 1929 la speculazione riprese e soltanto ad agosto alzò il tasso
al 6\%, una misura troppo tardiva.\\
Questa crisi finanziaria ebbe ripercussioni su tutta la società dato che erano nate aziende in ogni
settore sull'onda della speculazione. Ora queste aziende non riuscivano più a sostenersi e chiudevano
licenziando migliai di lavoratori. All'inizio del 1931 i disoccupati erano 8 milioni, dopo un anno 
13.\\
La \textbf{causa strutturale} del grande crollo fu l'eccesso di capacità produttiva. La diffusione
del taylorismo fu un altro forte motivo che produsse il crollo in quanto la sua espansione prevedeva
una riorganizzazione dei sistemi di produzione. La suddivisione del lavoro aumentava la quantità
di prodotti ma mancavano i mercati a sostenerli. Infatti, anche con l'aumento dell'export, i mercati
non riuscivano ad assorbire tutto il prodotto. Verso la metà degli anni Venti però l'Europa tornava
ad essere una grande produttrice. Quindi da una sola grande nazione, l'America, che gestiva tutto
il mercato mondiale, si è giunti ad un sistema policentrico che comportò un enorme produzione di
eccedenze.\\
Dato che la produzione di scala aveva aumentato il numero di prodotti, doveva aumentare il potere
d'acquisto dei cittadini e per fare ciò si diffusero le \textbf{agevolazioni creditizie}. Ma 
nonostante l'aumento generale dei redditi, non si riusciva ad assorbire quella quantità di merci
prodotte. Questo provocò la riduzione della produzione e il conseguente licenziamento.\\
Questa crisi, dagli Stati Uniti, si diffuse in tutto il mondo. Principalmente per gli stretti
rapporti che intercorrevano fra i vari stati, anche in Occidente si diffusero sovraproduzione e
stagnamento.

\subsection{Gli effetti della crisi}
La crisi ebbe l'effetto di far \textbf{sostenere i prezzi} e \textbf{abbassae la produzione}.
Questo comportò una forte disoccupazione. Ciò fu possibile perché ormai i prezzi non erano regolati
da rapporti di domanda e offerta, ma erano imposti dai grandi trust che, pur di tutelare i profitti,
decisero di ridurre la produzione e sotenere i prezzi. Gli stati risposero con il 
\textbf{protezionismo} che tutelava il mercato interno a discapito di quello estero. Gli Stati Uniti
con \textbf{Hoover} furono i primi ad adottare questa politica. Gli altri stati si adeguarono e 
quindi si ottennero tanti piccoli mercati nazionali.\\
Gran Bretagna e Germania hanno sofferto di più, quest'ultima fu resa di fatto dipendente dagli 
investitori statuinitensi. Così che quando furono ritirati a causa della crisi, la Germania sprofondò
ancora di più.\\
Oltre ai soliti dazi doganali, gli stati adottarono anche misure più razionali come accordi 
bilaterali. Questa crisi comportò anche una scossa del fondamento dell'economia monetaria del tempo:
\textbf{il valore della moneta non è più dipendente dal valore aureo} (evento chiave è il governo 
inglese nel 1931 decise di rendere inconvertibile la sterlina). Questo significava che dopo 
l'Inghilterra, gli altri paesi aumentarono le svalutazioni per non perdere la competitività in quanto
non si poteva più riscuotere il valore in oro della sterlina. Quest'economia produsse una 
\textbf{politica di potenza} tra i vari stati.

\subsection{Roosvelt e il New Deal}
Roosvelt venne eletto presidente nel 1932. La sua campagna si fondava su due principi fondamentali
\begin{description}
  \item[Rilancio dell'economia] sostenendo il mercato, rimuovendo la miseria ed aiutando la società
  \item[Mettere sotto controllo il sistema bancario] per impedire le grandi speculazioni di borsa
\end{description}
\textbf{Lo stato quindi interveniva nella vita economica} e ciò era qualcosa di nuovo, mai accaduto
prima. L'intervento economico americano era profondamente democratico in quanto si fondava sulla
redistribuzione del reddito.\\
Nel 1933, con \textbf{Emergence Banking Act} la Federal Reserve viene rafforzata e sulle banche,
holding e la Borsa vengono messe sotto più rigidi controlli. Venne introdotta una garanzia
statale sui piccoli depositi. Favorì la ripartizione delle quote di mercato limitando la
concorrenza sleale. Creò la \textbf{Work Progress Administration} per aprire cantieri pubblici al
fine di riassorbire la disoccupazione.\\
Le grandi \textit{corporations} come la General Motors erano contrarie e la stessa corte suprema 
dichiarò incostituzionale la manovra di Roosvelt. Approvò inoltre la \textbf{legge Wagner} nel 1935
che riconobbe pienamente i diritti sindacali dei lavoratori.\\
Il \textbf{Social Security Act} del 1935 fondò le basi dello stato sociale che per la prima volta
proteggeva il lavoratore con assicurazioni e sussidi. Infine si mise in atto una tassazione 
progressiva.

\subsection{Keynesismo}
John Maynard Keynes fu uno dei più grandi economisti del 1900. Modificò il liberismo tradizionale.
Nel suo ``Teoria generale dell'occupazione, dell'interesse e della moneta'' del 1936 \textbf{rifiuta
che il mercato venga lasciato libero} di raggiungere l'equlibrio spontaneamente e che \textbf{le 
risorse economiche vengano usate integralmente}. Nella crisi del 1929 il liberismo non funziona,
lo stato deve intervenire tramite lavori pubblici (commesse, profitti per le imprese, \ldots). Questo
però ha dei risvolti negativi
\begin{description}
  \item[Aumento del debito pubblico] Se è utile a far rinascere l'economia però è accettabile
  \item[Inflazione] Aumenta il denaro e aumenta la domanda
\end{description}
È importante che lo stato \textbf{raccolga le imposte} per gli investimenti pubblici. La 
\textbf{tassazione} deve essere \textbf{progressiva} (se è povero, il risparmio diventa domanda,
se è ricco diventa ancora più ricchezza con la flat-tax). Era inoltre favorevole a sussidi.\\
Gli anti-keynesiani lo criticano come socialista (per l'intervento dello stato). In realtà
``voleva salvare il capitalismo dai capitalisti'' senza perdere la democrazia.

