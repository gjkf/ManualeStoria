%!TEX ROOT=storia.tex
\section{Fascismo}
\subsection{Nascita del movimento fascista}
Benito Mussolini nasce da una famiglia rivoluzionaria. A 21 anni è socialista rivoluzionario e scappa
in Svizzera per evitare la leva militare. Torna in Italia e milita con il PSI attivamente. Nel 1912 
c'è il congresso in cui Mussolini propone di espellere tutti i leader riformisti come 
\textbf{Ivano Bonomi} in quanto non avevano criticato abbastanza duramente l'operazione in Libia e 
avevano dato fiducia la re. Bonomi e gli altri vengono espulsi e Mussolini diventa direttore 
dell'\textit{Avanti}.\\
All'inizio della guerra si schiera contro ma fra settembre ed ottobre scrive che se rimangono
neutralisti verranno isolati e arriva a scrivere a favore dell'intervento. La direzione del partito
licenzia Mussolini dalla direzione del giornale e le espelle. Mussolini allora fonda il 
\textbf{Popolo d'Italia}, il suo giornale finanziato dai Perrone e dai servizi segreti francesi 
(avevano interesse ci fosse un giornale interventista). È un \textbf{giornale socialista 
interventista}.\\
Mussolini va in guerra, combatte e viene ferito, così poi dimesso dall'esercito. Con il passare del
tempo, sopratutto dopo Caporetto il socialismo di Mussolini diventa debole e le sue posizioni si 
fanno sempre più \textbf{nazionaliste}. Finita la guerra, mancano i finanziatori così dà vita al
suo movimento. Il 23 marzo 1919 nascono i \textbf{Fasci Italiani di Combattimento} a Milano, in 
piazza San Sepolcro. Il programma prevedeva
\begin{description}
  \item[Movimento nazionalista rivoluzionario] Anti-dogmatico, anti-pregiudiziale, guerra al di sopra
    di tutto e tutti
  \item[Le idee come mezzo per l'azione politica]
  \item[Suffragio universale]
  \item[Legge elettorale proporzionale]
  \item[Maggiore età a 18 anni]
  \item[Abolizione del Senato]
  \item[Assemblea nazionale] per fare la costituzione e scegliere la forma di governo
  \item[8 ore di lavoro] Socialista
  \item[Minimo di paga]
  \item[I rappresentati dei lavoratori partecipavano ad organizzare le fabbriche]
  \item[I rappresentati dei lavoratori gestiscono industri e servizi]
  \item[Nazionalizzazione delle fabbriche di armi]
  \item[Nazionalismo] in politica estera
  \item[Espropriazione della ricchezza] Imposte progressive
  \item[Sequestro dei beni delle congregazioni religiose]
\end{description}
Si diffonde princpalmente tra gli ex combattenti ed ex arditi.\\ [\baselineskip]
Il 15 aprile c'è uno sciopero a cui i FIC si oppongono e assaltano la direzione dell'Avanti, 
devastandola.

\subsection{Mussolini acquista potere: la nascita del partito}
Le autorità facevano finta di non vedere i fenomeni di squadrismo, erano complici in quanto non 
avevano simpatie per i socialisti. Giolitti ordinava venisse rispettata la legge ma spesso i
prefetti lasciavano correre. \textbf{Giolitti era in difficoltà}.\\
Nel 1921 il \textbf{PPI toglie l'appoggio} al governo in quanto quest'ultimo aveva firmato una legge
a favore della nominalità dei titoli azionari (poter sapere chi compra e vende implica la possibilità
di tassare). Il \textbf{5 novembre 1921} ci sono nuove elezioni a suffragio universale con legge
proporzionale. Dato che non c'è un partito liberale di massa, \textbf{Giolitti propone un'alleza a
Mussolini} (dei Blocchi Nazionali, spera che i Fascisti al potere torneranno nella legalità). Con
questo accordo, i liberali hanno la maggioranza relativa, il PSI perde voti e il PPI ne guadagna.
\textbf{I FCI hanno 35 deputati}. Mussolini dichiara subito che \textbf{non avrebbe sostenuto 
Giolitti}. Mussolini tra maggio 1921 ottobre 1922 continuerà ad usare le squadre d'assalto. Giolitti
non ha più la maggioranza, \textbf{nuovo governo a Bonomi} con l'obiettivo di riportare l'ordine.
Favorsice il \textbf{Patto di Pacificazione} ad agosto del 1921. È un accordo sindacati 
socialisti-fascisti per mettere al bando le violenze (questo provoca malcontento dei Ras (capi 
locali) fascisti, tra cui Farinacci). Mussolini accetta perché altrimenti avrebbe dimostrato di 
essere loro la causa dei disordini inoltre a Sarzana (Liguria) i Carabinieri si erano opposti
ai fascisti (alcuni Ras credono che Mussolini si stia ``imborghesendo'').\\ [\baselineskip]
Nel novembre del 1921 c'è il \textbf{Congresso fascista} che trasforma il movimento in 
\textbf{Partito Nazionale Fascista} e rinuncia al patto di Pacificazione (dando la colpa al PSI).
Questo cambiamento porta ad una centralizzazione del potere nelle mani di Mussolini e meno potere
ai Ras.

\subsection{La marcia su Roma e le divisioni socialiste}
A febbraio del 1922, \textbf{Bonomi si dimette e sale Facta} che ad agosto si dimette ma gli viene
riconferito l'incarico.\\
Nel 1922 lo squadrismo si fa più duro andando ad attaccare anche le istituzioni di città imporanti.
A Milano costringono alle dimissioni il sindaco, danno potere al prefetto. Ad inzio agosto, la CGIL
indice uno \textbf{sciopero legislativo} per far rispettare la legge ma fallisce.\\ [\baselineskip]
Mussolini si prepara alla presa di potere:
\begin{description}
  \item[Cambia posizioni sulla monarchia] Originariamente era contrario ma per non avere contro
    l'esercito deve aprirsi alla monarchia
  \item[Toglie ogni riferimento socialista] Sostiene il capitalismo e fa incontro con Confindustria
  \item[Elimina l'anticlericalismo] Favorito da Pio \rom{11} (conservatore)
\end{description}
Ad inizio ottobre il PSI si scinde un'altra volta
\begin{description}
  \item[Riformisti] Arginare il PNF con accordi con i liberali o il PPI
  \item[Massimalisti] Contrari ad accordi
\end{description}
I riformisti escono dal partito e fondano il \textbf{Partito Socialista Unitario} con Matteotti 
segretario.\\[\baselineskip]
Ormai si parlava già di una marcia su Roma e Mussolini in un discorso a Napoli disse ``Se non ci 
daranno il potere, ce lo prenderemo calando su Roma'' (24 ottobre). Mussolini organizza a Milano la
marcia. \textbf{Il 27 ottobre 1922} inizia la Marcia su Roma. Alcune squadre prendono il potere
in alcune città attaccando luoghi di comunicazione, caserme, \ldots senza quasi resistenza. Dove c'è
non arrivano a prendere il potere. Erano circa 40 mila squadristi.\\
Facta vuole dichiarare lo stato d'assedio ed informa il re. Inizialmente era contrario ma poi
cambia idea: \textbf{Vittorio Emanuele \rom{3} quasi legalizzava il colpo di stato}. Voleva dare 
qualche ministero a Mussolini e far entrare nel governo i fascisti (sempre con l'idea che una volta
al potere, i fascisti si sarebbero costituzionalizzati). \textbf{Il 30 ottobre Mussolini arriva a
Roma ed entra al Quirinale con la camicia nera}.\\
Il primo governo è di coalizione. 5 ministeri sono dati ai Fascisti, \textbf{Mussolini} ha il 
\textbf{ministero dell'interno ed è presidente del consiglio}. C'erano anche liberali, popolari e
nazionalisti al governo. Il ministero della guerra era a Diaz (rassicura il re e l'esercito). A
novembre tiene il primo discorso alle camere, il \textbf{Discorso del Bivacco}. Non chiede la fiducia
in quanto il governo non è nato in parlamento ma nelle piazze. Voleva essere diverso dai liberali.
\textit{``Avrei potuto fare di quest'aula sorda e grigia un bivacco di manipoli''}, ovvero avrebbe
potuto fare un governo di soli fascisti, portando gli squadristi al Quirinale. \textit{``Avrei potuto
sprangare il parlamento. Avrei potuto ma non l'ho fatto, non ancora.''}. Soltanto PSU, PCI e PSI non
votano la fiducia al governo.\\ [\baselineskip]
A fine del 1922 viene istituito il \textbf{Gran consiglio del Fascismo} che non era previsto nello
Statuto. Il suo compito era quello di dare la direzione politica al governo, fatto solo di fascisti,
non era elettivo. Agli inizi del 1923 viene istituita la \textbf{Milizia volontaria per la Sicurezza
Nazionale}. Era la legalizzazione delle squadre. Dato che l'esercito temeva di essere soppiantato,
Mussolini fa sì che i miliziani giurino anche davanti al re.

\subsection{Riforme, elezioni e delitto Matteotti}
Nel 1923 viene fatta la riforma scolastica. \textbf{Gentile} era il ministro, oltre che filosofo
idealista. Vengono riorganizzati gli studi: i licei erano il vertice, d'élite che dovevano
formare la classe dirigente. Solo chi andava al liceo poteva andare all'università. Viene introdotta
la religione cattolica anche nelle elementari (su richiesta del PPI). Viene introdotto l'esame di
stato (richiesta del PPI, per equparare il titolo di studi di scuole pubbliche e private).\\
[\baselineskip]
Mussolini voleva assorbire le altre forze politiche all'interno del fascismo. Viene anche abolita la
nominalità dei titoli azionari (a favore del PPI). Così molti popolari aderiscono al fascismo, il
partito nazionalista si fonde per primo con il fascismo (Federzoni e Rocco).\\ [\baselineskip]
Luigi \textbf{Acerbo} nel 1923 scrive la legge elettorale. Prevedeva un forte premio di maggioranza
(con la maggioranza relativa si ha il 66\% dei deputati). Questo costringeva gli esponenti degli
altri partiti ad allearsi con il fascismo. Si formano così le \textbf{Liste Nazionali}.\\
Ad aprile del 1924 ci sono le prime elezioni, molto violente a causa della Milizia. I fascisti 
vincono con il 65\%.\\
In giugno si riunisce il parlamento e \textbf{Matteotti} (PSU) \textbf{denuncia apertamente le 
violenze}. Il parlamento era eletto illegalmente e chiedeva che il re sciogliesse il parlamento. Il
giorno successivo il Popolo d'Italia (giornale di Mussolini) minaccia Matteotti. Viene rapito qualche
giorno dopo e ad agosto viene ritrovato il cadavere. Questo fece molto scalpore e molti cambiarono
idea. \textbf{Mussolini è in difficoltà}. Cede il ministero dell'interno a Federzoni. Tiene a 
bada la milizia e prende tempo confidando sull'appoggio del re. Lascia anche che le indagini facciano
il loro corso e trovino i responsabili. Mussolini è responsabile ma non direttamente, verranno 
condannati ma per poco tempo.\\
C'erano forti opposizioni politiche: molti anti-fascisti decidono di non partecipare ai lavori del
parlamento. Si ritrano nella sala dell'\textbf{Aventino}. Essi sono PSU, PSI e liberali guidati da
Amendola. Volevano chiamare in causa il re per destituire Mussolini e fare nuove elezioni.\\
Giolitti e i suoi non partecipano. Dopo l'omicidio Giolitti diventa decisamente anti-fascista. Non
partecipa perché crede che la politica si faccia in parlamento, non fuori. Neanche il PCI partecipa
credendo che l'Aventino fosse inutile e che era da chiamare in causa il popolo.\\
\textbf{Il re asseconda Mussolini} in quanto prende in considerazione solo ciò che accade in 
parlamento. 

\subsection{Dittatura fascista}
Il \textbf{3 gennaio 1925} Mussolini fa il discorso in parlamento che dà inizio alla 
\textbf{dittatura fascista}. \textit{``Se il fascismo è stata un'associazione a delinquere, io ne
sono il capo''}, così si prende la responsabilità storica e politica delle azioni squadriste. È la
fine dello stato liberale. Tra il 1925 e il 2928 c'è la costruzione dello stato totalitario fascista.
Vengono firmate le \textbf{Leggi fascistissime}:
\begin{description}
  \item[Le proposte di legge dovevano essere approvate dal capo del governo prima che arrivino al
    parlamento] Non c'è divisione dei poteri, non è presidente del consiglio ma capo del governo
  \item[Tutti i partiti fuori legge tranne il PNF] Non c'è libertà politica
  \item[Tutte le organizzazioni sindacali fuori legge] Non c'è libertà di associazione. Confindustria
    fa l'accordo di Palazzo Vidoni dove gli industriali fanno accordi con i sindacati fascisti
  \item[Solo i giornali fascisti sono legali] Non c'è libertà di stampa. Mussolini inviava le
    \textit{veline} (ordini, direttive su cosa scrivere)
  \item[Viene istituito l'OVRA] Organizzazione per la Vigilanza e Repressione dell'Antifascismo
  \item[Istituito il Tribunale Speciale per la Difesa dello Stato] Tribunale politico che infliggeva
    anche la pena di morte
  \item[Definiti i poteri del Gran Consiglio] Il Gran Consiglio sceglie dei nomi e il re sceglie tra
    quelli. Sceglie inoltre i candidati dei deputati (400 nomi, gli elettori dicevano o sì o no, voto
    non segreto)
\end{description}
L'11 febbraio 1929 si firmano i \textbf{Patti Lateranensi} (Mussolini e Cardinale Gasparri 
(segretario di stato vaticano)). Sono divisi in 3:
\begin{itemize}
  \item Trattato internazionale: si riconoscevano recpirocamente
  \item Convenzione finanziaria: Mussolini pagava alla Chiesa quanto avrebbe pagato dalle Guarentige
    per i territori sottratti alla Chiesa
  \item Concordato (=accordo tra Chiesa ed uno stato):
    \begin{description}
      \item[Religione cattolica come religione di stato] 
      \item[Privilegi per il clero] No leva militare, \ldots
      \item[I sacerdoti apostati non potevano ricoprire cariche pubbliche] Gli ex sacerdoti, era
        retroattiva
      \item[Il matrimonio in Chiesa ha valore anche civile] Il sacerdote diventa un funzionario dello
        stato
      \item[La religione cattolica insegnata per legge fino ai licei] Diventa il fondamento e 
        coronamento dell'insegnamento pubblico
      \item[``Congrua'' ai sacerdoti] Una somma di denaro
      \item[Libertà d'azione religiosa, culturale ed educativa] purché non si occupasse di politica
    \end{description}
\end{itemize}
Mussolini sapeva di aver concesso molto alla Chiesa ma gli conveniva avere il sostegno dei cattolici.
Anche il papa lo definisce ``Uomo della provvidenza''.\\
Vengono fatte nuove elezioni con la legge elettorale, sono un successo strepitoso.
