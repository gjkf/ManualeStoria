%!TEX ROOT=storia.tex

\section{Unione Sovietica}
Nel 1922, Lenin si ammala, morirà due anni dopo. La lotta per la successione fu piuttosto dura ed era
sia sui programmi che per abmizione personale. Dopo Lenin, \textbf{Trotzkij} era il più prestigioso.
Poi però sarà Stalin a prevalere. Stalin non aveva le capacità oratorie di Lenin o Trotzkij ma era un
abile organizzatore. Infatti nel \textbf{1922 Stalin diventa Segretario Generale} del partito. Usò
questo potere anche in senso politico: essere iscritti al partito comunista voleva dire poter
avere posizioni di potere. Infatti molti giovani si iscriveranno, anche se non avevano fatto la
rivoluzione. Con il suo potere \textbf{Stalin favorisce l'ascesa di uomini a lui fedeli}.\\
La \textbf{NEP} era uno degli argomenti su cui si discuteva di più. Trotzkij era sempre stato
contrario, Stalin e Bucharin invece la sostenevano vedendola come una manovra temporanea. Un altro
argomento di divisione era la \textbf{burocratizzazione} della politica:
\begin{description}
  \item[Trotzkij] Voleva la \textit{rivoluzione permanente}, un potere in mano a pochi ma un continuo
    ricambio, il comunismo vero è mondiale, deve diffondersi (marxista)
  \item[Stalin] il comunismo può anche essere solo in un paese (va contro l'ideologia marxista ma è
    più realista), non c'erano le condizioni per provocare una rivoluzione.
\end{description}
Fra il 1924 e il 1925 Trotzkij si trova in minoranza nella derizione perché attorno a Stalin c'è una
coalizione per evitare che Trotzkij prenda il potere (si aveva paura diventasse troppo forte). Viene
quindi estromesso e così anche i suoi seguaci. Viene espulso dal partito e gli viene ritirata la
tessera. Nel 1929 viene espulso dall'URSS e vive in Europa, scrivendo libri. 10 anni dopo va in 
Messico dove dei sicari lo raggiungeranno.

\subsection{Stalin prende il potere}
Bucharin e gli altri si accorgono che Stalin sta prendendo troppo potere. Quindi \textbf{Kamenev e 
Zinov'ev} prendono l'idea trozkijana di abolire la NEP e vanno contro Stalin. Vengono estromessi 
entrambi. Fino al 1928 Stalin e Bucharin rimangono al potere.\\
Nel \textbf{1928 Stalin propone di togliere la NEP}: aveva fatto una valutazione politico-economica.
L'URSS era il paese più avanzato dal punto di vista politico ma economicamente non poteva competere.
In caso di guerra sarebbe stata debole e quindi in pericolo. Bisogna risollevare le industrie.
Bucharin era contrario, viene quindi estromesso.\\
Dal \textbf{1928} al 1932 viene messo in atto il \textbf{Primo Piano Quinquennale} di economia 
pianificata. In esso il governo controlla e pianifica gli obiettivi di produzione. 
\textbf{L'industrializzazione} era l'obiettivo, doveva reggere il confronto con i capitalisti. Era
favorita l'industria pesante (bellica, siderurgica, di estrazione, chimica, \ldots) tralasciando i
beni di consumo. Erano meno importanti militarmente e non facevano crescere l'economia. L'agricoltura
era un mezzo per industrializzarsi: si vendevano i prodotti della terra per acquistare macchinari.
I contadini vendevano il grano ad un prezzo stabilito dallo stato (c'era ancora un po' di proprietà
privata). Non era affatto conveniente per i contadini quindi nascondono i raccolti, l'armata rossa
interviene duramente. Tra il 1928 e il 1929 alcuni non coltivano più la terra. Lo stato non guadagna
abbastanza e quindi nel \textbf{1929} si modifica il piano e \textbf{si collettivizza la terra},
si volevano eliminare i kulaki (contadini ``ricchi'') e così si sterminarono nei gulag.\\
Dove c'erano sacche di resistenza si reprimevano duramente (in Ucraina venne causata volontariamente
una carestia). Il piano però portò anche i risultati che ci si aspettava: aumento del 50\% della
produzione industriale (solo di alcuni settori). Ci fu anche un crollo della produzione agricola 
però.\\
Dal \textbf{1933} al 1937 viene effettuato il \textbf{Secondo Piano Quinquennale} sulla falsariga
del primo. Dal \textbf{1938} in poi si attua il \textbf{Terzo Piano Quinquennale}. Entrambi questi
avevano l'obiettivo di sviluppare l'industria pesante dando però importanza anche all'agricoltura.
\textbf{Con l'inudstrializzazione nascono città industriali} vicino alle fabbriche e quindi era
necessaria maggiore produttività. Si vengono a creare due tipi di aziende
\begin{description}
  \item[Kolchoz] Cooperative con proprietà privata, le famiglie si dividevano i compiti. Avevano
    degli obiettivi di produzione
  \item[Sovchoz] Aziende agricole grandi sotto lo Stato, i lavoratori erano di fatto dipendenti.
\end{description}
Questa divisione del lavoro ebbe un forte sucecsso in quanto la produzione agricola andò ad 
aumentare.\\
Questa organizzazione si mantiene fino al 1991.

\subsection{Conseguenze della pianificazione}
Sicuramente attorno agli anni Quaranta l'URSS era tra le nazioni più avanzate in alcuni settori. Il
problema stava negli altri. Ad esempio \textbf{mancavano beni di consumo} e \textbf{il costo sociale
è stato altissimo} (erano vietati gli scioperi in quanto era come un boicottaggio dei piani, il 
lavoro era militarizzato, c'era solo il sindacato comunista, viene eliminata l'uguaglianza salariale
e chi produce di più guadagna di più).\\
Alcuni però aderivano in modo convinto. Celebre è l'esempio di Stackanov, un minatore da record che
diventa il simbolo del governo.\\ [\baselineskip]
La manodopera si sposta nelle città (circa 25 milioni di persone) e questo ha anche comportato un
peggioramento delle condizioni di lavoro nelle fabbriche.

\subsection{Il sistema totalitario}
Attorno agli anni Trenta Stalin fonda un vero e proprio \textbf{sistema totalitario}: monopartitismo,
controllo dello stato in ogni ambito, anche culturale. L'unica arte era il realismo socialista, le
avanguardie erano considerate arte degenerata. C'era \textbf{il culto del capo} (\textit{vodz}).
C'era la polizia politica,  l'\textbf{NKVD} (commissari del popolo per gli affari interni) che
generava paura e terrore. Stalin divenne famoso anche per le \textbf{purghe}. Dopo il 1934 ci furono
molti processi politici contro dirigenti del partito. La prima vittima fu \textbf{Kyrov} che aveva
ottenuto molti voti contro Stalin nelle elezioni. I processi si concludevano con confessioni, spesso
estorte sotto minaccia. Solo Bucharin non confessò. Nel 1938 più di 20 mila ufficiali vennero 
processati. Stalin trovava in loro un capro espiatorio di quando i piani non andavano bene.\\
Stalin perpetrò queste purghe anche attraverso i \textbf{Gulag}, dei campi di prigionia e lavoro
in cui le condizioni erano durissime. I kulaki furono le prime vittime. Avevano anche una funzione
economica, erano in zone minerarie, spesso impervie. L'essere così distanti dal centro di potere
serviva per tenere lontani gli oppositori politici. Prima di Stalin ci furono circa 100 mila 
prigionieri. Negli anni Trenta più di 2 milioni.
