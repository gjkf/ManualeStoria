%!TEX ROOT=storia.tex

\section{Unione Sovietica}
Nel 1922, Lenin si ammala, morirà due anni dopo. La lotta per la successione fu piuttosto dura ed era
sia sui programmi che per abmizione personale. Dopo Lenin, \textbf{Trotzkij} era il più prestigioso.
Poi però sarà Stalin a prevalere. Stalin non aveva le capacità oratorie di Lenin o Trotzkij ma era un
abile organizzatore. Infatti nel \textbf{1922 Stalin diventa Segretario Generale} del partito. Usò
questo potere anche in senso politico: essere iscritti al partito comunista voleva dire poter
avere posizioni di potere. Infatti molti giovani si iscriveranno, anche se non avevano fatto la
rivoluzione. Con il suo potere \textbf{Stalin favorisce l'ascesa di uomini a lui fedeli}.\\
La \textbf{NEP} era uno degli argomenti su cui si discuteva di più. Trotzkij era sempre stato
contrario, Stalin e Bucharin invece la sostenevano vedendola come una manovra temporanea. Un altro
argomento di divisione era la \textbf{burocratizzazione} della politica:
\begin{description}
  \item[Trotzkij] Voleva la \textit{rivoluzione permanente}, un potere in mano a pochi ma un continuo
    ricambio, il comunismo vero è mondiale, deve diffondersi (marxista)
  \item[Stalin] il comunusmo può anche essere solo in un paese (va contro l'ideologia marxista ma è
    più realista), non c'erano le condizioni per provocare una rivoluzione.
\end{description}
Fra il 1924 e il 1925 Trotzkij si trova in minoranza nella derizione perché attorno a Stalin c'è una
coalizione per evitare che Trotzkij prenda il potere (si aveva paura diventasse troppo forte). Viene
quindi estromesso e così anche i suoi seguaci. Viene espulso dal partito e gli viene ritirata la
tessera. Nel 1929 viene espulso dall'URSS e vive in Europa, scrivendo libri. 10 anni dopo va in 
Messico dove dei sicari lo raggiungeranno.

\subsection{Stalin prende potere}
Bucharin e gli altri si accorgono che Stalin sta prendendo troppo potere. Quindi \textbf{Kamenev e 
Zinov'ev} prendono l'idea trozkijana di abolire la NEP e vanno contro Stalin. Vengono estromessi 
entrambi. Fino al 1928 Stalin e Bucharin rimangono al potere.\\
Nel \textbf{1928 Stalin propone di togliere la NEP}: aveva fatto una valutazione politico-economica.
L'URSS era il paese più avanzato dal punto di vista politico ma economicamente non poteva competere.
In caso di guerra sarebbe stata debole e quindi in pericolo. Bisogna risollevare le industrie.
Bucharin era contrario, viene quindi estromesso.\\
Dal \textbf{1928} al 1932 viene messo in atto il \textbf{Primo Piano Quinquennale} di economia 
pianificata. In esso il governo controlla e pianifica gli obiettivi di produzione. 
\textbf{L'industrializzazione} era l'obiettivo, doveva reggere il confronto con i capitalisti. Era
favorita l'industria pesante (bellica, siderurgica, di estrazione, chimica, \ldots) tralasciando i
beni di consumo. Erano meno importanti militarmente e non facevano crescere l'economia. L'agricoltura
era un mezzo per industrializzarsi: si vendevano i prodotti della terra per acquistare macchinari.
I contadini vendevano il grano ad un prezzo stabilito dallo stato (c'era ancora un po' di proprietà
privata). Non era affatto conveniente per i contadini quindi nascondono i raccolti, l'armata rossa
interviene duramente. Tra il 1928 e il 1929 alcuni non coltivano più la terra. Lo stato non guadagna
abbastanza e quindi nel \textbf{1929} si modifica il piano e \textbf{si collettivizza la terra},
si volevano eliminare i kulaki (contadini ``ricchi'') e così si sterminarono nei gulag.\\
Dove c'erano sacche di resistenza si reprimevano duramente (in Ucraina venne causata volontariamente
una carestia). Il piano però portò anche i risultati che ci si aspettava: aumento del 50\% della
produzione industriale (solo di alcuni settori). Ci fu anche un crollo della produzione agricola 
però.
