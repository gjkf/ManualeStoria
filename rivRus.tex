%!TEX ROOT=storia.tex

\section{Rivoluzione Russa}
La Russia era il paese più vasto, un impero multietnico con più di 100 milioni di abitanti di cui
la metà russi, gli altri di varie etnie (Ucraini, Armeni, \ldots) con lingue e culture diverse. Non
sempre accettavano di buon grado il governo russo.\\
L'80\% della popolazione era contadina, erano analfabeti e solo nel 1861 era stata vieteta la
servitù della gleba. Solo alla fine dell'800 comincia un minimo di rivoluzione industriale (a San
Pietroburgo (la capitale), a Mosca (per i tessuti), negli Urali (per il ferro) e nel Mar Nero (per il
petrolio)). L'agricoltura era arretrata e i capitali per le industrie erano principalmente 
provenienti dall'estero. C'era poca borghesia e tanta nobiltà che non aveva la mentalità 
imprenditoriale.\\
Lo Zar era \textbf{Nicola \rom{2}} che regnava con un potere autocratico, senza costituzione, 
parlamento, diritti o libertà. La Chiesa ortodossa legittimava il potere dello zar.\\ [\baselineskip]
Nel 1905 era in \textbf{guerra con il Giappone}. La Russia perde e aumenta il malcontento. 
Manifestazioni di protesta e l'esercito le reprime con la forza. Continuano e lo zar concede la
\textbf{Duma}, un parlamento con potere legislativo, e libertà di stampa e associazione. Negli anni
seguenti però pian piano riduce i poteri alla Duma e riduce anche i diritti e il diritto di voto.\\
Nel 1914 arriva in guerra con circa 6 milioni di uomini. I più numerosi ma i peggio armati. I beni
di prima necessità scarseggiano sia al fronte che in città.\\
Nel \textbf{1916} lo zar convoca la Duma per ricevere sostegno per introdurre nuove tasse, la Duma si
oppone e viene sciolta. I leader politici si tengono in contatto.

\subsection{Inizio delle manifestazioni}
\textbf{23 febbraio 1917} a Pietrograto (= Pietroburgo) si tiene la prima manifestazione 
rivoluzionaria. Le successive manifestazioni vengono represse dall'esercito ma successivamente i
soldati sostengono le manifestazioni. Lo zar richiama dal fronte alcune truppe fedeli. Non arrivarono
mai in quanto bloccate dai ferrovieri, a favore della rivoluzione. Le manifestazioni si diffondono
fino a Mosca e al fronte; ad inizio marzo la situazione è fuori controllo.\\
I generali consigliano l'abdicazione dello zar, infatti nel \textbf{2 marzo 1917} lo zar abdica a
favore del fratello Michele che rinuncia al trono. In Russia termina così la dinastia Romanov. La
Duma elegge un governo provvisorio. Gli obiettivi di questo governo erano
\begin{description}
  \item[Continuare la guerra] Mantenere un legame con la Francia e l'Inghilterra, arrivare alla
    conquista
  \item[Democratizzare lo stato] Fare una costituzione che garantisca diritti e libertà. Si poteva
    fare solo dopo la fine della guerra
  \item[Promulgare una riforma agraria] Dare terra ai contadini
\end{description}
Il governo era sostenuto dai \textbf{cadetti} (liberali, democratici, borghesi russi) e dal
\textbf{partito social-rivoluzionario} (non è marxista, si richiamava alla tradizione russa che aveva
come base i contadini russi e i villaggi (Mir)) e i \textbf{menscevichi} (marxisti riformisti,
deve svilupparsi la politica come democrazia rappresentativa). Lenin nel frattempo era in esilio,
i capi bolscevichi erano divisi sul sostenere o meno il governo.\\
Accanto al governo si formano i \textbf{Soviet} ovvero dei consigli di fabbrica o di settore, eletti.
C'era il soviet della città che riunisce i rappresentanti dei locali soviet. Va contro l'idea
liberal-democratica rappresentativa. Il soviet voleva essere un esempio di \textbf{democrazia
diretta} anche se in realtà non era così.\\
Si ottiene così una situazione di diarchia: da un lato c'è il governo provvisorio della Duma, da un
altro i soviet.

\subsection{Lenin al potere}
Lenin torna a Pietrogrado grazie ai servizi segreti tedeschi. Tornato propone le \textbf{Tesi 
d'aprile} che in generale raccolgono il programma leninista. Lenin \textbf{vuole la pace subito}, il
governo (borghese) invece vuole continuare la guerra con obiettivi imperialisti.\\
La \textbf{Quarta tesi} è importante: i bolscevichi erano in minoranza nei soviet. Così i bolscevichi
vogliono prima prendere la maggioranza nei soviet e poi trasferire tutto il potere ai soviet,
esautorando il governo. Quindi Lenin vuole \textbf{fare la rivoluzione, subito} (andando un po'
contro Marx). Il modello era la Comune di Parigi. Lenin vuole nazionalizzare la terra senza 
indennizzo per i proprietari, così aumenta il consenso tra i contadini.\\
Un'altra tesi \textbf{voleva cambiare il nome del partito in ``Partito Comunista''}. L'ultima creava
la \textbf{Terza Internazionale} dei partiti rivoluzionari.\\ [\baselineskip]
Vengono proposte delle ffensive contro i tedeschi ma senza alcun successo (enormi diserzioni,
screditavano il governo, aumenta il malcontento). Nel luglio del 1917 ci sono \textbf{manifestazioni
a Pietrogrado}, il governo (Karenskij era a capo) risponde con la forza mettendo fuori legge il 
partito bolscevico.\\
Ad agosto il generale Kornikov voleva prendere il potere portando i propri soldati a Pietrogrado ma
i bolscevichi interrompono i collegamenti ferroviari e non ha successo. Da questo momento i 
bolscevichi soon organizzati.\\ [\baselineskip]
Tra \textbf{agosto e settembre 1917} i bolscevichi prendono la maggioranza nei soviet delle maggiori
città, il mese successivo l'obiettivo diventa il potere. Lenin propone il colpo di stato, Trotzkij
organizza le guardie rosse. \textbf{24--25 ottobre} (6--7 novembre) le guardie rose prendono il 
controllo delle vie di comunicazione e dei centri di potere. Assaltano il Palazzo d'Inverno in cui si
era riunito il governo provvisorio. I ministri fuggono.\\
Viene istituito il \textbf{Consiglio dei commissari del popolo} (eletti tramite democrazia diretta in
teoria). Furono stabiliti due decreti
\begin{description}
  \item[Sulla guerra] Appello ai paesi in guerra per interromperla senza indennità o spartizioni. 
    Pace incondizionata subito. Questo mette a favore i proletari
  \item[Sulla terra] Latifondisti espropriati senza indennizzo, creare aziende statali per 
    organizzare, in realtà i contadini si prendevano la terra e la amministravano privatamente. 
    Potava il favore dei contadini
\end{description}
\textbf{A gennaio del 1918 viene convocata l'assemblea costituente}. La maggioranza è ai socialisti 
rivoluzionari, i bolscevichi hanno circa un quarto dei voti. Il primo giorno i menscevichi e i
socialisti rivoluzionari criticano i bolscevichi, così l'assemblea viene sciolta la sera stessa
in quanto una forma di governo borghese. Si allontanano così dalla democrazia. I bolscevichi quindi
governano da soli con un minimo supporto della parte più esterma dei socialisti rivoluzionari.
Tortzkij tratta la pace con la Germania: \textbf{Trattatto di Brest-Litovsk}. La Russia perde un
quarto dei territori riconoscendo l'indipendenza dell'Ucraina, Estonia, Lettonia e Lituania.\\
[\baselineskip]
Lenin doveva fare un governo comunista da zero, con tutto il mondo ostile. Era fiducioso che la
rivoluzione fosse vicina anche in occidente. I primi provvedimenti che fa sono:
\begin{itemize}
  \item Nelle fabbriche l'organizzazione del lavoro la faceva il soviet
  \item Abolita la leva obbligatoria
  \item Uguaglianza nei confronti della legge
\end{itemize}

\subsection{La nascita dello stato sovietico e la guerra civile}
Il governo comunista era forte fuori dalle città principali dove invece le \textbf{armate bianche},
guidate da generali zaristi, vincevano. A luglio la famiglia dello zar e lo zar stesso vengono
fucilati. Le armate sono aiutate dalla comunità internazionale per viveri e denaro. Le truppe inviate
nel 1918 però erano per buona parte contadini che sentivano delle riforme nella russa comunista.\\
Il governo di Lenin è in pericolo, vuole fare un esercito e quindi reintroduce la leva obbligatoria.
Vengono anche richiamati degli ufficiali zaristi a guidare l'esercito. Trotzkij organizza
\textbf{l'Armata Rossa}. Le industrie belliche vengono fatte lavorare a pieno regime (anche se erano
gestite dai soviet, erano sottomesse alle direttive del governo). Nelle campagne i contadini 
nascondono i raccolti e li vendono al mercato nero perché il governo fissava i prezzi. Il governo
attua una \textbf{politica di requisizione}, portando via il raccolto. Lenin definisce questa 
politica \textbf{``Comunismo della guerra''}, sul modello tedesco, questa è la dittatura del 
proletariato, non l'autentico comunismo. Questo comporta anche l'abolizione di libertà di 
associazione politica (mono-partitismo). Viene istituita la CEKA, la polizia politica.\\
[\baselineskip]
Nel 1920 la Polonia attacca la Russia tentando di conquistare territori approfittandone della 
debolezza. Nulla di fatto. Durante la guerra civile era stata riconquistata l'Ucraina. \textbf{Si
ribella la base navale di Kronstadt} criticando il governo leninista in quanto troppo autoritario e 
non marxista. Viene mandata l'armata rossa, metà morti e gli altri imprigionati.\\
Nei primi mesi del 1921 \textbf{la guerra civile finisce} e in marzo si riunisce il decimo congresso
del partito comunista dove vengono prese due decisioni sull'organizzazione interna e sull'economia.
\begin{description}
  \item[Organizzazione interna] Viene condannato il frazionismo, non ci devono essere correnti di 
    partito
  \item[Economia] (NEP) La produzione industriale viene ridotta al 13\%, Lenin vuole reintrodurre
    alcune cose capitaliste (consentire agli imprenditore qualche dipendente (circa 10), consentire
    il commercio al dettaglio, consentire ai propietari terrieri di avere dipendenti). I livelli
    più alti dell'economia erano statali, quelli più bassi privati.
\end{description} 
Trotzkij era contrario in quanto rivedeva una rinascita della borghesia. Lenin sperava di migliorare
l'economia in ginocchio. Queste riforme durano fino al 1928.\\ [\baselineskip]
Nel 1924 viene fatta una \textbf{costituzione} (la più importante di tutte quelle che furono fatte).
Formalmente era una repubblica federale (nasce l'\textbf{URSS}), in realtà non era così. Il 
parlamento è il congresso dei soviet, era una dittatura del parrtito.\\
L'ultima tesi di Aprle voleva una Terza Internazionale, a Mosca viene istituito il \textbf{Cominter},
l'internazionale comunista. Anche partiti europei entrano a farne parte. Si poteva solo se si 
cambiava nome in ``Partito Comunista'' e avere come modello il leninismo. I riformisti inoltre
vanno espulsi dal partito.
