%!TEX ROOT=storia.tex

\section{Rivoluzione Russa}
La Russia era il paese pi� vasto, un impero multietnico con pi� di 100 milioni di abitanti di cui
la met� russi, gli altri di varie etnie (Ucraini, Armeni, \ldots) con lingue e culture diverse. Non
sempre accettavano di buon grado il governo russo.\\
L'80\% della popolazione era contadina, erano analfabeti e solo nel 1861 era stata vieteta la
servit� della gleba. Solo alla fine dell'800 comincia un minimo di rivoluzione industriale (a San
Pietroburgo (la capitale), a Mosca (per i tessuti), negli Urali (per il ferro) e nel Mar Nero (per il
petrolio)). L'agricoltura era arretrata e i capitali per le industrie erano principalmente 
provenienti dall'estero. C'era poca borghesia e tanta nobilt� che non aveva la mentalit� 
imprenditoriale.\\
Lo Zar era \textbf{Nicola \rom{2}} che regnava con un potere autocratico, senza costituzione, 
parlamento, diritti o libert�. La Chiesa ortodossa legittimava il potere dello zar.\\ [\baselineskip]
Nel 1905 era in \textbf{guerra con il Giappone}. La Russia perde e aumenta il malcontento. 
Manifestazioni di protesta e l'esercito le reprime con la forza. Continuano e lo zar concede la
\textbf{Duma}, un parlamento con potere legislativo, e libert� di stampa e associazione. Negli anni
seguenti per� pian piano riduce i poteri alla Duma e riduce anche i diritti e il diritto di voto.\\
Nel 1914 arriva in guerra con circa 6 milioni di uomini. I pi� numerosi ma i peggio armati. I beni
di prima necessit� scarseggiano sia al fronte che in citt�.\\
Nel \textbf{1916} lo zar convoca la Duma per ricevere sostegno per introdurre nuove tasse, la Duma si
oppone e viene sciolta. I leader politici si tengono in contatto.\\ [\baselineskip]
\textbf{23 febbraio 1917} a Pietrograto (= Pietroburgo) si tiene la prima manifestazione 
rivoluzionaria.
