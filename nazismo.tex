%!TEX ROOT=storia.tex

\section{Nazismo}
Fino alla crisi del 1929, il partito nazista non aveva preso molti voti. Al tempo \textbf{Bruning} 
era il cancelliere (CDU). La politica che si stava attuando era di stampo liberista che però non
risolveva la crisi economica, anzi. Ci sono nel \textbf{1930 nuove elezioni}. SPD e CDU perdono voti
a favore dei comunisti e nazisti. In pochi anni era stata la terza crisi economica, il popolo aveva
perso fiducia.\\
I comunisti acquistano i voti degli operai che abbandonano l'SPD, i nazisti invece prendono i voti
dai borghesi che avevano accettato con riserva la repubblica di Weimar, la crisi era una 
dimostrazione del suo fallimento e intimoriti dalla rivoluzione rossa, votano per l'estrema destra.\\
Il \textbf{1932} fu l'anno più duro della crisi. Ci furono 3 elezioni nel giro di poco tempo
\begin{enumerate}
  \item \textbf{Presidenziali}: Hitler si candida, l'SPD si allea con la CDU contro Hitler. 
    \textbf{Hindemburg rieletto}.
  \item \textbf{Politiche}: il partito nazista diventa il primo partito con quasi il 37\%. Hindemburg
    non vuole dare il governo a Hitler e tenta di formare altri governi, troppo deboli.
  \item \textbf{Politiche}: a novembre non ci sono grandi cambiamenti. Hindemburg cede alle pressioni
    dei nazisti, degli industriali e dell'esercito. \textbf{1/1/1933, Hitler cancelliere}. 
\end{enumerate}
Il governo che nascerà sarà di coalizione, ci saranno ministri cattolici ma non socialisti.

\subsection{Nascita del sistema totalitario}
\textbf{Il 22 febbraio 1933} il parlamento viene incendiato. I nazisti incolpano i socialisti e 
mettono fuori legge il partito comunista. I leader o fuggono o sono rinchiusi a Dachau.\\
\textbf{A marzo 1933} ci sono nuove elezioni (Hitler al 44\%), il 23 marzo \textbf{Hitler fa votare 
la legge dei Pieni Poteri}. Il governo ha non solo il potere esecutivo ma anche legislativo. Il 
governo aveva il potere di cambiare la costituzione. La CDU si autoscioglie e lo stesso faranno anche
altri partiti, si sperava che Hitler avrebbe riportato l'ordine.\\
A giugno del 1933 è a tutti gli effetti un sistema totalitario: monopartitismo, monosindacalismo,
controllo della stampa e dei mezzi di comunicazione.\\
Hitler deve prendere il controllo delle \textbf{SA} (con a capo Röhm) che lo avevano aiutato durante 
le elezioni. Stavano diventando un problema in quanto l'esercito temeva di essere surclassato e gli
industriali erano intimoriti dalla loro propaganda (volevano completare anche gli aspetti sociali
del programma del 1920 che ormai Hitler aveva eliminato). Si giunge al \textbf{30 giugno 1934},
denominata la \textbf{notte dei lunghi coltelli}: le SS catturano i capi delle SA e li fa uccidere.
(Le SS erano nate dopo il 1920 come guardie del corpo di Hitler).\\
Nell'\textbf{agosto del 1934} Hindemburg muore, \textbf{Hitler unisce la carica di cancelliere e di
presidente della repubblica}. Hitler ora ha tutto il potere nelle sue mani, tutto basato sul culto
del capo, del Führer.\\
Nonostante tutto aveva un grande consenso fra il popolo (aveva abolito le libertà e controllava i
mezzi di comunicazione (Goebbels era il ministro della propaganda)). Questo perché Hitler aveva
assoggettato la cultura: bisognava essere iscritti al partito per esporre, pubblicare, fare tutto.
La società era controllata in modo capillare anche dalla \textbf{Gestapo} (la polizia segreta, 
Himmler ne era a capo, era capo anche delle SS).

\subsection{Pianificazione e politica estera}
Dal 1933 l'economia si riprende, non solo in Germania ma in tutto il mondo. Con il Nazismo da una
politica liberista si è passati ad un forte intervento del governo con commesse interne (viene
fondata la Volkswagen). Si dà così lavoro ai disoccupati.\\
Nel \textbf{1936} c'è il \textbf{Piano Quadriennale} il cui obiettivo era quello di mettere la 
Germania in condizione di fare la guerra (per riprendersi i territori, lo spazio vitale). Non si
statalizzano le industrie, lo Stato è il committente e stabilendo gli obiettivi ci si accordava sui
prezzi, favorevoli agli industriali. Viene abolito lo sciopero e ogni sindacato non nazista. I salari
diminuiscono. Nonostante ciò aumentano gli occupati, e quindi il consenso del governo. In questo modo
lo stato si indebitava. Hitler non poteva andare vanti così, voleva accelerare una guerra anche
perché una vittoria avrebbe aiutato l'economia. I generali erano più cauti.\\
Questa sua politica ebbe forti opposizioni da parte dei socialisti e comunisti,da esponenti cattolici
e protestanti. \textbf{Nel luglio del 1933} la Chiesta stipula un \textbf{concordato} con i nazisti.
(Eugenio Pacelli è il firmatario, che poi diventerà Papa Pio \rom{12}). La Chiesa vuole mantenere 
libertà d'azione, accettando lo scioglimento del partito.

\subsection{Antisemitismo}
Sin dall'inizio c'era una matrice antisemita. Sin dal 1933 cominciano provvedimenti. C'erano circa
\textbf{500 mila} ebrei su 60 milioni di tedeschi. Qualcuno era ricco, ma erano pochi, erano 
borghesi. La comunità più importante era a Berlino (circa 200 mila). Molti avevano combattuto e 
ottenuto riconoscimenti al valor militare.\\
\textbf{Dal 1933 al 1939} l'obiettivo era costringere gli ebrei ad emigrare. Si diceva di non andare
nei negozi ebrei, si boicottavano. Non potevano assumere cariche pubbliche (anche studenti).
\textbf{Nel settembre 1935} si promulgano le \textbf{Leggi di Norimberga}. Erano due leggi:
\begin{description}
  \item[Cittadinanza] Solo chi era di sangue tedesco era tedesco
  \item[Difesa del sangue tedesco] Vietati i rapporti sessuali tra tedeschi e non tedeschi
\end{description}
Il \textbf{9 novembre 1938} c'è la così detta \textbf{notte dei lunghi coltelli} dove vennero
perpretrate violenze contro gli ebrei. La giustificazione era un attentato a Parigi contro un
diplomatico tedesco ucciso da un ebreo. Decine di morti oltre alla devastazione materiale. Fu
presentato come un moto spontaneo del popolo ma era stato organizzato dalle SS e dalla Gestapo.\\
Nel 1939 erano rimasti circa la metà degli ebrei. Non tutti potevano scappare, altri avevano 
sottovalutato i tempi. Con la guerra ovviamente le condizioni peggiorano. E non sono solo gli ebrei
tedeschi ad essere colpiti ma anche quelli delle nazioni conquistate.\\
Nel \textbf{settembre 1939} comincia la guerra, i tedeschi conquistano molti territori e quindi
incontrano ebrei, ghettizzati, fucilati o costretti a lavorare nelle fabbriche tedesche. Nel
\textbf{giugno 1941} c'è l'\textbf{Operazione Barbarossa} per la conquista dell'URSS, altri ebrei.\\
Vengono creati reparti speciali delle SS per catturare i dirigenti del partito comunista e gli ebrei
e fucilarli. Questo sistema non poteva essere portato avanti e quindi \textbf{il 20 gennaio 1942} 
si riuniscono a Vannsee su proposta di Göring. Partecipano i dirigenti SS (Eichmann, non Himmler).
Si dovevano catturare tutti gli ebrei (circa 11 milioni), imprigionarli e farli lavorare per il Reich
tedesco. Molti sicuramente sarebbero morti, i superstiti erano i più pericolosi ed erano quindi da
eliminare. Vengono così \textbf{istituiti i campi di sterminio} (Treblinka, Auschwitz erano i 
principali)
