%!TEX ROOT=storia.tex

\section{\textit{``Origini del Totalitarismo''}, A.~Arendt}
Un sistema totalitario ha le solite caratteristiche: no libertà, controllo della stampa e della vita,
monopartitismo, \ldots C'è un'ideologia ufficiale, una verità assoluta. Il terrore è un mezzo di 
controllo. I regimi hanno bisogno di un \textbf{nemico oggettivo}, una persona, un gruppo la cui
sola esistenza è considerata un pericolo per la nazione. Si utilizza questo nemico per giustificare
le azioni. I regimi vogliono \textbf{creare un nuovo tipo di uomo} (soldati per il fascismo,
ariani per il nazismo e socialisti per lo stalinismo).\\
Il fascismo non è fatto rientrare nei sistemi totalitari. Viene considerato da De Felice un
\textbf{totalitarismo imperfetto} in quanto condivide molti caratteri generali ma con delle 
limitazioni: non ha avuto un così completo controllo della società in quanto c'erano due istituzioni
che non poteva controllare: la \textbf{monarchia} (il re era a capo dello stato e delle forze armate)
e la \textbf{chiesa} (con i Patti Lateranensi c'è un accordo). Non poteva assoggettare le due 
istituzioni.
