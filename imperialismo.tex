%!TEX ROOT=storia.tex

\section{Imperialismo}
In questo periodo di espansione economica si nota anche un'espansione coloniale. Più precisamente
avviene il fenomeno dell'\textbf{imperialismo}. Dalla fine dell'800 si attua una politica di
potenza coloniale che aveva come principali cause economiche (avere un mercato dove vendere i
propri prodotti, nuove materie prime, più mano d'opera, nuovi contratti statali, \ldots). Secondo
Lenin ``L'Imperialismo è la fase suprema del capitalismo''. L'imperialismo è quindi una causa 
dell'economia. Nascono da questo i movimenti \textbf{nazionalisti} non solo per questioni economiche
ma anche politiche (più territori si controllano, più si è prestigiosi) e militari. Alcuni movimenti
nazionalisti sfociano nel razzismo e nell'anti-semitismo.\\ [\baselineskip]
L'impero più grande era quello inglese (possedeva $\sfrac{1}{4}$ delle terre emerse e $\sfrac{1}{4}$
della popolazione). Quello francese era secondo ma meno ricco. Poi venivano tutti gli altri.\\
L'\textbf{Africa} era la nuova terra di conquista. Nel 1885 la spartizione era stata fatta a tavolino
su proposta di Bismark. Le spartizioni non tenevano minimamente conto delle popolazioni. 
L'Inghilterra voleva collegare Egitto e Sud Africa, la Francia voleva andare ad est (Marocco e 
Algeria), la Germania il Belgio e l'Italia quello che rimaneva. In \textbf{Asia} l'Inghilterra
ha l'India e la Birmania, la Francia l'Indonesia. La Cina non è stata conquistata perché non ci
furono accordi a riguardo. Il Giapppone ha anche lui un impero (Corea). La Russia si espande verso
est fino al Giappone e a sud fino all'Afghanistan. Anche gli Stati Uniti, nati come stato coloniale
si espandono verso l'America centro-meridionale. Spacciavano le conquiste come ``liberazioni''. Gli
USA aiutano Cuba con l'indipendenza dalla Spagna però scrivono loro la costituzione e tengono le
basi militari. Fanno lo stesso a Puerto Rico e nelle Filippine. Fanno nascere un movimento di rivolta
a Panama e nasce lo stato panamense. Gli USA hanno il controllo del canale per un secolo.
