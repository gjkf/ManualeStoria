%!TEX ROOT=storia.tex

\section{Et� Giolittiana}
L'et� giolittiana va dal 1900 al 1914.

\subsection{Economia}
Il periodo � caratterizzato da generale crescita economica. L'industrializzazione cresce ma solo in
alcune zone (Lombardia, Piemonte, Liguria) e solo alcuni settori.

\subsection{Societ�}
C'� una fortissima emigrazione (circa 500k all'anno).

\subsection{Politica}
Giolitti era un piemontese liberale. La formazione dei sindacati era inevitabile in quanto � una 
tendenza causata dall'industrializzazione. Lo stato \textbf{non deve impedire l'organizzazione}
perch� altrimenti si organizzano clandestinamente contro lo Stato. Non deve reprimere manifestazioni
pacifiche.\\
Giolitti cerc� alleanze con socialisti (offre a Turati un ministero se il PSI si fosse alelato con
il governo, rifiuta per non dividere il partito). Quando Giolitti era presidente del Consiglio, era
anche ministro dell'Interno.\\
Nel 1906 � fondata la \textbf{CGL} legata al partito socialista. Guidata da socialisti riformisti.\\
Nel 1910 � fondata la \textbf{Confindustria}.\\
Giolitti ha portato avanti importanti riforme tra cui le \textbf{prime leggi per regolamentare il 
lavoro} (obbligo del riposo festivo, vietato per donne e bambini il lavoro notturno). Nel 1911 viene 
creata l'INA (Istituto Nazionale Assicurazioni) a cui � dato il monopolio delle assicurazioni sulla 
vita. In questo modo i lavoratori erano pi� sicuri e i fondi andavano a formare un sistema 
previdenziale. Nel 1913 viene data una pensione agli infortunati sul lavoro.\\
Vengono \textbf{nazionalizzate le ferrovie} cos� si sarebbe risparmiato e si sarebbero collegati 
anche i punti pi� sfavorevoli. Fa anche riforme per il sud che dovevano favorire lo sviluppo
(costruito un acquedotto in Puglia, \ldots), in realt� non ebbero grandi risultati in quanto i
provvedimenti erano clientelari (favoritismi, \ldots).\\ [\baselineskip]
Nel 1912 � stata varata una \textbf{riforma elettorale} che permetteva il suffragio universale 
maschile per chi avesse avuto 21 anni e fato la leva militare o 30 altrimenti. Circa 9 milioni di
elettori. Uninominale a doppio turno (1 deputato per collegio, 50\% dei voti al primo turno, 
ballottaggio dei primi due). Nel 1913 le prime elezioni di massa. I socialisti erano organizzati per 
massa, non i liberali. Cos� si form� il \textbf{Patto Gentiloni} che sanciva che i candidati liberali
cattolici sarebbero stati sostenuti dalla Chiesa se poi in parlamento non avessero sostenuto 
provvedimenti che la Chiesa riteneva scomodi (divorzio, scuole cattoliche, \ldots). La Chiesa temeva
i socialisti, viene cos� eliminato il ``Non Expedit'' e i cattolici entrano nella vita dello stato
italiano.

\subsection{Esteri}
Tra il 1911 e il 1912 Giolitti intraprende una guerra coloniale. Furono presi accordi segreti con la 
Francia: l'Italia permette concede il Marocco alla Francia, la Francia non ostacola l'Italia. La
Chiesa sosteneva la guerra come fosse di civilt�. La guerra fu durissima, quasi barbara 
(avvelenamenti, capi di concentramento).\\
Nel 1912 si stipula la \textbf{Pace di Losanna}. La Libia ora � colonia Italiana. La Libia era 
allettante per l'economia secondo Giolitti, non tutti erano d'accordo (Sanvemini disse che la Libia
era una ``Scatola di Sabbia'').
